\documentclass[10pt,openright,oneside,CJKmath]{MyBook}
\usepackage{layout}
\usepackage{annotate-equations}

\tikzset{annotate equations/arrow/.style={color=ForestGreen, >=latex', very thick, dashed}}
\renewcommand{\eqnannotationfont}{\sffamily\large}
\usepackage{amsmath, amssymb}
\usepackage{longtable,booktabs,calc}
\begin{document}

\mainmatter

\setchapterimage{j-20}


\lstdefinestyle{cpp}{
  language=C++,
  basicstyle=\ttfamily\small,
  keywordstyle=\color{blue},
  stringstyle=\color{teal},
  commentstyle=\color{gray},
  tabsize=4,
  showstringspaces=false,
  breaklines=true,
  frame=single
}
\chapter{动画基础}
\section{UAnimInstance 派生类的并行更新实现思路}

\subsection{核心结论摘要 (TL;DR)}
\begin{itemize}
  \item 你无法也不需要手动把 \texttt{NativeUpdateAnimation} 挪到线程池;并行是引擎的“\emph{动画评估} (Evaluation) / 更新 (Update)”阶段自动调度。
  \item Game Thread 负责:收集角色/移动等状态 $\rightarrow$ 拷贝到 \texttt{FAnimInstanceProxy}。
  \item Worker 线程负责:运行动画图节点更新 + 姿势计算(使用 Proxy 中的只读数据副本)。
  \item 关键是:自定义 \texttt{FAnimInstanceProxy},并保证线程安全的数据传递(值复制,不在工作线程访问 Actor/World)。
\end{itemize}

\subsection{引擎并行动画流程概览}
典型一帧相关调用链(简化):

\ci{SkeletalMeshComponent Tick (GT)} $\rightarrow$ \ci{UAnimInstance::UpdateAnimation (GT)} $\rightarrow$
\ci{复制状态进 Proxy (GT)} $\rightarrow$ \ci{排任务} $\rightarrow$ \ci{Proxy::Update / Evaluate (Worker)} $\rightarrow$
\ci{姿势回写 (GT/Render)}

\subsection{触发并行的前提}
\begin{enumerate}
  \item 组件层标志(不同版本命名略有差异):
  \begin{itemize}
    \item \ci{USkeletalMeshComponent::bAllowParallelUpdateAnimation = true}
    \item \ci{USkeletalMeshComponent::bParallelAnimationEvaluation = true}(或旧字段 \ci{bUseParallelEvaluation})
  \end{itemize}
  \item 项目设置里未关闭并行动画(Project Settings $\rightarrow$ Engine $\rightarrow$ Animation)。
  \item 没有在动画实例里做阻塞/非线程安全操作(例如蓝图里直接做 LineTrace)。
  \item 足够的动画实例数量或开销,调度器才会分派任务到线程池(单个很轻的实例可能仍在主线程)。
\end{enumerate}

\subsection{Proxy 的作用}
\textbf{目的}:隔离 Game Thread(收集来源数据)和 Worker Thread(计算姿势)之间的线程安全数据边界。  

\textbf{特点}:
\begin{itemize}
  \item 只保存 POD \footnote{POD = “Plain Old Data”,即没有复杂构造/析构、无虚函数、标准布局、可按字节安全拷贝的老式纯数据类型;在多线程和性能敏感路径用来保证简单、可预测、低成本的数据传递。}/ 线程安全结构(\texttt{float}, \texttt{bool}, \texttt{FVector}, \texttt{FRotator}, 简单数组等)。
  \item 不持有或访问 \texttt{AActor*}, \texttt{UObject*} 等非线程安全对象(除非只是初始化时一次性缓存且不在 Worker 中解引用——一般仍不推荐)。
  \item 可重写 \texttt{Update()}(对数据做纯数学运算)、可选 \texttt{Evaluate()}(自定义姿势混合,需要熟悉 Pose Pipeline)。
\end{itemize}

\subsection{示例:自定义 Proxy}
\begin{code}[caption={FAuraAnimInstanceProxy}]
#pragma once
#include "Animation/AnimInstanceProxy.h"

struct FAuraAnimInstanceProxy : public FAnimInstanceProxy
{
    FAuraAnimInstanceProxy(UAnimInstance* Instance)
        : FAnimInstanceProxy(Instance)
    {}

    // 线程安全副本
    float GroundSpeed = 0.f;
    bool  bShouldMove = false;

    void SetMovementData(float InSpeed, bool bInMove)
    {
        GroundSpeed = InSpeed;
        bShouldMove = bInMove;
    }

    virtual void Update(float DeltaSeconds) override
    {
        // 可在此基于已复制数据做一些派生计算(勿访问世界/Actor)。
        FAnimInstanceProxy::Update(DeltaSeconds);
    }

    // 如需自定义姿势求值:
    // virtual void Evaluate(FPoseContext& Output) override { ... }
};
\end{code}

\subsection{修改 UAnimInstance 派生类}
在你的 \texttt{UAnimInstBase}(见现有的 \texttt{AnimInstBase.h})添加 Proxy 支持:

\subsection*{头文件补充}
\begin{code}[caption={AnimInstBase.h 关键补充}]
#include "AuraAnimInstanceProxy.h"

UCLASS()
class AURA_API UAnimInstBase : public UAnimInstance
{
    GENERATED_BODY()
public:
    virtual void NativeInitializeAnimation() override;
    virtual void NativeUpdateAnimation(float DeltaSeconds) override;

protected:
    // ...existing code...
    // 创建与销毁 Proxy
    virtual FAnimInstanceProxy* CreateAnimInstanceProxy() override
    {
        return new FAuraAnimInstanceProxy(this);
    }
    virtual void DestroyAnimInstanceProxy(FAnimInstanceProxy* InProxy) override
    {
        delete InProxy;
    }

    FAuraAnimInstanceProxy& GetAuraProxy()
    {
        return *static_cast<FAuraAnimInstanceProxy*>(GetProxyOnGameThread());
    }
};
\end{code}

\subsection*{CPP 实现示例}
\begin{code}[caption={AnimInstBase.cpp 示例}]
void UAnimInstBase::NativeInitializeAnimation()
{
    Super::NativeInitializeAnimation();
    AuraCharacter = Cast<AAuraCharacterBase>(TryGetPawnOwner());
    if (AuraCharacter)
    {
        CharacterMovement = AuraCharacter->GetCharacterMovement();
    }
}

void UAnimInstBase::NativeUpdateAnimation(float DeltaSeconds)
{
    Super::NativeUpdateAnimation(DeltaSeconds);

    if (!AuraCharacter)
        AuraCharacter = Cast<AAuraCharacterBase>(TryGetPawnOwner());
    if (!CharacterMovement && AuraCharacter)
        CharacterMovement = AuraCharacter->GetCharacterMovement();
    if (!CharacterMovement)
        return;

    GroundSpeed = CharacterMovement->Velocity.Size2D();
    bShouldMove = GroundSpeed > 2.f &&
                  !CharacterMovement->GetCurrentAcceleration().IsNearlyZero();

    // 线程安全复制
    GetAuraProxy().SetMovementData(GroundSpeed, bShouldMove);
}
\end{code}

\subsection{线程安全守则}
\begin{itemize}
  \item \textbf{允许(放 Proxy)}:数值缓存、状态标志、已解算出的 Blend 权重、数学预计算结果。
  \item \textbf{禁止(Worker 中做)}:访问 \texttt{GetWorld()}, 执行 Trace、修改 Gameplay Ability、广播动态多播委托、访问非线程安全容器。
  \item \textbf{不要} 在 Proxy 中 \cii{new/delete UObject} 或修改全局单例。
\end{itemize}

\subsection{可覆盖的并行策略函数(可选)}
\begin{itemize}
  \item \texttt{bool UAnimInstance::ShouldUseParallelUpdateAnimation() const}:强制允许/禁止并行 Update。
  \item \texttt{bool USkeletalMeshComponent::CanUseParallelEvaluation() const}:组件层逻辑判定。
\end{itemize}

\subsection{验证并行是否生效}
\begin{enumerate}
  \item 运行 \texttt{stat anim},观察 “Parallel Evaluation” 任务。
  \item 用 \texttt{stat taskgraph} 检查是否出现动画任务分派。
  \item 多放几个角色实例,确保有足够工作量促使并行。
  \item 在 Proxy::Update 加轻量日志(注意:过多日志会反向影响并行)。
\end{enumerate}

\subsection{常见问题与排查}
\begin{tabular}{p{4.2cm} p{10cm}}
\textbf{现象} & \textbf{可能原因/解决} \\
未见并行任务 & 规模太小;未启用组件标志;引擎判定内联执行 \\
Worker 崩溃 & 在 Proxy::Update/Evaluate 访问了 Actor/World/ASC \\
姿势不同步 & 数据未在同一帧复制(Game Thread 修改但没 Set 到 Proxy) \\
性能无提升 & 动画复杂度低;线程调度开销抵消收益 \\
\end{tabular}

\subsection{与 GameplayAbility 协同注意}
\begin{itemize}
  \item 能力触发(GAS)仍在 Game Thread:不要在 Proxy 里直接调用 ASC。
  \item 如果动画蓝图需要显示 GA 状态:在 Game Thread 读取 ASC Tag/Attribute $\rightarrow$ 映射成简单 bool/float $\rightarrow$ 复制进 Proxy。
  \item 避免在蓝图动画节点图内部直接访问大量 GAS 数据(会放大锁与同步成本)。
\end{itemize}

\subsection{优化建议(根据需求择用)}
\begin{enumerate}
  \item 将复杂的“基于速度状态机决策”留在图节点;仅把最原始原始输入(速度、是否欲移动)复制给 Proxy。
  \item 大量(百+)角色时:确保动画蓝图中没有每帧动态创建对象或高分支复杂蓝图逻辑。
  \item 如需自定义 Evaluate:谨慎操作 Pose(拷贝/Blend)避免额外内存分配。
\end{enumerate}

\subsection{最小工作清单 (Checklist)}
\begin{enumerate}
  \item 创建 \texttt{FAuraAnimInstanceProxy}。
  \item 在 \texttt{UAnimInstBase} 覆盖 \texttt{CreateAnimInstanceProxy / DestroyAnimInstanceProxy}。
  \item 在 \texttt{NativeUpdateAnimation} 读取 Movement $\rightarrow$ 同步到 Proxy。
  \item 启用组件并行动画相关标志。
  \item 使用 \texttt{stat anim} 验证。
\end{enumerate}

\subsection{可扩展方向}
\begin{itemize}
  \item \textbf{预测性运算}:在 \cii{Proxy::Update} 中做曲线或 Blend 参数预计算,减少 Game Thread 蓝图节点开销。
  \item \textbf{分层数据}:对多种姿势条件(Combat/Idle/Cast)压缩为 bit mask 再放 Proxy。
  \item \textbf{共享数据缓存}:对全局静态(只读)表格/曲线放到线程安全共享结构(例如 \texttt{TSharedPtr<const FData>}),Worker 线程只读访问。
\end{itemize}

\subsection{总结语句便于记忆}
\[
{“GT 收集  \rightarrow Proxy 拷贝  \rightarrow Worker 评估  \rightarrow GT 提交”}
\]
\[
{核心:数据下沉为值;避免跨线程指针访问;让引擎自己并行。}
\]
\section{问题重述}
已经实现了自定义 \cii{AnimInstanceProxy},希望把在 Proxy 中维护或计算的数据提供给动画蓝图(Anim Blueprint)的图节点(状态机、过渡条件、Blend 等)使用。应如何设计数据流?

\subsection{核心结论}
\begin{itemize}
  \item 动画蓝图直接读取的是 \textbf{AnimInstance(UAnimInstance 派生类)上的 UPROPERTY 变量},而不是 Proxy 的内部成员。
  \item 典型的数据方向是:Game Thread 在 \cii{NativeUpdateAnimation()} 更新实例变量;引擎把这些简单值复制到并行任务使用的结构中。
  \item 如果某个值只在 Proxy 的并行阶段(\cii{Proxy::Update / Evaluate})里计算,则只能在 \textbf{本帧结束后(回到 Game Thread)} 再拷回 AnimInstance,供下一帧蓝图使用(存在一帧延迟)。
\end{itemize}

\subsection{数据流类型区分}
\subsubsection{单向(最常见)}
Game Thread:
\[
\text{NativeUpdateAnimation()} \Rightarrow \text{更新 AnimInstance UPROPERTY}
\]
Worker 线程:
\[
\text{并行评估读取“冻结”快照(引擎内部复制)}
\]
无需从 Proxy 回拷。

\subsubsection{双向(较少,需要回拷)}
Worker 线程里计算出的派生值(昂贵或需多线程):
\[
\text{Proxy::Update/Evaluate} \Longrightarrow \text{(存放 Proxy 成员)}
\]
帧末 Game Thread:
\[
\text{PostUpdateAnimation()} \Longrightarrow \text{拷贝到 AnimInstance 变量}
\]
下帧蓝图使用。

\subsection{为何不是实时回写}
并行评估任务在 Worker 线程运行时,Anim Graph 的本帧求值已经在进行中;不能在同一帧插入“回写后再重新评估”。因此 Proxy 结果只影响下一帧。

\subsection{推荐模式}
\begin{enumerate}
  \item \textbf{能在 \ci{NativeUpdateAnimation} 计算的就不要放 Proxy}: 比如速度、是否移动、是否在空中。
  \item \textbf{Proxy 只保存多线程只读副本或昂贵派生值}。
  \item \textbf{需要蓝图使用的派生值}: 在 \cii{PostUpdateAnimation()} 回拷(或合适的后置钩子)。
\end{enumerate}

\subsection{示例:Proxy 内计算 \(\rightarrow\) 下一帧蓝图读取}
\subsubsection*{Proxy 定义}
\begin{lstlisting}[style=cpp]
struct FAuraAnimInstanceProxy : public FAnimInstanceProxy
{
    FAuraAnimInstanceProxy(UAnimInstance* InInstance)
        : FAnimInstanceProxy(InInstance) {}

    float GroundSpeed = 0.f;
    bool  bShouldMove = false;

    float ComputedLeanAngle = 0.f; // 派生值示例

    void SetMovementData(float InSpeed, bool bInShouldMove)
    {
        GroundSpeed = InSpeed;
        bShouldMove = bInShouldMove;
    }

    virtual void Update(float DeltaSeconds) override
    {
        FAnimInstanceProxy::Update(DeltaSeconds);
        // 纯数学运算(线程安全)
        ComputedLeanAngle = /* 例如基于速度/方向的倾斜角 */;
    }

    float GetComputedLeanAngle() const { return ComputedLeanAngle; }
};
\end{lstlisting}

\subsubsection*{AnimInstance 中声明可供蓝图读取的变量}
\begin{lstlisting}[style=cpp]
UPROPERTY(VisibleAnywhere, BlueprintReadOnly, Transient, Category="Anim|Runtime")
float LeanAngle = 0.f;
\end{lstlisting}

\subsubsection*{Game Thread 更新基础数据并写入 Proxy}
\begin{lstlisting}[style=cpp]
void UAnimInstBase::NativeUpdateAnimation(float DeltaSeconds)
{
    Super::NativeUpdateAnimation(DeltaSeconds);
    // 收集移动组件信息 ...
    GroundSpeed = CharacterMovement->Velocity.Size2D();
    bShouldMove = GroundSpeed > 2.f &&
                  !CharacterMovement->GetCurrentAcceleration().IsNearlyZero();

    // 写入 Proxy 副本(线程安全数据)
    GetAuraProxy().SetMovementData(GroundSpeed, bShouldMove);
}
\end{lstlisting}

\subsubsection*{帧末回拷派生值}
\begin{lstlisting}[style=cpp]
void UAnimInstBase::PostUpdateAnimation()
{
    Super::PostUpdateAnimation();
    const FAuraAnimInstanceProxy& P =
        static_cast<const FAuraAnimInstanceProxy&>(GetProxyOnGameThread());
    LeanAngle = P.GetComputedLeanAngle(); // 下一帧 Anim Graph 可用
}
\end{lstlisting}

\subsection{可选:蓝图函数直接访问 Proxy(一般不推荐用于状态机)}
\begin{lstlisting}[style=cpp]
UFUNCTION(BlueprintCallable, Category="Anim|Proxy")
float GetProxyLeanAngle() const
{
    const FAuraAnimInstanceProxy& P =
        static_cast<const FAuraAnimInstanceProxy&>(GetProxyOnGameThread());
    return P.GetComputedLeanAngle();
}
\end{lstlisting}
注意:不要在并行评估路径(图节点内部)依赖这种函数;只适用于事件图(Event Graph)或通知回调中临时获取。

\subsection{数据流时序图}
\[
\begin{array}{lcl}
\text{Frame N (GT)} &: & \text{NativeUpdateAnimation() 写入 UPROPERTY 与 Proxy} \\
\text{Parallel (Workers)} &: & \text{Proxy::Update/Evaluate 使用副本并计算派生值} \\
\text{Frame N End (GT)} &: & \text{PostUpdateAnimation() 回拷派生值} \\
\text{Frame N+1 (GT)} &: & \text{Anim Graph/状态机读取 LeanAngle 等新值}
\end{array}
\]

\subsection{常见坑与对策}
\begin{tabular}{p{4cm} p{9.5cm}}
\textbf{现象} & \textbf{原因 / 解决} \\
蓝图读到旧值 & Proxy 结果未回拷或一帧延迟本就预期;检查 PostUpdateAnimation 是否执行 \\
崩溃访问 Actor & 在 Proxy::Update 中访问了 Actor/UObject;改为 GT 收集 + 值复制 \\
本帧想立即生效却失败 & 并行任务本帧后段才完成;只能接受一帧延迟或挪回 NativeUpdateAnimation \\
值来回跳 & 在多个位置(GT 与 Proxy)重复写同一状态;统一写入源头 + 单向复制 \\
\end{tabular}

\subsection{何时无需回拷}
\begin{itemize}
  \item 仅在 Proxy 内部临时使用的过渡辅助量。
  \item 计算成本低,直接放 \cii{NativeUpdateAnimation} 更简单。
\end{itemize}

\subsection{设计准则}
\[
\text{“蓝图需要实时”} \Rightarrow \text{GT 计算} \quad|\quad
\text{“可下一帧使用且昂贵”} \Rightarrow \text{Proxy 计算 + 帧末回拷}
\]

\subsection{调试建议}
\begin{enumerate}
  \item 打开 \cii{stat anim} 确认并行是否生效。
  \item 在调试阶段对回拷数据加断言或范围检查(例如 \(-90^\circ \le LeanAngle \le 90^\circ\))。
  \item 避免在回拷函数中做复杂逻辑(保持只做赋值)。
\end{enumerate}

\subsection{极简模版汇总}
\begin{lstlisting}[style=cpp]
// 1. Proxy 里计算
ComputedLeanAngle = ...;

// 2. PostUpdateAnimation 回拷
LeanAngle = Proxy.GetComputedLeanAngle();

// 3. 动画蓝图状态机用 LeanAngle
\end{lstlisting}

\subsection{一行总结}
\[
\boxed{\text{AnimBlueprint 只读 AnimInstance 变量;Proxy 结果若需蓝图使用,帧末回拷再生效。}}
\]

\section{问题概述}
已经使用自定义 \cii{FAnimInstanceProxy::Update + PostUpdate} 方式实现并行更新并成功测试。问题:与直接重写 \cii{UAnimInstance::NativeThreadSafeUpdateAnimation()} 相比,有哪些本质区别与利弊?

\subsection{一帧执行时间线 (简化)}
\begin{enumerate}[label=\arabic*.]
  \item Game Thread:
    \begin{itemize}
      \item \cii{NativeUpdateAnimation()}(非线程安全区,访问 Actor/World OK)
      \item 构建并行任务(收集 AnimInstance/Proxy)
    \end{itemize}
  \item Any Thread (Worker):
    \begin{itemize}
      \item (可选)\cii{NativeThreadSafeUpdateAnimation()}(若你重写)
      \item \cii{FAnimInstanceProxy::Update()}
      \item \cii{FAnimInstanceProxy::Evaluate()}
    \end{itemize}
  \item 回到 Game Thread:
    \begin{itemize}
      \item \cii{FAnimInstanceProxy::PostUpdate()}(你的回传点)
      \item 将姿势/曲线提交到组件 / 渲染
    \end{itemize}
\end{enumerate}

\subsection{设计理念差异}
\begin{center}
\begin{tabular}{p{4cm} p{5.2cm} p{5.2cm}}
\hline
维度 & \texttt{NativeThreadSafeUpdateAnimation} & Proxy (\texttt{Update + PostUpdate}) \\
\hline
接口归属 & AnimInstance 本体 & 独立轻量 Proxy 容器 \\
数据边界 & 直接操作实例成员(需自律) & 线程阶段只触碰已复制的 POD 数据 \\
线程安全策略 & 依赖开发者避免访问 UObject & 结构化隔离,降低误用 \\
并行演进方向 & UE5 中逐步弱化 & 官方推荐主线模式 \\
本帧可见性 & 可直接改实例变量(危险) & 下一帧才可见(通过 PostUpdate 回写) \\
误用风险 & 高:易访问 ASC/World & 低:Proxy 不含 Actor 指针(或弱引用只读) \\
维护成本 & 大项目易混杂逻辑 & 数据流更清晰(GT 收集 → 并行计算 → 回写) \\
\hline
\end{tabular}
\end{center}

\subsection{线程安全与数据回写}
\subsection*{ThreadSafe 方式}
\begin{itemize}
  \item 同一对象(AnimInstance)在 Worker 线程被直接修改。
  \item 任何不慎访问或写入非线程安全成员(如指向 Actor 成员)都可能产生竞态。
\end{itemize}

\subsection*{Proxy 方式}
\begin{itemize}
  \item 并行阶段对 Proxy 内部的值运算,避免共享可变状态。
  \item \cii{PostUpdate} 回到 Game Thread 再写回实例变量,清晰一帧延迟。
\end{itemize}

\subsection{本帧生效 vs. 下一帧生效}
\begin{itemize}
  \item 如果 “必须本帧” 被状态机或 AnimGraph 使用:\textbf{应在 \texttt{NativeUpdateAnimation} 完成计算}。
  \item 如果可以接受一帧延迟(如平滑滤波结果、倾斜角、呼吸幅度等):\textbf{Proxy 并行计算 + PostUpdate 回写}。
\end{itemize}

\subsection{何时使用各自方式}
\subsection*{适合 \ci{NativeUpdateAnimation}}
\begin{itemize}
  \item 访问 Actor/Component/World/GAS Tag。
  \item 本帧动画过渡条件要立刻用的字段。
\end{itemize}

\subsection*{适合 Proxy::Update}
\begin{itemize}
  \item 纯数学、向量插值、噪声、曲线采样。
  \item 与 Gameplay 无直接耦合的派生动画数据。
\end{itemize}

\subsection{性能视角}
\begin{itemize}
  \item 两者基础调用成本近似。
  \item Proxy 法借由结构化隔离降低未来 “误共享 + 缓存抖动” 风险。
  \item 大量角色(群演/AI)场景下 Proxy 法更容易统一调优与检测。
\end{itemize}

\subsection{Root Motion / Montage 交互}
\begin{itemize}
  \item Root Motion/Montage 核心仍由图节点与内部系统处理。
  \item 不建议在 \cii{NativeThreadSafeUpdateAnimation} 里直接 Play/Stop Montage(潜在竞态)。
  \item 使用 Proxy 法时,这类控制留在 Game Thread(\cii{NativeUpdateAnimation} 或其他安全钩子)更自然。
\end{itemize}

\subsection{典型分层模式(推荐)}
\begin{marker}
\ci{GT: NativeUpdateAnimation} (采集原始状态) $\rightarrow$
\ci{AnyThread: Proxy::Update} (派生计算) $\rightarrow $
\ci{GT: Proxy::PostUpdate} (回写) $\rightarrow$
\ci{下帧 AnimGraph 使用}
\end{marker}
\subsection{坑点对照}
\begin{center}
\begin{tabular}{p{4cm} p{4.5cm} p{5.5cm}}
\hline
问题 & ThreadSafeUpdate & Proxy 模式 \\
\hline
误访问 World/ASC & 常见 & 自然规避(无直接指针) \\
数据竞争调试困难 & 是 & 较少 \\
升级维护负担 & 较高 & 低 \\
同帧可见性需求 & 支持(但危险) & 不支持(设计如此) \\
\hline
\end{tabular}
\end{center}

\subsection{选择策略总结}
\begin{itemize}
  \item “实时 + 依赖世界” → GT 计算。
  \item “昂贵 + 可延迟一帧” → Proxy。
  \item “临时遗留代码” → 可保留 ThreadSafeUpdate 但加醒目标注(AnyThread 只做纯数据)。
  \item “团队协作/长线项目” → 统一 Proxy 流程,降低隐性障碍。
\end{itemize}

\subsection{迁移/共存建议}
\begin{enumerate}
  \item 把 ThreadSafeUpdate 中访问 Actor/Component 的逻辑上移到 \cii{NativeUpdateAnimation}。
  \item 只留下必须多线程的纯数学部分,逐步转入 Proxy。
  \item 为 AnimInstance 顶部添加注释:说明并行策略与禁止事项。
\end{enumerate}

\subsection{一行总结}
\[
\boxed{\text{Proxy 模式 = 明确的数据边界 + 安全并行;ThreadSafeUpdate = 旧接口,灵活但易踩坑。}}
\]

\subsection{附注(何时仍可保留 ThreadSafeUpdate)}
\begin{itemize}
  \item 重构成本过高的旧项目。
  \item 快速实验验证并行路径调度。
  \item 临时性能剖析原位插桩(之后迁回 Proxy)。
\end{itemize}


\section{问题重述}
\textbf{问题:} 在 Unreal Engine 5 官方示例项目 Lyra 中,动画蓝图(Anim Blueprint)使用蓝图事件的方式重写(实现) \cii{ThreadSafeUpdateAnimation},并在其中执行“线程安全动画更新”逻辑。为什么不直接采用 C++ 自定义 \cii{FAnimInstanceProxy}(如你当前项目的做法)?这是否是因为蓝图层“无法”实现 C++ 代理的功能,还是另有更深层的设计与权衡?

\subsection{简明结论(TL;DR)}
\begin{itemize}
  \item 不是因为 C++ Proxy 更差,而是\textbf{蓝图无法自行声明和替换 Proxy 类型};要用 Proxy 必须写 C++。
  \item Lyra 的核心目标是:\textbf{教学友好 + 快速调参 + 降低上手门槛}。
  \item 蓝图 \cii{ThreadSafeUpdateAnimation} 事件被用作“受限的 AnyThread 数学运算钩子”,仅进行纯数据插值/平滑,不访问非线程安全对象。
  \item 你的做法(C++ Proxy + PostUpdate 回写)是更工程化、更安全、更适合扩展的长期方案。
\end{itemize}

\subsection{两种路径的结构差异}
\subsubsection{执行序(简化时间线)}
\cii{Game Thread: NativeUpdateAnimation};$\Rightarrow$; \emph{(并行任务构建)} 
;$\Rightarrow$; \cii{AnyThread: ThreadSafeUpdate / Proxy::Update / Proxy::Evaluate}
;$\Rightarrow$; \cii{Game Thread: Proxy::PostUpdate}

\subsubsection{核心定位对比表}
\begin{center}
\begin{tabular}{p{3.2cm} | p{5.2cm} | p{5.5cm}}
\toprule
维度 & 蓝图 ThreadSafeUpdate & C++ Proxy (\texttt{Update + PostUpdate}) \\
\midrule
创建/接入门槛 & 纯蓝图即可 & 必须 C++ 代码派生 \\
线程安全保障方式 & 依赖团队自律(不要访问 Actor/World) & 结构隔离(只复制 POD) \\
实例变量写入时机 & 可“本帧”直接改(潜在风险) & 通常在 PostUpdate 回写(下一帧生效) \\
调参/热更效率 & 极高 & 需编译 \\
误用风险 & 高:易误用非线程安全节点 & 低:无 Actor 直接引用 \\
教学可视化 & 直观(蓝图事件) & 隐藏于 C++ 底层 \\
适合规模 & 小~中 & 中~大(可长期维护) \\
后期性能聚合 & 分散,难集中优化 & 集中(Proxy 统一入口) \\
行为显式性 & 逻辑混在 AnimInstance 内 & 数据流阶段明确 \\
\bottomrule
\end{tabular}
\end{center}

\subsection{Lyra 的具体考量}
\subsubsection{1. 教学/示例属性}
Lyra 面向“想要学习官方结构”的广泛人群:
\begin{itemize}
  \item 展示并行动画的“概念”,不强迫初学者立刻理解 Proxy 内存边界。
  \item 通过蓝图事件名字直接传达“这是线程安全更新区”。
\end{itemize}

\subsubsection{2. 蓝图层能力边界}
\begin{itemize}
  \item 蓝图无法自定义 \cii{FAnimInstanceProxy} 类型,也不能替换引擎内部对 AnimInstance 的 Proxy 绑定。
  \item 若要接入 Proxy 必须 C++:\cii{CreateAnimInstanceProxy / DestroyAnimInstanceProxy}。
  \item 因此,为了让动画师也能“参与”并行阶段处理,只好开放 ThreadSafe 蓝图事件。
\end{itemize}

\subsubsection{3. 快速迭代优先}
\begin{itemize}
  \item 调整插值系数、偏移、速度平滑参数——蓝图热重载更高效。
  \item 示例项目侧重体验与学习,不强求严苛工程归一。
\end{itemize}

\subsubsection{4. 受控的“最小安全区”策略}
Lyra 在蓝图 ThreadSafeUpdate 中“自我约束”:
\begin{itemize}
  \item 只处理上一帧/GT 缓存的纯值(速度、方向、加速度、姿态偏差)。
  \item 禁止访问:\cii{GetWorld()}, Actor Components, AbilitySystem, Montage Play。
  \item 即便有本帧生效的变量写入,也限制其对 Gameplay 状态的副作用。
\end{itemize}

\subsection{为什么不是直接全部 Proxy 化?}
\begin{enumerate}[leftmargin=1.4em]
  \item \textbf{教学曲线:} Proxy 对初学者不直观。
  \item \textbf{动画师工作流:} 他们需要立即调参,不依赖程序改 C++。
  \item \textbf{示例定位:} 不要求覆盖所有工业级最佳实践。
  \item \textbf{复杂度控制:} 折中降低项目初期理解门槛。
\end{enumerate}

\subsection{Proxy 模式的工程增益}
\begin{itemize}
  \item \textbf{数据边界清晰:} GT(收集)/ AnyThread(纯运算)/ GT(回写)阶段确定。
  \item \textbf{防止竞态:} 不会意外在并行中修改 Actor/ASC。
  \item \textbf{集中优化:} 批量角色可统一做 SIMD/Cache 友好策略。
  \item \textbf{代码审查友好:} 线程敏感区域语义性更强。
\end{itemize}

\subsection{何时需要从 Lyra 式 ThreadSafeUpdate 迁移到 Proxy}
\begin{itemize}
  \item 多个 AnimBP 需要共享复杂派生逻辑(如 LeanAngle、AimOffset、Procedural 数学)。
  \item 角色数量激增(>50)并出现性能瓶颈。
  \item 偶发线程崩溃、数据撕裂、未定义行为。
  \item 需要引入自动性能分析与可预测优化点。
\end{itemize}

\subsection{折中混合建议}
\begin{enumerate}[leftmargin=1.4em]
  \item 使用 Proxy 处理“昂贵 + 可延迟一帧”的数据:运动平滑、姿态偏差、方向加速度积分。
  \item 蓝图只保留“最终微调”乘法系数或 UI 可视控制参数。
  \item 增加一个调试开关变量(如 \cii{bEnableBlueprintThreadSafeTweaks})便于 A/B 性能对比。
  \item 为蓝图 ThreadSafeUpdate 加注释(Warning Banner)列出允许操作白名单。
\end{enumerate}

\subsection{常见误解澄清}
\begin{itemize}
  \item \textbf{误解:} “Lyra 用蓝图,说明这是官方推荐最终模式。”  
  \textbf{更正:} Lyra 是教学示例,不等于大型生产最佳实践。
  \item \textbf{误解:} “蓝图事件假并行,其实仍在 GT。”  
  \textbf{更正:} 该事件确实在 AnyThread 调用,只是需要你自己保持逻辑纯净。
  \item \textbf{误解:} “Proxy 必然比 ThreadSafeUpdate 快很多。”  
  \textbf{更正:} 简单逻辑差异甚微;优势体现在可扩展、可审查与防误用。
\end{itemize}

\subsection{选型决策速查表}
\begin{center}
\renewcommand{\arraystretch}{1.12}
\begin{tabular}{p{3.4cm}| p{7.6cm}| p{2.2cm}}
\toprule
场景 & 推荐手段 & 备注 \\
\midrule
教学/试验 & 蓝图 ThreadSafeUpdate & 快速 \\\hline
中/大型长线 & C++ Proxy & 工程化 \\\hline
本帧状态机必需 & \ci{NativeUpdateAnimation} (GT) & 立即可见 \\\hline
昂贵派生+可延迟 & Proxy::Update + PostUpdate & 安全 \\\hline
动画师调参 & 蓝图普通事件(GT) & 避免并行写 \\\hline
大规模性能调优 & 集中在 Proxy & 易聚合 \\
\bottomrule
\end{tabular}
\end{center}

\subsection{内部 Wiki 可复用摘要}
\begin{quote}
Lyra 使用蓝图 ThreadSafeUpdate 是教学与快速调参折中;蓝图不能直接实现自定义 Proxy,因此官方示例给出一个受控并行数学入口。工业化生产建议使用:GT 收集输入 → Proxy 并行纯数据计算 → PostUpdate 回写;蓝图只做少量微调,避免在 AnyThread 访问 World/Actor/ASC。
\end{quote}

\subsection{一行强化记忆}
\[
\boxed{\text{Lyra 蓝图 ThreadSafeUpdate = 教学/调参折中;Proxy = 工程化长线方案。}}
\]

\subsection{如需下一步}
可继续请求:
\begin{itemize}
  \item “将 ThreadSafeUpdate 逻辑迁到 Proxy” 最小示例
  \item 通用 Lean/Aim 计算 Proxy 模板
  \item 并行性能基准测试脚本建议
\end{itemize}

%/////
\section{蓝图ThreadSafeUpdate并行逻辑迁移到动画代理的示例}
把原本放在 \textbf{蓝图} \texttt{ThreadSafeUpdate}(或 \cppsign~\texttt{NativeThreadSafeUpdateAnimation})里的并行计算逻辑,迁移到自定义 \texttt{FAnimInstanceProxy::Update}(AnyThread)中;Game Thread 只做\emph{指针/原始数据采集};结果在 \texttt{Proxy::PostUpdate} 回填到 AnimInstance 的 Blueprint 可读变量。这样避免在并行线程直接访问 Actor / World / Component,保持线程安全。

\subsection{设计最小职责拆分}
\begin{enumerate}
  \item \textbf{GameThread(NativeUpdateAnimation)}:采集并缓存角色移动组件指针。
  \item \textbf{AnyThread(Proxy::Update)}:读取已缓存指针的轻量字段(如 \texttt{Velocity}),计算 \texttt{GroundSpeed} 与逻辑标志 \texttt{bShouldMove}。
  \item \textbf{GameThread(Proxy::PostUpdate)}:把计算结果回写到 AnimInstance \cii{UPROPERTY},供蓝图 \cii{PropertyAccess} 读取。
  \item \textbf{Blueprint}:无需再实现 ThreadSafeUpdate,直接读 \texttt{GroundSpeed} / \texttt{bShouldMove}。
\end{enumerate}

\subsection{关键注意点}
\begin{itemize}
  \item 必须覆写 \texttt{CreateAnimInstanceProxy} / \texttt{DestroyAnimInstanceProxy},否则不会使用自定义 Proxy。
  \item Proxy 内部仅放 POD / 简单类型;指针用 \texttt{TWeakObjectPtr} 或仅在 Game Thread 赋值。
  \item \textbf{AnyThread 禁止} 访问:Actor 方法、ASC、World、动态创建 UObject、写 AnimInstance 成员。
  \item 结果数据通过 \cii{PostUpdate} 安全回填。
\end{itemize}

\subsection{头文件最小实现 (\texttt{AnimInstBase.h})}
\begin{code}[caption={AnimInstBase.h}]
#pragma once
#include "Animation/AnimInstance.h"
#include "AnimInstBase.generated.h"

class UCharacterMovementComponent;
class AAuraCharacterBase;

/**
 * 最小 AnimInstance:只演示 Proxy 迁移 GroundSpeed / bShouldMove
 */
UCLASS()
class UAnimInstBase : public UAnimInstance
{
    GENERATED_BODY()
public:
    UAnimInstBase();

    virtual void NativeInitializeAnimation() override;
    virtual void NativeUpdateAnimation(float DeltaSeconds) override;

    // 获取自定义 Proxy(仅 GameThread 调用)
    class FAuraAnimInstanceProxy& GetAuraProxy();

    // Blueprint 可直接 PropertyAccess 的数据(PostUpdate 回填)
    UPROPERTY(BlueprintReadOnly, Category="Movement")
    float GroundSpeed = 0.f;

    UPROPERTY(BlueprintReadOnly, Category="Movement")
    bool bShouldMove = false;

protected:
    UPROPERTY(Transient)
    AAuraCharacterBase* AuraCharacter = nullptr;

    UPROPERTY(Transient)
    UCharacterMovementComponent* CharacterMovement = nullptr;

    // 建立 / 销毁自定义 Proxy
    virtual FAnimInstanceProxy* CreateAnimInstanceProxy() override;
    virtual void DestroyAnimInstanceProxy(FAnimInstanceProxy* InProxy) override;
};

/**
 * 自定义 Proxy:
 *  - AnyThread Update 中进行逻辑计算
 *  - PostUpdate 把缓存结果写回 AnimInstance
 */
class FAuraAnimInstanceProxy : public FAnimInstanceProxy
{
public:
    FAuraAnimInstanceProxy(UAnimInstance* InInstance)
        : FAnimInstanceProxy(InInstance) {}

    // GameThread 推送可供并行读取的指针
    void SetMovementComponent(const UCharacterMovementComponent* InMove)
    {
        CharacterMovementComp = InMove;
    }

    // AnyThread 逻辑
    virtual void Update(float DeltaSeconds) override;

    // GameThread 回填
    virtual void PostUpdate(UAnimInstance* InAnimInstance) const override;

private:
    TWeakObjectPtr<const UCharacterMovementComponent> CharacterMovementComp;

    // 并行安全缓存
    float CachedGroundSpeed = 0.f;
    bool  bCachedShouldMove = false;
};
\end{code}

\subsection{实现文件最小实现 (\texttt{AnimInstBase.cpp})}
\begin{code}[caption={AnimInstBase.cpp}]
#include "Animation/AnimInstBase.h"
#include "Character/AuraCharacterBase.h"
#include "GameFramework/CharacterMovementComponent.h"

// ---------------- Proxy ----------------

void FAuraAnimInstanceProxy::Update(float DeltaSeconds)
{
    FAnimInstanceProxy::Update(DeltaSeconds);

    // 只进行轻量级读操作
    if (const UCharacterMovementComponent* Move = CharacterMovementComp.Get())
    {
        CachedGroundSpeed = Move->Velocity.Size2D();
        bCachedShouldMove = !FMath::IsNearlyZero(CachedGroundSpeed);
    }
    else
    {
        CachedGroundSpeed = 0.f;
        bCachedShouldMove = false;
    }
}

void FAuraAnimInstanceProxy::PostUpdate(UAnimInstance* InAnimInstance) const
{
    FAnimInstanceProxy::PostUpdate(InAnimInstance);
    if (auto* AuraAnim = Cast<UAnimInstBase>(InAnimInstance))
    {
        AuraAnim->GroundSpeed = CachedGroundSpeed;
        AuraAnim->bShouldMove = bCachedShouldMove;
    }
}

// --------------- AnimInstance ---------------

UAnimInstBase::UAnimInstBase() {}

void UAnimInstBase::NativeInitializeAnimation()
{
    AuraCharacter = Cast<AAuraCharacterBase>(GetOwningActor());
    if (AuraCharacter)
    {
        CharacterMovement = AuraCharacter->GetCharacterMovement();
    }
}

void UAnimInstBase::NativeUpdateAnimation(float DeltaSeconds)
{
    if (CharacterMovement)
    {
        // GameThread 仅推送组件指针,不做重计算
        GetAuraProxy().SetMovementComponent(CharacterMovement);
    }
}

FAuraAnimInstanceProxy& UAnimInstBase::GetAuraProxy()
{
    return *static_cast<FAuraAnimInstanceProxy*>(
        GetProxyOnGameThread<FAnimInstanceProxy>()
    );
}

FAnimInstanceProxy* UAnimInstBase::CreateAnimInstanceProxy()
{
    return new FAuraAnimInstanceProxy(this);
}

void UAnimInstBase::DestroyAnimInstanceProxy(FAnimInstanceProxy* InProxy)
{
    delete InProxy;
}
\end{code}

\subsection{迁移对照步骤}
\begin{enumerate}
  \item 原 \texttt{ThreadSafeUpdate} 中的速度采集与移动判定:挪到 \texttt{FAuraAnimInstanceProxy::Update}。
  \item 删除或不再使用旧 Blueprint ThreadSafeUpdate 节点(避免重复写变量)。
  \item 保留 \texttt{NativeUpdateAnimation},只负责推送 MovementComponent 指针。
  \item Blueprint 直接读取 \texttt{GroundSpeed} / \texttt{bShouldMove},无需额外节点。
\end{enumerate}

\subsection{扩展(可选)}
\begin{itemize}
  \item \textbf{YawDelta}:GameThread 保存上一帧 Yaw,Proxy::Update 计算差值,PostUpdate 回填。
  \item \textbf{加速度}:缓存上一帧速度(GameThread 或 Proxy 内部双缓存)求差。
  \item \textbf{事件队列}:Proxy 写 Flag,PostUpdate 在 GameThread 触发通知 / 调试输出。
\end{itemize}

\subsection{Checklist 快速核对}
\begin{itemize}
  \item[\checkmark] 实现并注册自定义 Proxy
  \item[\checkmark] AnyThread 不访问 Actor / World
  \item[\checkmark] 数据回填集中于 PostUpdate
  \item[\checkmark] Blueprint PropertyAccess 正常复用
\end{itemize}

\subsection{一行记忆公式}
\[
\text{Proxy 模式} = \text{GameThread 采集} \rightarrow \text{AnyThread 只读计算} \rightarrow \text{GameThread 回填}.
\]


\end{document}
