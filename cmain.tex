\documentclass{MyBook}
\usepackage{layout}
\begin{document}
\frontmatter


\mainmatter
\setchapterimage{j20}
\setchapterimage{background}
\chapter{Boost.Parser by Example}


\section{Terminology}

First, let's cover some terminology that we'll be using throughout the docs:

A \emph{semantic action} is an arbitrary bit of logic associated with a parser, that is only executed when the parser matches.

Simpler parsers can be combined to form more complex parsers. Given some combining operation \texttt{C}, and parsers \texttt{P0}, \texttt{P1}, ... \texttt{PN}, \texttt{C(P0,\ P1,\ ...\ PN)} creates a new parser \texttt{Q}. This creates a \emph{parse tree}. \texttt{Q} is the parent of \texttt{P1}, \texttt{P2} is the child of \texttt{Q}, etc. The parsers are applied in the top-down fashion implied by this topology. When you use \texttt{Q} to parse a string, it will use \texttt{P0}, \texttt{P1}, etc. to do the actual work. If \texttt{P3} is being used to parse the input, that means that \texttt{Q} is as well, since the way \texttt{Q} parses is by dispatching to its children to do some or all of the work. At any point in the parse, there will be exactly one parser without children that is being used to parse the input; all other parsers being used are its ancestors in the parse tree.

A \emph{subparser} is a parser that is the child of another parser.

The \emph{top-level parser} is the root of the tree of parsers.

The \emph{current parser} or \emph{bottommost parser} is the parser with no children that is currently being used to parse the input.

A \emph{rule} is a kind of parser that makes building large, complex parsers easier. A \emph{subrule} is a rule that is the child of some other rule. The \emph{current rule} or \emph{bottommost rule} is the one rule currently being used to parse the input that has no subrules. Note that while there is always exactly one current parser, there may or may not be a current rule --- rules are one kind of parser, and you may or may not be using one at a given point in the parse.

The \emph{top-level parse} is the parse operation being performed by the top-level parser. This term is necessary because, though most parse failures are local to a particular parser, some parse failures cause the call to \texttt{parse()} to indicate failure of the entire parse. For these cases, we say that such a local failure "causes the top-level parse to fail".

Throughout the Boost.Parser documentation, I will refer to "the call to \texttt{parse()}". Read this as "the call to any one of the functions described in The \texttt{parse()} API". That includes \texttt{prefix\_parse()}, \texttt{callback\_parse()}, and \texttt{callback\_prefix\_parse()}.

There are some special kinds of parsers that come up often in this documentation.

One is a \emph{sequence parser}; you will see it created using \texttt{operator>>}, as in \texttt{p1\ >>\ p2\ >>\ p3}. A sequence parser tries to match all of its subparsers to the input, one at a time, in order. It matches the input iff all its subparsers do.

Another is an \emph{alternative parser}; you will see it created using \texttt{operator\textbar{}}, as in \texttt{p1\ \textbar{}\ p2\ \textbar{}\ p3}. An alternative parser tries to match all of its subparsers to the input, one at a time, in order; it stops after matching at most one subparser. It matches the input iff one of its subparsers does.

Finally, there is a \emph{permutation parser}; it is created using \texttt{operator\textbar{}\textbar{}}, as in \texttt{p1\ \textbar{}\textbar{}\ p2\ \textbar{}\textbar{}\ p3}. A permutation parser tries to match all of its subparsers to the input, in any order. So the parser \texttt{p1\ \textbar{}\textbar{}\ p2\ \textbar{}\textbar{}\ p3} is equivalent to \texttt{(p1\ >>\ p2\ >>\ p3)\ \textbar{}\ (p1\ >>\ p3\ >>\ p2)\ \textbar{}\ (p2\ >>\ p1\ >>\ p3)\ \textbar{}\ (p2\ >>\ p3\ >>\ p1)\ \textbar{}\ (p3\ >>\ p1\ >>\ p2)\ \textbar{}\ (p3\ >>\ p2\ >>\ p1)}. Hopefully its terseness is self-explanatory. It matches the input iff all of its subparsers do, regardless of the order they match in.

Boost.Parser parsers each have an \emph{attribute} associated with them, or explicitly have no attribute. An attribute is a value that the parser generates when it matches the input. For instance, the parser \texttt{double\_} generates a \texttt{double} when it matches the input. \emph{\texttt{ATTR}}\texttt{()} is a notional macro that expands to the attribute type of the parser passed to it; \emph{\texttt{ATTR}}\texttt{(double\_)} is \texttt{double}. This is similar to the \texttt{attribute} type trait.

Next, we'll look at some simple programs that parse using Boost.Parser. We'll start small and build up from there.

\section{Hello, Whomever}

This is just about the most minimal example of using Boost.Parser that one could write. We take a string from the command line, or \texttt{"World"} if none is given, and then we parse it:

\begin{code}[caption={示例代码}]
#include <boost/parser/parser.hpp>

#include <iostream>
#include <string>


namespace bp = boost::parser;

int main(int argc, char const * argv[])
{
    std::string input = "World";
    if (1 < argc)
        input = argv[1];

    std::string result;
    bp::parse(input, *bp::char_, result);
    std::cout << "Hello, " << result << "!\n";
    auto const action = [&result](auto & ctx) { std::cout << "Got one!\n"; result.push_back(_attr(ctx)); }
\end{code}

The expression \texttt{*bp::char\_} is a parser-expression. It uses one of the many parsers that Boost.Parser provides: \texttt{char\_}. Like all Boost.Parser parsers, it has certain operations defined on it. In this case, \texttt{*bp::char\_} is using an overloaded \ci{operator*} as the C++ version of a Kleene star operator. Since C++ has no postfix unary \ci{*} operator, we have to use the one we have, so it is used as a prefix.

So, \ci{*bp::char\_} means "any number of characters". In other words, it really cannot fail. Even an empty string will match it.

The parse operation is performed by calling the \ci{parse()} function, passing the parser as one of the arguments:

\begin{code}
bp::parse(input, *bp::char_, result);
\end{code}

The arguments here are: \ci{input}, the range to parse; \ci{*bp::char\_}, the parser used to do the parse; and \ci{result}, an out-parameter into which to put the result of the parse. Don't get too caught up on this method of getting the parse result out of \ci{parse()}; there are multiple ways of doing so, and we'll cover all of them in subsequent sections.

Also, just ignore for now the fact that Boost.Parser somehow figured out that the result type of the \ci{*bp::char\_} parser is a \ci{std::string}. There are clear rules for this that we'll cover later.

The effects of this call to \ci{parse()} is not very interesting --- since the parser we gave it cannot ever fail, and because we're placing the output in the same type as the input, it just copies the contents of \ci{input} to \ci{result}.

\section{A Trivial Example}

Let's look at a slightly more complicated example, even if it is still trivial. Instead of taking any old \ci{char}s we're given, let's require some structure. Let's parse one or more \ci{double}s, separated by commas.

The Boost.Parser parser for \ci{double} is \ci{double\_}. So, to parse a single \ci{double}, we'd just use that. If we wanted to parse two \ci{double}s in a row, we'd use:

\begin{code}
boost::parser::double_ >> boost::parser::double_
\end{code}

\ci{operator>>} in this expression is the sequence-operator; read it as "followed by". If we combine the sequence-operator with Kleene star, we can get the parser we want by writing:

\begin{code}
boost::parser::double_ >> *(',' >> boost::parser::double_)
\end{code}

This is a parser that matches at least one \ci{double} --- because of the first \ci{double\_} in the expression above --- followed by zero or more instances of a-comma-followed-by-a-\ci{double}. Notice that we can use \ci{'{},'{}} directly. Though it is not a parser, \ci{operator>>} and the other operators defined on Boost.Parser parsers have overloads that accept character/parser pairs of arguments; these operator overloads will create the right parser to recognize \ci{'{},'{}}.

\begin{code}
#include <boost/parser/parser.hpp>

#include <iostream>
#include <string>


namespace bp = boost::parser;

int main()
{
    std::cout << "Enter a list of doubles, separated by commas.  No pressure. ";
    std::string input;
    std::getline(std::cin, input);

    auto const result = bp::parse(input, bp::double_ >> *(',' >> bp::double_));

    if (result) {
        std::cout << "Great! It looks like you entered:\n";
        for (double x : *result) {
            std::cout << x << "\n";
        }
    } else {
        std::cout
            << "Good job!  Please proceed to the recovery annex for cake.\n";
    }
}
\end{code}

The first example filled in an out-parameter to deliver the result of the parse. This call to \ci{parse()} returns a result instead. As you can see, the result is contextually convertible to \ci{bool}, and \ci{*result} is some sort of range. In fact, the return type of this call to \ci{parse()} is \ci{std::optional<std::vector<double>>}. Naturally, if the parse fails, \ci{std::nullopt} is returned. We'll look at how Boost.Parser maps the type of the parser to the return type, or the filled in out-parameter's type, a bit later.



If I run it in a shell, this is the result:

\begin{code}
$ example/trivial
Enter a list of doubles, separated by commas.  No pressure. 5.6,8.9
Great! It looks like you entered:
5.6
8.9
$ example/trivial
Enter a list of doubles, separated by commas.  No pressure. 5.6, 8.9
Good job!  Please proceed to the recovery annex for cake.
\end{code}

It does not recognize \ci{"5.6,\ 8.9"}. This is because it expects a comma followed \emph{immediately} by a \ci{double}, but I inserted a space after the comma. The same failure to parse would occur if I put a space before the comma, or before or after the list of \ci{double}s.

One more thing: there is a much better way to write the parser above. Instead of repeating the \ci{double\_} subparser, we could have written this:

\begin{code}
bp::double_ % ','
\end{code}

That's semantically identical to \ci{bp::double\_\ >>\ *('{},'{}\ >>\ bp::double\_)}. This pattern --- some bit of input repeated one or more times, with a separator between each instance --- comes up so often that there's an operator specifically for that, \ci{operator\%}. We'll be using that operator from now on.

\section{A Trivial Example That Gracefully Handles Whitespace}

Let's modify the trivial parser we just saw to ignore any spaces it might find among the \ci{double}s and commas. To skip whitespace wherever we find it, we can pass a \emph{skip parser} to our call to \ci{parse()} (we don't need to touch the parser passed to \ci{parse()}). Here, we use \ci{ws}, which matches any Unicode whitespace character.

\begin{code}
#include <boost/parser/parser.hpp>

#include <iostream>
#include <string>


namespace bp = boost::parser;

int main()
{
    std::cout << "Enter a list of doubles, separated by commas.  No pressure. ";
    std::string input;
    std::getline(std::cin, input);

    auto const result = bp::parse(input, bp::double_ % ',', bp::ws);

    if (result) {
        std::cout << "Great! It looks like you entered:\n";
        for (double x : *result) {
            std::cout << x << "\n";
        }
    } else {
        std::cout
            << "Good job!  Please proceed to the recovery annex for cake.\n";
    }
}
\end{code}

The skip parser, or \emph{skipper}, is run between the subparsers within the parser passed to \ci{parse()}. In this case, the skipper is run before the first \ci{double} is parsed, before any subsequent comma or \ci{double} is parsed, and at the end. So, the strings \ci{"3.6,5.9"} and \ci{"\ 3.6\ ,\ \textbackslash{}t\ 5.9\ "} are parsed the same by this program.

Skipping is an important concept in Boost.Parser. You can skip anything, not just whitespace; there are lots of other things you might want to skip. The skipper you pass to \ci{parse()} can be an arbitrary parser. For example, if you write a parser for a scripting language, you can write a skipper to skip whitespace, inline comments, and end-of-line comments.

We'll be using skip parsers almost exclusively in the rest of the documentation. The ability to ignore the parts of your input that you don't care about is so convenient that parsing without skipping is a rarity in practice.

\section{Semantic Actions}

Like all parsing systems (lex \& yacc, Boost.Spirit, etc.), Boost.Parser has a mechanism for associating semantic actions with different parts of the parse. Here is nearly the same program as we saw in the previous example, except that it is implemented in terms of a semantic action that appends each parsed \ci{double} to a result, instead of automatically building and returning the result. To do this, we replace the \ci{double\_} from the previous example with \ci{double\_{[}action{]}}; \ci{action} is our semantic action:

\begin{code}
#include <boost/parser/parser.hpp>

#include <iostream>
#include <string>


namespace bp = boost::parser;

int main()
{
    std::cout << "Enter a list of doubles, separated by commas. ";
    std::string input;
    std::getline(std::cin, input);

    std::vector<double> result;
    auto const action = [&result](auto & ctx) {
        std::cout << "Got one!\n";
        result.push_back(_attr(ctx));
    };
    auto const action_parser = bp::double_[action];
    auto const success = bp::parse(input, action_parser % ',', bp::ws);

    if (success) {
        std::cout << "You entered:\n";
        for (double x : result) {
            std::cout << x << "\n";
        }
    } else {
        std::cout << "Parse failure.\n";
    }
}
\end{code}

Run in a shell, it looks like this:

\begin{code}
$ example/semantic_actions
Enter a list of doubles, separated by commas. 4,3
Got one!
Got one!
You entered:
4
3
\end{code}

In Boost.Parser, semantic actions are implemented in terms of invocable objects that take a single parameter to a parse-context object. The parse-context object represents the current state of the parse. In the example we used this lambda as our invocable:

\begin{code}
auto const action = [&result](auto & ctx) {
    std::cout << "Got one!\n";
    result.push_back(_attr(ctx));
};
\end{code}

We're both printing a message to \ci{std::cout} and recording a parsed result in the lambda. It could do both, either, or neither of these things if you like. The way we get the parsed \ci{double} in the lambda is by asking the parse context for it. \ci{\_attr(ctx)} is how you ask the parse context for the attribute produced by the parser to which the semantic action is attached. There are lots of functions like \ci{\_attr()} that can be used to access the state in the parse context. We'll cover more of them later on. The Parse Context defines what exactly the parse context is and how it works.

Note that you can't write an unadorned lambda directly as a semantic action. Otherwise, the compile will see two \ci{'{}{[}'{}} characters and think it's about to parse an attribute. Parentheses fix this:

\begin{code}
p[([](auto & ctx){/*...*/})]
\end{code}

Before you do this, note that the lambdas that you write as semantic actions are almost always generic (having an \ci{auto\ \&\ ctx} parameter), and so are very frequently re-usable. Most semantic action lambdas you write should be written out-of-line, and given a good name. Even when they are not reused, named lambdas keep your parsers smaller and easier to read.



\subsection{Semantic actions inside rules}

There are some other forms for semantic actions, when they are used inside of \ci{rules}. See More About Rules for details.

\section{Parsing to Find Subranges}

So far we've seen examples that parse some text and generate associated attributes. Sometimes, you want to find some subrange of the input that contains what you're looking for, and you don't want to generate attributes at all.

There are two \emph{directive}s that affect the attribute type of any parser, \ci{raw{[}{]}} and \ci{string\_view{[}{]}}. (We'll get to directives in more detail in the Directives section later. For now, you just need to know that a directive wraps a parser, and changes some aspect of how it functions.)

\subsection{raw{[}{]}}

\ci{raw{[}{]}} changes the attribute of its parser to be a \ci{subrange} whose \ci{begin()} and \ci{end()} return the bounds of the sequence being parsed that match \ci{p}.

\begin{code}
namespace bp = boost::parser;
auto int_parser = bp::int_ % ',';            // ATTR(int_parser) is std::vector<int>
auto subrange_parser = bp::raw[int_parser];  // ATTR(subrange_parser) is a subrange

// Parse using int_parser, generating integers.
auto ints = bp::parse("1, 2, 3, 4", int_parser, bp::ws);
assert(ints);
assert(*ints == std::vector<int>({1, 2, 3, 4}));

// Parse again using int_parser, but this time generating only the
// subrange matched by int_parser.  (prefix_parse() allows matches that
// don't consume the entire input.)
auto const str = std::string("1, 2, 3, 4, a, b, c");
auto first = str.begin();
auto range = bp::prefix_parse(first, str.end(), subrange_parser, bp::ws);
assert(range);
assert(range->begin() == str.begin());
assert(range->end() == str.begin() + 10);

static_assert(std::is_same_v<
              decltype(range),
              std::optional<bp::subrange<std::string::const_iterator>>>);
\end{code}

Note that the \ci{subrange} has the iterator type \ci{std::string::const\_iterator}, because that's the iterator type passed to \ci{prefix\_parse()}. If we had passed \ci{char\ const\ *} iterators to \ci{prefix\_parse()}, that would have been the iterator type. The only exception to this comes from Unicode-aware parsing (see Unicode Support). In some of those cases, the iterator being used in the parse is not the one you passed. For instance, if you call \ci{prefix\_parse()} with \ci{char8\_t\ *} iterators, it will create a UTF-8 to UTF-32 transcoding view, and parse the iterators of that view. In such a case, you'll get a \ci{subrange} whose iterator type is a transcoding iterator. When that happens, you can get the underlying iterator --- the one you passed to \ci{prefix\_parse()} --- by calling the \ci{.base()} member function on each transcoding iterator in the returned \ci{subrange}.

\begin{code}
auto const u8str = std::u8string(u8"1, 2, 3, 4, a, b, c");
auto u8first = u8str.begin();
auto u8range = bp::prefix_parse(u8first, u8str.end(), subrange_parser, bp::ws);
assert(u8range);
assert(u8range->begin().base() == u8str.begin());
assert(u8range->end().base() == u8str.begin() + 10);
\end{code}

\subsection{string\_view{[}{]}}

\ci{string\_view{[}{]}} has very similar semantics to \ci{raw{[}{]}}, except that it produces a \ci{std::basic\_string\_view<CharT>} (where \ci{CharT} is the type of the underlying range begin parsed) instead of a \ci{subrange}. For this to work, the underlying range must be contiguous. Contiguity of iterators is not detectable before C++20, so this directive is only available in C++20 and later.

\begin{code}
namespace bp = boost::parser;
auto int_parser = bp::int_ % ',';              // ATTR(int_parser) is std::vector<int>
auto sv_parser = bp::string_view[int_parser];  // ATTR(sv_parser) is a string_view

auto const str = std::string("1, 2, 3, 4, a, b, c");
auto first = str.begin();
auto sv1 = bp::prefix_parse(first, str.end(), sv_parser, bp::ws);
assert(sv1);
assert(*sv1 == str.substr(0, 10));

static_assert(std::is_same_v<decltype(sv1), std::optional<std::string_view>>);
\end{code}

Since \ci{string\_view{[}{]}} produces \ci{string\_view}s, it cannot return transcoding iterators as described above for \ci{raw{[}{]}}. If you parse a sequence of \ci{CharT} with \ci{string\_view{[}{]}}, you get exactly a \ci{std::basic\_string\_view<CharT>}. If the parse is using transcoding in the Unicode-aware path, \ci{string\_view{[}{]}} will decompose the transcoding iterator as necessary. If you pass a transcoding view to \ci{parse()} or transcoding iterators to \ci{prefix\_parse()}, \ci{string\_view{[}{]}} will still see through the transcoding iterators without issue, and give you a \ci{string\_view} of part of the underlying range.

\begin{code}
auto sv2 = bp::parse("1, 2, 3, 4" | bp::as_utf32, sv_parser, bp::ws);
assert(sv2);
assert(*sv2 == "1, 2, 3, 4");

static_assert(std::is_same_v<decltype(sv2), std::optional<std::string_view>>);
\end{code}

\section{The Parse Context}

Now would be a good time to describe the parse context in some detail. Any semantic action that you write will need to use state in the parse context, so you need to know what's available.

The parse context is an object that stores the current state of the parse --- the current- and end-iterators, the error handler, etc. Data may seem to be "added" to or "removed" from it at different times during the parse. For instance, when a parser \ci{p} with a semantic action \ci{a} succeeds, the context adds the attribute that \ci{p} produces to the parse context, then calls \ci{a}, passing it the context.

Though the context object appears to have things added to or removed from it, it does not. In reality, there is no one context object. Contexts are formed at various times during the parse, usually when starting a subparser. Each context is formed by taking the previous context and adding or changing members as needed to form a new context object. When the function containing the new context object returns, its context object (if any) is destructed. This is efficient to do, because the parse context has only about a dozen data members, and each data member is less than or equal to the size of a pointer. Copying the entire context when mutating the context is therefore fast. The context does no memory allocation.

\begin{marker}
All these functions that take the parse context as their first parameter will find by found by Argument-Dependent Lookup. You will probably never need to qualify them with \ci{boost::parser::}. 
\end{marker}

\subsection{Accessors for data that are always available}

By convention, the names of all Boost.Parser functions that take a parse context, and are therefore intended for use inside semantic actions, contain a leading underscore.

\subsection{\_pass()}

\ci{\_pass()} returns a reference to a \ci{bool} indicating the success of failure of the current parse. This can be used to force the current parse to pass or fail:

\begin{code}
[](auto & ctx) {
    // If the attribute fails to meet this predicate, fail the parse.
    if (!necessary_condition(_attr(ctx)))
        _pass(ctx) = false;
}
\end{code}

Note that for a semantic action to be executed, its associated parser must already have succeeded. So unless you previously wrote \ci{\_pass(ctx)\ =\ false} within your action, \ci{\_pass(ctx)\ =\ true} does nothing; it's redundant.

\subsection{\_begin(), \_end() and \_where()}

\ci{\_begin()} and \ci{\_end()} return the beginning and end of the range that you passed to \ci{parse()}, respectively. \ci{\_where()} returns a \ci{subrange} indicating the bounds of the input matched by the current parse. \ci{\_where()} can be useful if you just want to parse some text and return a result consisting of where certain elements are located, without producing any other attributes. \ci{\_where()} can also be essential in tracking where things are located, to provide good diagnostics at a later point in the parse. Think mismatched tags in XML; if you parse a close-tag at the end of an element, and it does not match the open-tag, you want to produce an error message that mentions or shows both tags. Stashing \ci{\_where(ctx).begin()} somewhere that is available to the close-tag parser will enable that. See Error Handling and Debugging for an example of this.

\subsection{\_error\_handler()}

\ci{\_error\_handler()} returns a reference to the error handler associated with the parser passed to \ci{parse()}. Using \ci{\_error\_handler()}, you can generate errors and warnings from within your semantic actions. See Error Handling and Debugging for concrete examples.

\subsection{Accessors for data that are only sometimes available}

\subsection{\_attr()}

\ci{\_attr()} returns a reference to the value of the current parser's attribute. It is available only when the current parser's parse is successful. If the parser has no semantic action, no attribute gets added to the parse context. It can be used to read and write the current parser's attribute:

\begin{code}
[](auto & ctx) { _attr(ctx) = 3; }
\end{code}

If the current parser has no attribute, a \ci{none} is returned.

\subsection{\_val()}

\ci{\_val()} returns a reference to the value of the attribute of the current rule being used to parse (if any), and is available even before the rule's parse is successful. It can be used to set the current rule's attribute, even from a parser that is a subparser inside the rule. Let's say we're writing a parser with a semantic action that is within a rule. If we want to set the current rule's value to some function of subparser's attribute, we would write this semantic action:

\begin{code}
[](auto & ctx) { _val(ctx) = some_function(_attr(ctx)); }
\end{code}

If there is no current rule, or the current rule has no attribute, a \ci{none} is returned.

You need to use \ci{\_val()} in cases where the default attribute for a \ci{rule}'s parser is not directly compatible with the attribute type of the \ci{rule}. In these cases, you'll need to write some code like the example above to compute the \ci{rule}'s attribute from the \ci{rule}'s parser's generated attribute. For more info on \ci{rules}, see the next page, and More About Rules.

\subsection{\_globals()}

\ci{\_globals()} returns a reference to a user-supplied object that contains whatever data you want to use during the parse. The "globals" for a parse is an object --- typically a struct --- that you give to the top-level parser. Then you can use \ci{\_globals()} to access it at any time during the parse. We'll see how globals get associated with the top-level parser in The \ci{parse()} API later. As an example, say that you have an early part of the parse that needs to record some black-listed values, and that later parts of the parse might need to parse values, failing the parse if they see the black-listed values. In the early part of the parse, you could write something like this.

\begin{code}
[](auto & ctx) {
    // black_list is a std::unordered_set.
    _globals(ctx).black_list.insert(_attr(ctx));
}
\end{code}

Later in the parse, you could then use \ci{black\_list} to check values as they are parsed.

\begin{code}
[](auto & ctx) {
    if (_globals(ctx).black_list.contains(_attr(ctx)))
        _pass(ctx) = false;
}
\end{code}

\subsection{\_locals()}

\ci{\_locals()} returns a reference to one or more values that are local to the current rule being parsed, if any. If there are two or more local values, \ci{\_locals()} returns a reference to a \ci{boost::parser::tuple}. Rules with locals are something we haven't gotten to yet (see More About Rules), but for now all you need to know is that you can provide a template parameter (\ci{LocalState}) to \ci{rule}, and the rule will default construct an object of that type for use within the rule. You access it via \ci{\_locals()}:

\begin{code}
[](auto & ctx) {
    auto & local = _locals(ctx);
    // Use local here.  If 'local' is a hana::tuple, access its members like this:
    using namespace hana::literals;
    auto & first_element = local[0_c];
    auto & second_element = local[1_c];
}
\end{code}

If there is no current rule, or the current rule has no locals, a \ci{none} is returned.

\subsection{\_params()}

\ci{\_params()}, like \ci{\_locals()}, applies to the current rule being used to parse, if any (see More About Rules). It also returns a reference to a single value, if the current rule has only one parameter, or a \ci{boost::parser::tuple} of multiple values if the current rule has multiple parameters. If there is no current rule, or the current rule has no parameters, a \ci{none} is returned.

Unlike with \ci{\_locals()}, you \textbf{do not} provide a template parameter to \ci{rule}. Instead you call the \ci{rule}'s \ci{with()} member function (again, see More About Rules).

\begin{marker}[title=Note ]
\ci{none} is a type that is used as a return value in Boost.Parser for parse context accessors. \ci{none} is convertible to anything that has a default constructor, convertible from anything, assignable form anything, and has templated overloads for all the overloadable operators. The intention is that a misuse of \ci{\_val()}, \ci{\_globals()}, etc. should compile, and produce an assertion at runtime. Experience has shown that using a debugger for investigating the stack that leads to your mistake is a far better user experience than sifting through compiler diagnostics. See the Rationale section for a more detailed explanation. 
\end{marker}

\subsection{{\ci{\_no\_case()}}{\_no\_case()}}

\ci{\_no\_case()} returns \ci{true} if the current parse context is inside one or more (possibly nested) \ci{no\_case{[}{]}} directives. I don't have a use case for this, but if I didn't expose it, it would be the only thing in the context that you could not examine from inside a semantic action. It was easy to add, so I did.

\section{Rule Parsers}

This example is very similar to the others we've seen so far. This one is different only because it uses a \ci{rule}. As an analogy, think of a parser like \ci{char\_} or \ci{double\_} as an individual line of code, and a \ci{rule} as a function. Like a function, a \ci{rule} has its own name, and can even be forward declared. Here is how we define a \ci{rule}, which is analogous to forward declaring a function:

\begin{code}
bp::rule<struct doubles, std::vector<double>> doubles = "doubles";
\end{code}

This declares the rule itself. The \ci{rule} is a parser, and we can immediately use it in other parsers. That definition is pretty dense; take note of these things:

\begin{itemize}
\item
  The first template parameter is a tag type \ci{struct\ doubles}. Here we've declared the tag type and used it all in one go; you can also use a previously declared tag type.
\item
  The second template parameter is the attribute type of the parser. If you don't provide this, the rule will have no attribute.
\item
  This rule object itself is called \ci{doubles}.
\item
  We've given \ci{doubles} the diagnstic text \ci{"doubles"} so that Boost.Parser knows how to refer to it when producing a trace of the parser during debugging.
\end{itemize}

Ok, so if \ci{doubles} is a parser, what does it do? We define the rule's behavior by defining a separate parser that by now should look pretty familiar:

\begin{code}
auto const doubles_def = bp::double_ % ',';
\end{code}

This is analogous to writing a definition for a forward-declared function. Note that we used the name \ci{doubles\_def}. Right now, the \ci{doubles} rule parser and the \ci{doubles\_def} non-rule parser have no connection to each other. That's intentional --- we want to be able to define them separately. To connect them, we declare functions with an interface that Boost.Parser understands, and use the tag type \ci{struct\ doubles} to connect them together. We use a macro for that:

\begin{code}
BOOST_PARSER_DEFINE_RULES(doubles);
\end{code}

This macro expands to the code necessary to make the rule \ci{doubles} and its parser \ci{doubles\_def} work together. The \ci{\_def} suffix is a naming convention that this macro relies on to work. The tag type allows the rule parser, \ci{doubles}, to call one of these overloads when used as a parser.

\ci{BOOST\_PARSER\_DEFINE\_RULES} expands to two overloads of a function called \ci{parse\_rule()}. In the case above, the overloads each take a \ci{struct\ doubles} parameter (to distinguish them from the other overloads of \ci{parse\_rule()} for other rules) and parse using \ci{doubles\_def}. You will never need to call any overload of \ci{parse\_rule()} yourself; it is used internally by the parser that implements \ci{rules}, \ci{rule\_parser}.

Here is the definition of the macro that is expanded for each rule:

\begin{code}
#define BOOST_PARSER_DEFINE_IMPL(_, rule_name_)                                \
    template<                                                                  \
        typename Iter,                                                         \
        typename Sentinel,                                                     \
        typename Context,                                                      \
        typename SkipParser>                                                   \
    decltype(rule_name_)::parser_type::attr_type parse_rule(                   \
        decltype(rule_name_)::parser_type::tag_type *,                         \
        Iter & first,                                                          \
        Sentinel last,                                                         \
        Context const & context,                                               \
        SkipParser const & skip,                                               \
        boost::parser::detail::flags flags,                                    \
        bool & success,                                                        \
        bool & dont_assign)                                                    \
    {                                                                          \
        auto const & parser = BOOST_PARSER_PP_CAT(rule_name_, _def);           \
        using attr_t =                                                         \
            decltype(parser(first, last, context, skip, flags, success));      \
        using attr_type = decltype(rule_name_)::parser_type::attr_type;        \
        if constexpr (boost::parser::detail::is_nope_v<attr_t>) {              \
            dont_assign = true;                                                \
            parser(first, last, context, skip, flags, success);                \
            return {};                                                         \
        } else if constexpr (std::is_same_v<attr_type, attr_t>) {              \
            return parser(first, last, context, skip, flags, success);         \
        } else if constexpr (std::is_constructible_v<attr_type, attr_t>) {     \
            return attr_type(                                                  \
                parser(first, last, context, skip, flags, success));           \
        } else {                                                               \
            attr_type attr{};                                                  \
            parser(first, last, context, skip, flags, success, attr);          \
            return attr;                                                       \
        }                                                                      \
    }                                                                          \
                                                                               \
    template<                                                                  \
        typename Iter,                                                         \
        typename Sentinel,                                                     \
        typename Context,                                                      \
        typename SkipParser,                                                   \
        typename Attribute>                                                    \
    void parse_rule(                                                           \
        decltype(rule_name_)::parser_type::tag_type *,                         \
        Iter & first,                                                          \
        Sentinel last,                                                         \
        Context const & context,                                               \
        SkipParser const & skip,                                               \
        boost::parser::detail::flags flags,                                    \
        bool & success,                                                        \
        bool & dont_assign,                                                    \
        Attribute & retval)                                                    \
    {                                                                          \
        auto const & parser = BOOST_PARSER_PP_CAT(rule_name_, _def);           \
        using attr_t =                                                         \
            decltype(parser(first, last, context, skip, flags, success));      \
        if constexpr (boost::parser::detail::is_nope_v<attr_t>) {              \
            parser(first, last, context, skip, flags, success);                \
        } else {                                                               \
            parser(first, last, context, skip, flags, success, retval);        \
        }                                                                      \
    }
\end{code}

Now that we have the \ci{doubles} parser, we can use it like we might any other parser:

\begin{code}
auto const result = bp::parse(input, doubles, bp::ws);
\end{code}

The full program:

\begin{code}
#include <boost/parser/parser.hpp>

#include <deque>
#include <iostream>
#include <string>


namespace bp = boost::parser;


bp::rule<struct doubles, std::vector<double>> doubles = "doubles";
auto const doubles_def = bp::double_ % ',';
BOOST_PARSER_DEFINE_RULES(doubles);

int main()
{
    std::cout << "Please enter a list of doubles, separated by commas. ";
    std::string input;
    std::getline(std::cin, input);

    auto const result = bp::parse(input, doubles, bp::ws);

    if (result) {
        std::cout << "You entered:\n";
        for (double x : *result) {
            std::cout << x << "\n";
        }
    } else {
        std::cout << "Parse failure.\n";
    }
}
\end{code}

All this is intended to introduce the notion of \ci{rules}. It still may be a bit unclear why you would want to use \ci{rules}. The use cases for, and lots of detail about, \ci{rules} is in a later section, More About Rules.

\begin{marker}[title=Note ]
The existence of \ci{rules} means that will probably never have to write a low-level parser. You can just put existing parsers together into \ci{rules} instead. 
\end{marker}

\section{{Parsing into \ci{struct}s and \ci{class}es}{Parsing into structs and classes}}

So far, we've seen only simple parsers that parse the same value repeatedly (with or without commas and spaces). It's also very common to parse a few values in a specific sequence. Let's say you want to parse an employee record. Here's a parser you might write:

\begin{code}
namespace bp = boost::parser;
auto employee_parser = bp::lit("employee")
    >> '{'
    >> bp::int_ >> ','
    >> quoted_string >> ','
    >> quoted_string >> ','
    >> bp::double_
    >> '}';
\end{code}

The attribute type for \ci{employee\_parser} is \ci{boost::parser::tuple<int,\ std::string,\ std::string,\ double>}. That's great, in that you got all the parsed data for the record without having to write any semantic actions. It's not so great that you now have to get all the individual elements out by their indices, using \ci{get()}. It would be much nicer to parse into the final data structure that your program is going to use. This is often some \ci{struct} or \ci{class}. Boost.Parser supports parsing into arbitrary aggregate \ci{struct}s, and non-aggregates that are constructible from the tuple at hand.

\subsection{Aggregate types as attributes}

If we have a \ci{struct} that has data members of the same types listed in the \ci{boost::parser::tuple} attribute type for \ci{employee\_parser}, it would be nice to parse directly into it, instead of parsing into a tuple and then constructing our \ci{struct} later. Fortunately, this just works in Boost.Parser. Here is an example of parsing straight into a compatible aggregate type.

\begin{code}
#include <boost/parser/parser.hpp>

#include <iostream>
#include <string>


struct employee
{
    int age;
    std::string surname;
    std::string forename;
    double salary;
};

namespace bp = boost::parser;

int main()
{
    std::cout << "Enter employee record. ";
    std::string input;
    std::getline(std::cin, input);

    auto quoted_string = bp::lexeme['"' >> +(bp::char_ - '"') >> '"'];
    auto employee_p = bp::lit("employee")
        >> '{'
        >> bp::int_ >> ','
        >> quoted_string >> ','
        >> quoted_string >> ','
        >> bp::double_
        >> '}';

    employee record;
    auto const result = bp::parse(input, employee_p, bp::ws, record);

    if (result) {
        std::cout << "You entered:\nage:      " << record.age
                  << "\nsurname:  " << record.surname
                  << "\nforename: " << record.forename
                  << "\nsalary  : " << record.salary << "\n";
    } else {
        std::cout << "Parse failure.\n";
    }
}
\end{code}

Unfortunately, this is taking advantage of the loose attribute assignment logic; the \ci{employee\_parser} parser still has a \ci{boost::parser::tuple} attribute. See The \ci{parse()} API for a description of attribute out-param compatibility.

For this reason, it's even more common to want to make a rule that returns a specific type like \ci{employee}. Just by giving the rule a \ci{struct} type, we make sure that this parser always generates an \ci{employee} struct as its attribute, no matter where it is in the parse. If we made a simple parser \ci{P} that uses the \ci{employee\_p} rule, like \ci{bp::int\ >>\ employee\_p}, \ci{P}'s attribute type would be \ci{boost::parser::tuple<int,\ employee>}.

\begin{code}
#include <boost/parser/parser.hpp>

#include <iostream>
#include <string>


struct employee
{
    int age;
    std::string surname;
    std::string forename;
    double salary;
};

namespace bp = boost::parser;

bp::rule<struct quoted_string, std::string> quoted_string = "quoted name";
bp::rule<struct employee_p, employee> employee_p = "employee";

auto quoted_string_def = bp::lexeme['"' >> +(bp::char_ - '"') >> '"'];
auto employee_p_def = bp::lit("employee")
    >> '{'
    >> bp::int_ >> ','
    >> quoted_string >> ','
    >> quoted_string >> ','
    >> bp::double_
    >> '}';

BOOST_PARSER_DEFINE_RULES(quoted_string, employee_p);

int main()
{
    std::cout << "Enter employee record. ";
    std::string input;
    std::getline(std::cin, input);

    static_assert(std::is_aggregate_v<std::decay_t<employee &>>);

    auto const result = bp::parse(input, employee_p, bp::ws);

    if (result) {
        std::cout << "You entered:\nage:      " << result->age
                  << "\nsurname:  " << result->surname
                  << "\nforename: " << result->forename
                  << "\nsalary  : " << result->salary << "\n";
    } else {
        std::cout << "Parse failure.\n";
    }
}
\end{code}

Just as you can pass a \ci{struct} as an out-param to \ci{parse()} when the parser's attribute type is a tuple, you can also pass a tuple as an out-param to \ci{parse()} when the parser's attribute type is a struct:

\begin{code}
// Using the employee_p rule from above, with attribute type employee...
boost::parser::tuple<int, std::string, std::string, double> tup;
auto const result = bp::parse(input, employee_p, bp::ws, tup); // Ok!
\end{code}

\begin{marker}[title=Important ]
This automatic use of \ci{struct}s as if they were tuples depends on a bit of metaprogramming. Due to compiler limits, the metaprogram that detects the number of data members of a \ci{struct} is limited to a maximum number of members. Fortunately, that limit is configurable; see \ci{BOOST\_PARSER\_MAX\_AGGREGATE\_SIZE}. 
\end{marker}

\subsection{{General \ci{class} types as attributes}{General class types as attributes}}

Many times you don't have an aggregate struct that you want to produce from your parse. It would be even nicer than the aggregate code above if Boost.Parser could detect that the members of a tuple that is produced as an attribute are usable as the arguments to some type's constructor. So, Boost.Parser does that.

\begin{code}
#include <boost/parser/parser.hpp>

#include <iostream>
#include <string>


namespace bp = boost::parser;

int main()
{
    std::cout << "Enter a string followed by two unsigned integers. ";
    std::string input;
    std::getline(std::cin, input);

    constexpr auto string_uint_uint =
        bp::lexeme[+(bp::char_ - ' ')] >> bp::uint_ >> bp::uint_;
    std::string string_from_parse;
    if (parse(input, string_uint_uint, bp::ws, string_from_parse))
        std::cout << "That yields this string: " << string_from_parse << "\n";
    else
        std::cout << "Parse failure.\n";

    std::cout << "Enter an unsigned integer followed by a string. ";
    std::getline(std::cin, input);
    std::cout << input << "\n";

    constexpr auto uint_string = bp::uint_ >> +bp::char_;
    std::vector<std::string> vector_from_parse;
    if (parse(input, uint_string, bp::ws, vector_from_parse)) {
        std::cout << "That yields this vector of strings:\n";
        for (auto && str : vector_from_parse) {
            std::cout << "  '" << str << "'\n";
        }
    } else {
        std::cout << "Parse failure.\n";
    }
}
\end{code}

Let's look at the first parse.

\begin{code}
constexpr auto string_uint_uint =
    bp::lexeme[+(bp::char_ - ' ')] >> bp::uint_ >> bp::uint_;
std::string string_from_parse;
if (parse(input, string_uint_uint, bp::ws, string_from_parse))
    std::cout << "That yields this string: " << string_from_parse << "\n";
else
    std::cout << "Parse failure.\n";
\end{code}

Here, we use the parser \ci{string\_uint\_uint}, which produces a \ci{boost::parser::tuple<std::string,\ unsigned\ int,\ unsigned\ int>} attribute. When we try to parse that into an out-param \ci{std::string} attribute, it just works. This is because \ci{std::string} has a constructor that takes a \ci{std::string}, an offset, and a length. Here's the other parse:

\begin{code}
constexpr auto uint_string = bp::uint_ >> +bp::char_;
std::vector<std::string> vector_from_parse;
if (parse(input, uint_string, bp::ws, vector_from_parse)) {
    std::cout << "That yields this vector of strings:\n";
    for (auto && str : vector_from_parse) {
        std::cout << "  '" << str << "'\n";
    }
} else {
    std::cout << "Parse failure.\n";
}
\end{code}

Now we have the parser \ci{uint\_string}, which produces \ci{boost::parser::tuple<unsigned\ int,\ std::string>} attribute --- the two \ci{char}s at the end combine into a \ci{std::string}. Those two values can be used to construct a \ci{std::vector<std::string>}, via the count, \ci{T} constructor.

Just like with using aggregates in place of tuples, non-aggregate \ci{class} types can be substituted for tuples in most places. That includes using a non-aggregate \ci{class} type as the attribute type of a \ci{rule}.

However, while compatible tuples can be substituted for aggregates, you \textbf{can't} substitute a tuple for some \ci{class} type \ci{T} just because the tuple could have been used to construct \ci{T}. Think of trying to invert the substitution in the second parse above. Converting a \ci{std::vector<std::string>} into a \ci{boost::parser::tuple<unsigned\ int,\ std::string>} makes no sense.

\section{Alternative Parsers}

Frequently, you need to parse something that might have one of several forms. \ci{operator\textbar{}} is overloaded to form alternative parsers. For example:

\begin{code}
namespace bp = boost::parser;
auto const parser_1 = bp::int_ | bp::eps;
\end{code}

\ci{parser\_1} matches an integer, or if that fails, it matches \emph{epsilon}, the empty string. This is equivalent to writing:

\begin{code}
namespace bp = boost::parser;
auto const parser_2 = -bp::int_;
\end{code}

However, neither \ci{parser\_1} nor \ci{parser\_2} is equivalent to writing this:

\begin{code}
namespace bp = boost::parser;
auto const parser_3 = bp::eps | bp::int_; // Does not do what you think.
\end{code}

The reason is that alternative parsers try each of their subparsers, one at a time, and stop on the first one that matches. \emph{Epsilon} matches anything, since it is zero length and consumes no input. It even matches the end of input. This means that \ci{parser\_3} is equivalent to \ci{eps} by itself.

\begin{marker}[title=Note ]
For this reason, writing \ci{eps\ \textbar{}\ p} for any parser p is considered a bug. Debug builds will assert when \ci{eps\ \textbar{}\ p} is encountered. 
\end{marker}

\begin{marker}[title=Warning ]
This kind of error is very common when \ci{eps} is involved, and also very easy to detect. However, it is possible to write \ci{P1\ >>\ P2}, where \ci{P1} is a prefix of \ci{P2}, such as \ci{int\_\ \textbar{}\ int\ >>\ int\_}, or \ci{repeat(4){[}hex\_digit{]}\ \textbar{}\ repeat(8){[}hex\_digit{]}}. This is almost certainly an error, but is impossible to detect in the general case --- remember that \ci{rules} can be separately compiled, and consider a pair of rules whose associated \ci{\_def} parsers are \ci{int\_} and \ci{int\_\ >>\ int\_}, respectively. 
\end{marker}

\section{Parsing Quoted Strings}

It is very common to need to parse quoted strings. Quoted strings are slightly tricky, though, when using a skipper (and you should be using a skipper 99\% of the time). You don't want to allow arbitrary whitespace in the middle of your strings, and you also don't want to remove all whitespace from your strings. Both of these things will happen with the typical skipper, \ci{ws}.

So, here is how most people would write a quoted string parser:

\begin{code}
namespace bp = boost::parser;
const auto string = bp::lexeme['"' >> *(bp::char_ - '"') > '"'];
\end{code}

Some things to note:

\begin{itemize}
\item
  the result is a string;
\item
  the quotes are not included in the result;
\item
  there is an expectation point before the close-quote;
\item
  the use of \ci{lexeme{[}{]}} disables skipping in the parser, and it must be written around the quotes, not around the \ci{operator*} expression; and
\item
  there's no way to write a quote in the middle of the string.
\end{itemize}

This is a very common pattern. I have written a quoted string parser like this dozens of times. The parser above is the quick-and-dirty version. A more robust version would be able to handle escaped quotes within the string, and then would immediately also need to support escaped escape characters.

Boost.Parser provides \ci{quoted\_string} to use in place of this very common pattern. It supports quote- and escaped-character-escaping, using backslash as the escape character.

\begin{code}
namespace bp = boost::parser;

auto result1 = bp::parse("\"some text\"", bp::quoted_string, bp::ws);
assert(result1);
std::cout << *result1 << "\n"; // Prints: some text

auto result2 =
    bp::parse("\"some \\\"text\\\"\"", bp::quoted_string, bp::ws);
assert(result2);
std::cout << *result2 << "\n"; // Prints: some "text"
\end{code}

As common as this use case is, there are very similar use cases that it does not cover. So, \ci{quoted\_string} has some options. If you call it with a single character, it returns a \ci{quoted\_string} that uses that single character as the quote-character.

\begin{code}
auto result3 = bp::parse("!some text!", bp::quoted_string('!'), bp::ws);
assert(result3);
std::cout << *result3 << "\n"; // Prints: some text
\end{code}

You can also supply a range of characters. One of the characters from the range must quote both ends of the string; mismatches are not allowed. Think of how Python allows you to quote a string with either \ci{'{}"'{}} or \ci{'{}\textbackslash{}'{}'{}}, but the same character must be used on both sides.

\begin{code}
auto result4 = bp::parse("'some text'", bp::quoted_string("'\""), bp::ws);
assert(result4);
std::cout << *result4 << "\n"; // Prints: some text
\end{code}

Another common thing to do in a quoted string parser is to recognize escape sequences. If you have simple escape sequencecs that do not require any real parsing, like say the simple escape sequences from C++, you can provide a \ci{symbols} object as well. The template parameter \ci{T} to \ci{symbols<T>} must be \ci{char} or \ci{char32\_t}. You don't need to include the escaped backslash or the escaped quote character, since those always work.

\begin{code}
// the c++ simple escapes
bp::symbols<char> const escapes = {
    {"'", '\''},
    {"?", '\?'},
    {"a", '\a'},
    {"b", '\b'},
    {"f", '\f'},
    {"n", '\n'},
    {"r", '\r'},
    {"t", '\t'},
    {"v", '\v'}};
auto result5 =
    bp::parse("\"some text\r\"", bp::quoted_string('"', escapes), bp::ws);
assert(result5);
std::cout << *result5 << "\n"; // Prints (with a CRLF newline): some text
\end{code}

\section{Parsing In Detail}

Now that you've seen some examples, let's see how parsing works in a bit more detail. Consider this example.

\begin{code}
namespace bp = boost::parser;
auto int_pair = bp::int_ >> bp::int_;         // Attribute: tuple<int, int>
auto int_pairs_plus = +int_pair >> bp::int_;  // Attribute: tuple<std::vector<tuple<int, int>>, int>
\end{code}

\ci{int\_pairs\_plus} must match a pair of \ci{int}s (using \ci{int\_pair}) one or more times, and then must match an additional \ci{int}. In other words, it matches any odd number (greater than 1) of \ci{int}s in the input. Let's look at how this parse proceeds.

\begin{code}
auto result = bp::parse("1 2 3", int_pairs_plus, bp::ws);
\end{code}

At the beginning of the parse, the top level parser uses its first subparser (if any) to start parsing. So, \ci{int\_pairs\_plus}, being a sequence parser, would pass control to its first parser \ci{+int\_pair}. Then \ci{+int\_pair} would use \ci{int\_pair} to do its parsing, which would in turn use \ci{bp::int\_}. This creates a stack of parsers, each one using a particular subparser.

Step 1) The input is \ci{"1\ 2\ 3"}, and the stack of active parsers is \ci{int\_pairs\_plus} -> \ci{+int\_pair} -> \ci{int\_pair} -> \ci{bp::int\_}. (Read "-\textgreater" as "uses".) This parses \ci{"1"}, and the whitespace after is skipped by \ci{bp::ws}. Control passes to the second \ci{bp::int\_} parser in \ci{int\_pair}.

Step 2) The input is \ci{"2\ 3"} and the stack of parsers looks the same, except the active parser is the second \ci{bp::int\_} from \ci{int\_pair}. This parser consumes \ci{"2"} and then \ci{bp::ws} skips the subsequent space. Since we've finished with \ci{int\_pair}'s match, its \ci{boost::parser::tuple<int,\ int>} attribute is complete. It's parent is \ci{+int\_pair}, so this tuple attribute is pushed onto the back of \ci{+int\_pair}'s attribute, which is a \ci{std::vector<boost::parser::tuple<int,\ int>>}. Control passes up to the parent of \ci{int\_pair}, \ci{+int\_pair}. Since \ci{+int\_pair} is a one-or-more parser, it starts a new iteration; control passes to \ci{int\_pair} again.

Step 3) The input is \ci{"3"} and the stack of parsers looks the same, except the active parser is the first \ci{bp::int\_} from \ci{int\_pair} again, and we're in the second iteration of \ci{+int\_pair}. This parser consumes \ci{"3"}. Since this is the end of the input, the second \ci{bp::int\_} of \ci{int\_pair} does not match. This partial match of \ci{"3"} should not count, since it was not part of a full match. So, \ci{int\_pair} indicates its failure, and \ci{+int\_pair} stops iterating. Since it did match once, \ci{+int\_pair} does not fail; it is a zero-or-more parser; failure of its subparser after the first success does not cause it to fail. Control passes to the next parser in sequence within \ci{int\_pairs\_plus}.

Step 4) The input is \ci{"3"} again, and the stack of parsers is \ci{int\_pairs\_plus} -> \ci{bp::int\_}. This parses the \ci{"3"}, and the parse reaches the end of input. Control passes to \ci{int\_pairs\_plus}, which has just successfully matched with all parser in its sequence. It then produces its attribute, a \ci{boost::parser::tuple<std::vector<boost::parser::tuple<int,\ int>>,\ int>}, which gets returned from \ci{bp::parse()}.

Something to take note of between Steps \#3 and \#4: at the beginning of \#4, the input position had returned to where is was at the beginning of \#3. This kind of backtracking happens in alternative parsers when an alternative fails. The next page has more details on the semantics of backtracking.

\subsection{Parsers in detail}

So far, parsers have been presented as somewhat abstract entities. You may be wanting more detail. A Boost.Parser parser \ci{P} is an invocable object with a pair of call operator overloads. The two functions are very similar, and in many parsers one is implemented in terms of the other. The first function does the parsing and returns the default attribute for the parser. The second function does exactly the same parsing, but takes an out-param into which it writes the attribute for the parser. The out-param does not need to be the same type as the default attribute, but they need to be compatible.

Compatibility means that the default attribute is assignable to the out-param in some fashion. This usually means direct assignment, but it may also mean a tuple -> aggregate or aggregate -> tuple conversion. For sequence types, compatibility means that the sequence type has \ci{insert} or \ci{push\_back} with the usual semantics. This means that the parser \ci{+boost::parser::int\_} can fill a \ci{std::set<int>} just as well as a \ci{std::vector<int>}.

Some parsers also have additional state that is required to perform a match. For instance, \ci{char\_} parsers can be parameterized with a single code point to match; the exact value of that code point is stored in the parser object.

No parser has direct support for all the operations defined on parsers (\ci{operator\textbar{}}, \ci{operator>>}, etc.). Instead, there is a template called \ci{parser\_interface} that supports all of these operations. \ci{parser\_interface} wraps each parser, storing it as a data member, adapting it for general use. You should only ever see \ci{parser\_interface} in the debugger, or possibly in some of the reference documentation. You should never have to write it in your own code.

\section{Backtracking}

As described in the previous page, backtracking occurs when the parse attempts to match the current parser \ci{P}, matches part of the input, but fails to match all of \ci{P}. The part of the input consumed during the parse of \ci{P} is essentially "given back".

This is necessary because \ci{P} may consist of subparsers, and each subparser that succeeds will try to consume input, produce attributes, etc. When a later subparser fails, the parse of \ci{P} fails, and the input must be rewound to where it was when \ci{P} started its parse, not where the latest matching subparser stopped.

Alternative parsers will often evaluate multiple subparsers one at a time, advancing and then restoring the input position, until one of the subparsers succeeds. Consider this example.

\begin{code}
namespace bp = boost::parser;
auto const parser = repeat(53)[other_parser] | repeat(10)[other_parser];
\end{code}

Evaluating \ci{parser} means trying to match \ci{other\_parser} 53 times, and if that fails, trying to match \ci{other\_parser} 10 times. Say you parse input that matches \ci{other\_parser} 11 times. \ci{parser} will match it. It will also evaluate \ci{other\_parser} 21 times during the parse.

The attributes of the \ci{repeat(53){[}other\_parser{]}} and \ci{repeat(10){[}other\_parser{]}} are each \ci{std::vector<}\emph{\ci{ATTR}}\ci{(other\_parser)>}; let's say that \emph{\ci{ATTR}}\ci{(other\_parser)} is \ci{int}. The attribute of \ci{parser} as a whole is the same, \ci{std::vector<int>}. Since \ci{other\_parser} is busy producing \ci{int}s --- 21 of them to be exact --- you may be wondering what happens to the ones produced during the evaluation of \ci{repeat(53){[}other\_parser{]}} when it fails to find all 53 inputs. Its \ci{std::vector<int>} will contain 11 \ci{int}s at that point.

When a repeat-parser fails, and attributes are being generated, it clears its container. This applies to parsers such as the ones above, but also all the other repeat parsers, including ones made using \ci{operator+} or \ci{operator*}.

So, at the end of a successful parse by \ci{parser} of 10 inputs (since the right side of the alternative only eats 10 repetitions), the \ci{std::vector<int>} attribute of \ci{parser} would contain 10 \ci{int}s.

\begin{marker}[title=Note ]
Users of Boost.Spirit may be familiar with the \ci{hold{[}{]}} directive. Because of the behavior described above, there is no such directive in Boost.Parser. 
\end{marker}

\subsection{Expectation points}

Ok, so if parsers all try their best to match the input, and are all-or-nothing, doesn't that leave room for all kinds of bad input to be ignored? Consider the top-level parser from the Parsing JSON example.

\begin{code}
auto const value_p_def =
    number | bp::bool_ | null | string | array_p | object_p;
\end{code}

What happens if I use this to parse \ci{"\textbackslash{}""}? The parse tries \ci{number}, fails. It then tries \ci{bp::bool\_}, fails. Then \ci{null} fails too. Finally, it starts parsing \ci{string}. Good news, the first character is the open-quote of a JSON string. Unfortunately, that's also the end of the input, so \ci{string} must fail too. However, we probably don't want to just give up on parsing \ci{string} now and try \ci{array\_p}, right? If the user wrote an open-quote with no matching close-quote, that's not the prefix of some later alternative of \ci{value\_p\_def}; it's ill-formed JSON. Here's the parser for the \ci{string} rule:

\begin{code}
auto const string_def = bp::lexeme['"' >> *(string_char - '"') > '"'];
\end{code}

Notice that \ci{operator>} is used on the right instead of \ci{operator>>}. This indicates the same sequence operation as \ci{operator>>}, except that it also represents an expectation. If the parse before the \ci{operator>} succeeds, whatever comes after it \textbf{must} also succeed. Otherwise, the top-level parse is failed, and a diagnostic is emitted. It will say something like "Expected '"'{} here.", quoting the line, with a caret pointing to the place in the input where it expected the right-side match.

Choosing to use \ci{>} versus \ci{>>} is how you indicate to Boost.Parser that parse failure is or is not a hard error, respectively.

\section{Symbol Tables}

When writing a parser, it often comes up that there is a set of strings that, when parsed, are associated with a set of values one-to-one. It is tedious to write parsers that recognize all the possible input strings when you have to associate each one with an attribute via a semantic action. Instead, we can use a symbol table.

Say we want to parse Roman numerals, one of the most common work-related parsing problems. We want to recognize numbers that start with any number of "M"s, representing thousands, followed by the hundreds, the tens, and the ones. Any of these may be absent from the input, but not all. Here are three symbol Boost.Parser tables that we can use to recognize ones, tens, and hundreds values, respectively:

\begin{code}
bp::symbols<int> const ones = {
    {"I", 1},
    {"II", 2},
    {"III", 3},
    {"IV", 4},
    {"V", 5},
    {"VI", 6},
    {"VII", 7},
    {"VIII", 8},
    {"IX", 9}};

bp::symbols<int> const tens = {
    {"X", 10},
    {"XX", 20},
    {"XXX", 30},
    {"XL", 40},
    {"L", 50},
    {"LX", 60},
    {"LXX", 70},
    {"LXXX", 80},
    {"XC", 90}};

bp::symbols<int> const hundreds = {
    {"C", 100},
    {"CC", 200},
    {"CCC", 300},
    {"CD", 400},
    {"D", 500},
    {"DC", 600},
    {"DCC", 700},
    {"DCCC", 800},
    {"CM", 900}};
\end{code}

A \ci{symbols} maps strings of \ci{char} to their associated attributes. The type of the attribute must be specified as a template parameter to \ci{symbols} --- in this case, \ci{int}.

Any "M"s we encounter should add 1000 to the result, and all other values come from the symbol tables. Here are the semantic actions we'll need to do that:

\begin{code}
int result = 0;
auto const add_1000 = [&result](auto & ctx) { result += 1000; };
auto const add = [&result](auto & ctx) { result += _attr(ctx); };
\end{code}

\ci{add\_1000} just adds \ci{1000} to \ci{result}. \ci{add} adds whatever attribute is produced by its parser to \ci{result}.

Now we just need to put the pieces together to make a parser:

\begin{code}
using namespace bp::literals;
auto const parser =
    *'M'_l[add_1000] >> -hundreds[add] >> -tens[add] >> -ones[add];
\end{code}

We've got a few new bits in play here, so let's break it down. \ci{'{}M'{}\_l} is a \emph{literal parser}. That is, it is a parser that parses a literal \ci{char}, code point, or string. In this case, a \ci{char} \ci{'{}M'{}} is being parsed. The \ci{\_l} bit at the end is a UDL suffix that you can put after any \ci{char}, \ci{char32\_t}, or \ci{char\ const\ *} to form a literal parser. You can also make a literal parser by writing \ci{lit()}, passing an argument of one of the previously mentioned types.

Why do we need any of this, considering that we just used a literal \ci{'{},'{}} in our previous example? The reason is that \ci{'{}M'{}} is not used in an expression with another Boost.Parser parser. It is used within \ci{*'{}M'{}\_l{[}add\_1000{]}}. If we'd written \ci{*'{}M'{}{[}add\_1000{]}}, clearly that would be ill-formed; \ci{char} has no \ci{operator*}, nor an \ci{operator{[}{]}}, associated with it.

\begin{marker}[title=Tip ]
Any time you want to use a \ci{char}, \ci{char32\_t}, or string literal in a Boost.Parser parser, write it as-is if it is combined with a preexisting Boost.Parser subparser \ci{p}, as in \ci{'{}x'{}\ >>\ p}. Otherwise, you need to wrap it in a call to \ci{lit()}, or use the \ci{\_l} UDL suffix. 
\end{marker}

On to the next bit: \ci{-hundreds{[}add{]}}. By now, the use of the index operator should be pretty familiar; it associates the semantic action \ci{add} with the parser \ci{hundreds}. The \ci{operator-} at the beginning is new. It means that the parser it is applied to is optional. You can read it as "zero or one". So, if \ci{hundreds} is not successfully parsed after \ci{*'{}M'{}{[}add\_1000{]}}, nothing happens, because \ci{hundreds} is allowed to be missing --- it's optional. If \ci{hundreds} is parsed successfully, say by matching \ci{"CC"}, the resulting attribute, \ci{200}, is added to \ci{result} inside \ci{add}.

Here is the full listing of the program. Notice that it would have been inappropriate to use a whitespace skipper here, since the entire parse is a single number, so it was removed.

\begin{code}
#include <boost/parser/parser.hpp>

#include <iostream>
#include <string>


namespace bp = boost::parser;

int main()
{
    std::cout << "Enter a number using Roman numerals. ";
    std::string input;
    std::getline(std::cin, input);

    bp::symbols<int> const ones = {
        {"I", 1},
        {"II", 2},
        {"III", 3},
        {"IV", 4},
        {"V", 5},
        {"VI", 6},
        {"VII", 7},
        {"VIII", 8},
        {"IX", 9}};

    bp::symbols<int> const tens = {
        {"X", 10},
        {"XX", 20},
        {"XXX", 30},
        {"XL", 40},
        {"L", 50},
        {"LX", 60},
        {"LXX", 70},
        {"LXXX", 80},
        {"XC", 90}};

    bp::symbols<int> const hundreds = {
        {"C", 100},
        {"CC", 200},
        {"CCC", 300},
        {"CD", 400},
        {"D", 500},
        {"DC", 600},
        {"DCC", 700},
        {"DCCC", 800},
        {"CM", 900}};

    int result = 0;
    auto const add_1000 = [&result](auto & ctx) { result += 1000; };
    auto const add = [&result](auto & ctx) { result += _attr(ctx); };

    using namespace bp::literals;
    auto const parser =
        *'M'_l[add_1000] >> -hundreds[add] >> -tens[add] >> -ones[add];

    if (bp::parse(input, parser) && result != 0)
        std::cout << "That's " << result << " in Arabic numerals.\n";
    else
        std::cout << "That's not a Roman number.\n";
}
\end{code}

\begin{marker}[title=Important ]
\ci{symbols} stores all its strings in UTF-32 internally. If you do Unicode or ASCII parsing, this will not matter to you at all. If you do non-Unicode parsing of a character encoding that is not a subset of Unicode (EBCDIC, for instance), it could cause problems. See the section on Unicode Support for more information. 
\end{marker}

\subsection{Diagnostic messages}

Just like with a \ci{rule}, you can give a \ci{symbols} a bit of diagnostic text that will be used in error messages generated by Boost.Parser when the parse fails at an expectation point, as described in Error Handling and Debugging. See the \ci{symbols} constructors for details.

\section{Mutable Symbol Tables}

The previous example showed how to use a symbol table as a fixed lookup table. What if we want to add things to the table during the parse? We can do that, but we need to do so within a semantic action. First, here is our symbol table, already with a single value in it:

\begin{code}
bp::symbols<int> const symbols = {{"c", 8}};
assert(parse("c", symbols));
\end{code}

No surprise that it works to use the symbol table as a parser to parse the one string in the symbol table. Now, here's our parser:

\begin{code}
auto const parser = (bp::char_ >> bp::int_)[add_symbol] >> symbols;
\end{code}

Here, we've attached the semantic action not to a simple parser like \ci{double\_}, but to the sequence parser \ci{(bp::char\_\ >>\ bp::int\_)}. This sequence parser contains two parsers, each with its own attribute, so it produces two attributes as a tuple.

\begin{code}
auto const add_symbol = [&symbols](auto & ctx) {
    using namespace bp::literals;
    // symbols::insert() requires a string, not a single character.
    char chars[2] = {_attr(ctx)[0_c], 0};
    symbols.insert(ctx, chars, _attr(ctx)[1_c]);
};
\end{code}

Inside the semantic action, we can get the first element of the attribute tuple using UDLs provided by Boost.Hana, and \ci{boost::hana::tuple::operator{[}{]}()}. The first attribute, from the \ci{char\_}, is \ci{\_attr(ctx){[}0\_c{]}}, and the second, from the \ci{int\_}, is \ci{\_attr(ctx){[}1\_c{]}} (if \ci{boost::parser::tuple} aliases to \ci{std::tuple}, you'd use \ci{std::get} or \ci{boost::parser::get} instead). To add the symbol to the symbol table, we call \ci{insert()}.

\begin{code}
auto const parser = (bp::char_ >> bp::int_)[add_symbol] >> symbols;
\end{code}

During the parse, \ci{("X",\ 9)} is parsed and added to the symbol table. Then, the second \ci{'{}X'{}} is recognized by the symbol table parser. However:

\begin{code}
assert(!parse("X", symbols));
\end{code}

If we parse again, we find that \ci{"X"} did not stay in the symbol table. The fact that \ci{symbols} was declared const might have given you a hint that this would happen.

The full program:

\begin{code}
#include <boost/parser/parser.hpp>

#include <iostream>
#include <string>


namespace bp = boost::parser;

int main()
{
    bp::symbols<int> const symbols = {{"c", 8}};
    assert(parse("c", symbols));

    auto const add_symbol = [&symbols](auto & ctx) {
        using namespace bp::literals;
        // symbols::insert() requires a string, not a single character.
        char chars[2] = {_attr(ctx)[0_c], 0};
        symbols.insert(ctx, chars, _attr(ctx)[1_c]);
    };
    auto const parser = (bp::char_ >> bp::int_)[add_symbol] >> symbols;

    auto const result = parse("X 9 X", parser, bp::ws);
    assert(result && *result == 9);
    (void)result;

    assert(!parse("X", symbols));
}
\end{code}

\begin{marker}[title=Important ]
\ci{symbols} stores all its strings in UTF-32 internally. If you do Unicode or ASCII parsing, this will not matter to you at all. If you do non-Unicode parsing of a character encoding that is not a subset of Unicode (EBCDIC, for instance), it could cause problems. See the section on Unicode Support for more information. 
\end{marker}

It is possible to add symbols to a \ci{symbols} permanently. To do so, you have to use a mutable \ci{symbols} object \ci{s}, and add the symbols by calling \ci{s.insert\_for\_next\_parse()}, instead of \ci{s.insert()}. These two operations are orthogonal, so if you want to both add a symbol to the table for the current top-level parse, and leave it in the table for subsequent top-level parses, you need to call both functions.

It is also possible to erase a single entry from the symbol table, or to clear the symbol table entirely. Just as with insertion, there are versions of erase and clear for the current parse, and another that applies only to subsequent parses. The full set of operations can be found in the \ci{symbols} API docs.

{[}mpte There are two versions of each of the \ci{symbols} \ci{*\_for\_next\_parse()} functions --- one that takes a context, and one that does not. The one with the context is meant to be used within a semantic action. The one without the context is for use outside of any parse.{]}

\section{The Parsers And Their Uses}

Boost.Parser comes with all the parsers most parsing tasks will ever need. Each one is a \ci{constexpr} object, or a \ci{constexpr} function. Some of the non-functions are also callable, such as \ci{char\_}, which may be used directly, or with arguments, as in \ci{char\_}\ci{('{}a'{},\ '{}z'{})}. Any parser that can be called, whether a function or callable object, will be called a \emph{callable parser} from now on. Note that there are no nullary callable parsers; they each take one or more arguments.

Each callable parser takes one or more \emph{parse arguments}. A parse argument may be a value or an invocable object that accepts a reference to the parse context. The reference parameter may be mutable or constant. For example:

\begin{code}
struct get_attribute
{
    template<typename Context>
    auto operator()(Context & ctx)
    {
        return _attr(ctx);
    }
};
\end{code}

This can also be a lambda. For example:

\begin{code}
[](auto const & ctx) { return _attr(ctx); }
\end{code}

The operation that produces a value from a parse argument, which may be a value or a callable taking a parse context argument, is referred to as \emph{resolving} the parse argument. If a parse argument \ci{arg} can be called with the current context, then the resolved value of \ci{arg} is \ci{arg(ctx)}; otherwise, the resolved value is just \ci{arg}.

Some callable parsers take a \emph{parse predicate}. A parse predicate is not quite the same as a parse argument, because it must be a callable object, and cannot be a value. A parse predicate's return type must be contextually convertible to \ci{bool}. For example:

\begin{code}
struct equals_three
{
    template<typename Context>
    bool operator()(Context const & ctx)
    {
        return _attr(ctx) == 3;
    }
};
\end{code}

This may of course be a lambda:

\begin{code}
[](auto & ctx) { return _attr(ctx) == 3; }
\end{code}

The notional macro \emph{\ci{RESOLVE}}\ci{()} expands to the result of resolving a parse argument or parse predicate. You'll see it used in the rest of the documentation.

An example of how parse arguments are used:

\begin{code}
namespace bp = boost::parser;
// This parser matches one code point that is at least 'a', and at most
// the value of last_char, which comes from the globals.
auto last_char = [](auto & ctx) { return _globals(ctx).last_char; }
auto subparser = bp::char_('a', last_char);
\end{code}

Don't worry for now about what the globals are for now; the take-away is that you can make any argument you pass to a parser depend on the current state of the parse, by using the parse context:

\begin{code}
namespace bp = boost::parser;
// This parser parses two code points.  For the parse to succeed, the
// second one must be >= 'a' and <= the first one.
auto set_last_char = [](auto & ctx) { _globals(ctx).last_char = _attr(x); };
auto parser = bp::char_[set_last_char] >> subparser;
\end{code}

Each callable parser returns a new parser, parameterized using the arguments given in the invocation.

This table lists all the Boost.Parser parsers. For the callable parsers, a separate entry exists for each possible arity of arguments. For a parser \ci{p}, if there is no entry for \ci{p} without arguments, \ci{p} is a function, and cannot itself be used as a parser; it must be called. In the table below:

\begin{itemize}
\item
  each entry is a global object usable directly in your parsers, unless otherwise noted;
\item
  "code point" is used to refer to the elements of the input range, which assumes that the parse is being done in the Unicode-aware code path (if the parse is being done in the non-Unicode code path, read "code point" as "\ci{char}");
\item
  \emph{\ci{RESOLVE}}\ci{()} is a notional macro that expands to the resolution of parse argument or evaluation of a parse predicate (see The Parsers And Their Uses);
\item
  "\emph{\ci{RESOLVE}}\ci{(pred)\ ==\ true}" is a shorthand notation for "\emph{\ci{RESOLVE}}\ci{(pred)} is contextually convertible to \ci{bool} and \ci{true}"; likewise for \ci{false};
\item
  \ci{c} is a character of type \ci{char}, \ci{char8\_t}, or \ci{char32\_t};
\item
  \ci{str} is a string literal of type \ci{char\ const{[}{]}}, \ci{char8\_t\ const\ {[}{]}}, or \ci{char32\_t\ const\ {[}{]}};
\item
  \ci{pred} is a parse predicate;
\item
  \ci{arg0}, \ci{arg1}, \ci{arg2}, ... are parse arguments;
\item
  \ci{a} is a semantic action;
\item
  \ci{r} is an object whose type models \ci{parsable\_range};
\item
  \ci{p}, \ci{p1}, \ci{p2}, ... are parsers; and
\item
  \ci{escapes} is a \ci{symbols<T>} object, where \ci{T} is \ci{char} or \ci{char32\_t}.
\end{itemize}



\begin{marker}[title=Note ]
Some of the parsers in this table consume no input. All parsers consume the input they match unless otherwise stated in the table below. 
\end{marker}

\textbf{Table~26.6.~Parsers and Their Semantics}


\begin{marker}[title=Important ]
All the character parsers, like \ci{char\_}, \ci{cp} and \ci{cu} produce either \ci{char} or \ci{char32\_t} attributes. So when you see "\ci{std::string} if \emph{\ci{ATTR}}\ci{(p)} is \ci{char} or \ci{char32\_t}, otherwise \ci{std::vector<}\emph{\ci{ATTR}}\ci{(p)>}" in the table above, that effectively means that every sequences of character attributes get turned into a \ci{std::string}. The only time this does not happen is when you introduce your own rules with attributes using another character type (or use \ci{attribute} to do so). 
\end{marker}

\begin{marker}[title=Note ]
A slightly more complete description of the attributes generated by these parsers is in a subsequent section. The attributes are repeated here so you can use see all the properties of the parsers in one place. 
\end{marker}

If you have an integral type \ci{IntType} that is not covered by any of the Boost.Parser parsers, you can use a more verbose declaration to declare a parser for \ci{IntType}. If \ci{IntType} were unsigned, you would use \ci{uint\_parser}. If it were signed, you would use \ci{int\_parser}. For example:

\begin{code}
constexpr parser_interface<int_parser<IntType>> hex_int;
\end{code}

\ci{uint\_parser} and \ci{int\_parser} accept three more non-type template parameters after the type parameter. They are \ci{Radix}, \ci{MinDigits}, and \ci{MaxDigits}. \ci{Radix} defaults to \ci{10}, \ci{MinDigits} to \ci{1}, and \ci{MaxDigits} to \ci{-1}, which is a sentinel value meaning that there is no max number of digits.

So, if you wanted to parse exactly eight hexadecimal digits in a row in order to recognize Unicode character literals like C++ has (e.g. \ci{\textbackslash{}Udeadbeef}), you could use this parser for the digits at the end:

\begin{code}
constexpr parser_interface<uint_parser<unsigned int, 16, 8, 8>> hex_int;
\end{code}

\section{Directives}

A directive is an element of your parser that doesn't have any meaning by itself. Some are second-order parsers that need a first-order parser to do the actual parsing. Others influence the parse in some way. You can often spot a directive lexically by its use of \ci{{[}{]}}; directives always \ci{{[}{]}}. Non-directives might, but only when attaching a semantic action.

The directives that are second order parsers are technically directives, but since they are also used to create parsers, it is more useful just to focus on that. The directives \ci{repeat()} and \ci{if\_()} were already described in the section on parsers; we won't say much about them here.

\subsection{Interaction with sequence, alternative, and permutation parsers}

Sequence, alternative, and permutation parsers do not nest in most cases. (Let's consider just sequence parsers to keep thinkgs simple, but most of this logic applies to alternative parsers as well.) \ci{a\ >>\ b\ >>\ c} is the same as \ci{(a\ >>\ b)\ >>\ c} and \ci{a\ >>\ (b\ >>\ c)}, and they are each represented by a single \ci{seq\_parser} with three subparsers, \ci{a}, \ci{b}, and \ci{c}. However, if something prevents two \ci{seq\_parsers} from interacting directly, they \textbf{will} nest. For instance, \ci{lexeme{[}a\ >>\ b{]}\ >>\ c} is a \ci{seq\_parser} containing two parsers, \ci{lexeme{[}a\ >>\ b{]}} and \ci{c}. This is because \ci{lexeme{[}{]}} takes its given parser and wraps it in a \ci{lexeme\_parser}. This in turn turns off the sequence parser combining logic, since both sides of the second \ci{operator>>} in \ci{lexeme{[}a\ >>\ b{]}\ >>\ c} are not \ci{seq\_parsers}. Sequence parsers have several rules that govern what the overall attribute type of the parser is, based on the positions and attributes of it subparsers (see Attribute Generation). Therefore, it's important to know which directives create a new parser (and what kind), and which ones do not; this is indicated for each directive below.

\subsection{The directives}

\subsection{repeat()}

See The Parsers And Their Uses. Creates a \ci{repeat\_parser}.

\subsection{if\_()}

See The Parsers And Their Uses. Creates a \ci{seq\_parser}.

\subsection{omit{[}{]}}

\ci{omit{[}p{]}} disables attribute generation for the parser \ci{p}. Not only does \ci{omit{[}p{]}} have no attribute, but any attribute generation work that normally happens within \ci{p} is skipped.

This directive can be useful in cases like this: say you have some fairly complicated parser \ci{p} that generates a large and expensive-to-construct attribute. Now say that you want to write a function that just counts how many times \ci{p} can match a string (where the matches are non-overlapping). Instead of using \ci{p} directly, and building all those attributes, or rewriting \ci{p} without the attribute generation, use \ci{omit{[}{]}}.

Creates an \ci{omit\_parser}.

\subsection{raw{[}{]}}

\ci{raw{[}p{]}} changes the attribute from \emph{\ci{ATTR}}\ci{(p)} to to a view that delimits the subrange of the input that was matched by \ci{p}. The type of the view is \ci{subrange<I>}, where \ci{I} is the type of the iterator used within the parse. Note that this may not be the same as the iterator type passed to \ci{parse()}. For instance, when parsing UTF-8, the iterator passed to \ci{parse()} may be \ci{char8\_t\ const\ *}, but within the parse it will be a UTF-8 to UTF-32 transcoding (converting) iterator. Just like \ci{omit{[}{]}}, \ci{raw{[}{]}} causes all attribute-generation work within \ci{p} to be skipped.

Similar to the re-use scenario for \ci{omit{[}{]}} above, \ci{raw{[}{]}} could be used to find the \textbf{locations} of all non-overlapping matches of \ci{p} in a string.

Creates a \ci{raw\_parser}.

\subsection{string\_view{[}{]}}

\ci{string\_view{[}p{]}} is very similar to \ci{raw{[}p{]}}, except that it changes the attribute of \ci{p} to \ci{std::basic\_string\_view<C>}, where \ci{C} is the character type of the underlying range being parsed. \ci{string\_view{[}{]}} requires that the underlying range being parsed is contiguous. Since this can only be detected in C++20 and later, \ci{string\_view{[}{]}} is not available in C++17 mode.

Similar to the re-use scenario for \ci{omit{[}{]}} above, \ci{string\_view{[}{]}} could be used to find the \textbf{locations} of all non-overlapping matches of \ci{p} in a string. Whether \ci{raw{[}{]}} or \ci{string\_view{[}{]}} is more natural to use to report the locations depends on your use case, but they are essentially the same.

Creates a \ci{string\_view\_parser}.

\subsection{no\_case{[}{]}}

\ci{no\_case{[}p{]}} enables case-insensitive parsing within the parse of \ci{p}. This applies to the text parsed by \ci{char\_()}, \ci{string()}, and \ci{bool\_} parsers. The number parsers are already case-insensitive. The case-insensitivity is achieved by doing Unicode case folding on the text being parsed and the values in the parser being matched (see note below if you want to know more about Unicode case folding). In the non-Unicode code path, a full Unicode case folding is not done; instead, only the transformations of values less than \ci{0x100} are done. Examples:

\begin{code}
#include <boost/parser/transcode_view.hpp> // For as_utfN.

namespace bp = boost::parser;
auto const street_parser = bp::string(u8"Tobias Straße");
assert(!bp::parse("Tobias Strasse" | bp::as_utf32, street_parser));             // No match.
assert(bp::parse("Tobias Strasse" | bp::as_utf32, bp::no_case[street_parser])); // Match!

auto const alpha_parser = bp::no_case[bp::char_('a', 'z')];
assert(bp::parse("a" | bp::as_utf32, bp::no_case[alpha_parser])); // Match!
assert(bp::parse("B" | bp::as_utf32, bp::no_case[alpha_parser])); // Match!
\end{code}

Everything pretty much does what you'd naively expect inside \ci{no\_case{[}{]}}, except that the two-character range version of \ci{char\_} has a limitation. It only compares a code point from the input to its two arguments (e.g. \ci{'{}a'{}} and \ci{'{}z'{}} in the example above). It does not do anything special for multi-code point case folding expansions. For instance, \ci{char\_(U'{}ß'{},\ U'{}ß'{})} matches the input \ci{U"s"}, which makes sense, since \ci{U'{}ß'{}} expands to \ci{U"ss"}. However, that same parser \textbf{does not} match the input \ci{U"ß"}! In short, stick to pairs of code points that have single-code point case folding expansions. If you need to support the multi-expanding code points, use the other overload, like: \ci{char\_(U"abcd/*...*/ß")}.

\begin{marker}[title=Note ]
Unicode case folding is an operation that makes text uniformly one case, and if you do it to two bits of text \ci{A} and \ci{B}, then you can compare them bitwise to see if they are the same, except of case. Case folding may sometimes expand a code point into multiple code points (e.g. case folding \ci{"ẞ"} yields \ci{"ss"}. When such a multi-code point expansion occurs, the expanded code points are in the NFKC normalization form. 
\end{marker}

Creates a \ci{no\_case\_parser}.

\subsection{lexeme{[}{]}}

\ci{lexeme{[}p{]}} disables use of the skipper, if a skipper is being used, within the parse of \ci{p}. This is useful, for instance, if you want to enable skipping in most parts of your parser, but disable it only in one section where it doesn't belong. If you are skipping whitespace in most of your parser, but want to parse strings that may contain spaces, you should use \ci{lexeme{[}{]}}:

\begin{code}
namespace bp = boost::parser;
auto const string_parser = bp::lexeme['"' >> *(bp::char_ - '"') >> '"'];
\end{code}

Without \ci{lexeme{[}{]}}, our string parser would correctly match \ci{"foo\ bar"}, but the generated attribute would be \ci{"foobar"}.

Creates a \ci{lexeme\_parser}.

\subsection{skip{[}{]}}

\ci{skip{[}{]}} is like the inverse of \ci{lexeme{[}{]}}. It enables skipping in the parse, even if it was not enabled before. For example, within a call to \ci{parse()} that uses a skipper, let's say we have these parsers in use:

\begin{code}
namespace bp = boost::parser;
auto const one_or_more = +bp::char_;
auto const skip_or_skip_not_there_is_no_try = bp::lexeme[bp::skip[one_or_more] >> one_or_more];
\end{code}

The use of \ci{lexeme{[}{]}} disables skipping, but then the use of \ci{skip{[}{]}} turns it back on. The net result is that the first occurrence of \ci{one\_or\_more} will use the skipper passed to \ci{parse()}; the second will not.

\ci{skip{[}{]}} has another use. You can parameterize skip with a different parser to change the skipper just within the scope of the directive. Let's say we passed \ci{ws} to \ci{parse()}, and we're using these parsers somewhere within that \ci{parse()} call:

\begin{code}
namespace bp = boost::parser;
auto const zero_or_more = *bp::char_;
auto const skip_both_ways = zero_or_more >> bp::skip(bp::blank)[zero_or_more];
\end{code}

The first occurrence of \ci{zero\_or\_more} will use the skipper passed to \ci{parse()}, which is \ci{ws}; the second will use \ci{blank} as its skipper.

Creates a \ci{skip\_parser}.

\subsection{{merge{[}{]}, separate{[}{]}, and \ci{transform(f){[}{]}}}{merge{[}{]}, separate{[}{]}, and transform(f){[}{]}}}

These directives influence the generation of attributes. See Attribute Generation section for more details on them.

\ci{merge{[}{]}} and \ci{separate{[}{]}} create a copy of the given \ci{seq\_parser}.

\ci{transform(f){[}{]}} creates a \ci{tranform\_parser}.

\section{Combining Operations}

Certain overloaded operators are defined for all parsers in Boost.Parser. We've already seen some of them used in this tutorial, especially \ci{operator>>}, \ci{operator\textbar{}}, and \ci{operator\textbar{}\textbar{}}, which are used to form sequence parsers, alternative parsers, and permutation parsers, respectively.

Here are all the operator overloaded for parsers. In the tables below:

\begin{itemize}
\item
  \ci{c} is a character of type \ci{char} or \ci{char32\_t};
\item
  \ci{a} is a semantic action;
\item
  \ci{r} is an object whose type models \ci{parsable\_range} (see Concepts); and
\item
  \ci{p}, \ci{p1}, \ci{p2}, ... are parsers.
\end{itemize}

\begin{marker}[title=Note ]
Some of the expressions in this table consume no input. All parsers consume the input they match unless otherwise stated in the table below. 
\end{marker}

\textbf{Table~26.7.~Combining Operations and Their Semantics}



\hfill\break

\begin{marker}[title=Important ]
All the character parsers, like \ci{char\_}, \ci{cp} and \ci{cu} produce either \ci{char} or \ci{char32\_t} attributes. So when you see "\ci{std::string} if \emph{\ci{ATTR}}\ci{(p)} is \ci{char} or \ci{char32\_t}, otherwise \ci{std::vector<}\emph{\ci{ATTR}}\ci{(p)>}" in the table above, that effectively means that every sequences of character attributes get turned into a \ci{std::string}. The only time this does not happen is when you introduce your own rules with attributes using another character type (or use \ci{attribute} to do so). 
\end{marker}

There are a couple of special rules not captured in the table above:

First, the zero-or-more and one-or-more repetitions (\ci{operator*()} and \ci{operator+()}, respectively) may collapse when combined. For any parser \ci{p}, \ci{+(+p)} collapses to \ci{+p}; \ci{**p}, \ci{*+p}, and \ci{+*p} each collapse to just \ci{*p}.

Second, using \ci{eps} in an alternative parser as any alternative \textbf{except} the last one is a common source of errors; Boost.Parser disallows it. This is true because, for any parser \ci{p}, \ci{eps\ \textbar{}\ p} is equivalent to \ci{eps}, since \ci{eps} always matches. This is not true for \ci{eps} parameterized with a condition. For any condition \ci{cond}, \ci{eps(cond)} is allowed to appear anywhere within an alternative parser.

\begin{marker}[title=Note ]
When looking at Boost.Parser parsers in a debugger, or when looking at their reference documentation, you may see reference to the template \ci{parser\_interface}. This template exists to provide the operator overloads described above. It allows the parsers themselves to be very simple --- most parsers are just a struct with two member functions. \ci{parser\_interface} is essentially invisible when using Boost.Parser, and you should never have to name this template in your own code. 
\end{marker}

\section{Attribute Generation}

So far, we've seen several different types of attributes that come from different parsers, \ci{int} for \ci{int\_}, \ci{boost::parser::tuple<char,\ int>} for \ci{boost::parser::char\_\ >>\ boost::parser::int\_}, etc. Let's get into how this works with more rigor.

\begin{marker}[title=Note ]
Some parsers have no attribute at all. In the tables below, the type of the attribute is listed as "None." There is a non-\ci{void} type that is returned from each parser that lacks an attribute. This keeps the logic simple; having to handle the two cases --- \ci{void} or non-\ci{void} --- would make the library significantly more complicated. The type of this non-\ci{void} attribute associated with these parsers is an implementation detail. The type comes from the \ci{boost::parser::detail} namespace and is pretty useless. You should never see this type in practice. Within semantic actions, asking for the attribute of a non-attribute-producing parser (using \ci{\_attr(ctx)}) will yield a value of the special type \ci{boost::parser::none}. When calling \ci{parse()} in a form that returns the attribute parsed, when there is no attribute, simply returns \ci{bool}; this indicates the success of failure of the parse. 
\end{marker}

\begin{marker}[title=Warning ]
Boost.Parser assumes that all attributes are semi-regular (see \ci{std::semiregular}). Within the Boost.Parser code, attributes are assigned, moved, copy, and default constructed. There is no support for move-only or non-default-constructible types. 
\end{marker}

\subsection{The attribute type trait, attribute}

You can use \ci{attribute} (and the associated alias, \ci{attribute\_t}) to determine the attribute a parser would have if it were passed to \ci{parse()}. Since at least one parser (\ci{char\_}) has a polymorphic attribute type, \ci{attribute} also takes the type of the range being parsed. If a parser produces no attribute, \ci{attribute} will produce \ci{none}, not \ci{void}.

If you want to feed an iterator/sentinel pair to \ci{attribute}, create a range from it like so:

\begin{code}
constexpr auto parser = /* ... */;
auto first = /* ... */;
auto const last = /* ... */;

namespace bp = boost::parser;
// You can of course use std::ranges::subrange directly in C++20 and later.
using attr_type = bp::attribute_t<decltype(BOOST_PARSER_SUBRANGE(first, last)), decltype(parser)>;
\end{code}

There is no single attribute type for any parser, since a parser can be placed within \ci{omit{[}{]}}, which makes its attribute type \ci{none}. Therefore, \ci{attribute} cannot tell you what attribute your parser will produce under all circumstances; it only tells you what it would produce if it were passed to \ci{parse()}.

\subsection{Parser attributes}

This table summarizes the attributes generated for all Boost.Parser parsers. In the table below:

\begin{itemize}
\item
  \emph{\ci{RESOLVE}}\ci{()} is a notional macro that expands to the resolution of parse argument or evaluation of a parse predicate (see The Parsers And Their Uses); and
\item
  \ci{x} and \ci{y} represent arbitrary objects.
\end{itemize}

\textbf{Table~26.8.~Parsers and Their Attributes}



\hfill\break

\ci{char\_} is a bit odd, since its attribute type is polymorphic. When you use \ci{char\_} to parse text in the non-Unicode code path (i.e. a string of \ci{char}), the attribute is \ci{char}. When you use the exact same \ci{char\_} to parse in the Unicode-aware code path, all matching is code point based, and so the attribute type is the type used to represent code points, \ci{char32\_t}. All parsing of UTF-8 falls under this case.

Here, we're parsing plain \ci{char}s, meaning that the parsing is in the non-Unicode code path, the attribute of \ci{char\_} is \ci{char}:

\begin{code}
auto result = parse("some text", boost::parser::char_);
static_assert(std::is_same_v<decltype(result), std::optional<char>>));
\end{code}

When you parse UTF-8, the matching is done on a code point basis, so the attribute type is \ci{char32\_t}:

\begin{code}
auto result = parse("some text" | boost::parser::as_utf8, boost::parser::char_);
static_assert(std::is_same_v<decltype(result), std::optional<char32_t>>));
\end{code}

The good news is that usually you don't parse characters individually. When you parse with \ci{char\_}, you usually parse repetition of then, which will produce a \ci{std::string}, regardless of whether you're in Unicode parsing mode or not. If you do need to parse individual characters, and want to lock down their attribute type, you can use \ci{cp} and/or \ci{cu} to enforce a non-polymorphic attribute type.

\subsection{Combining operation attributes}

Combining operations of course affect the generation of attributes. In the tables below:

\begin{itemize}
\item
  \ci{m} and \ci{n} are parse arguments that resolve to integral values;
\item
  \ci{pred} is a parse predicate;
\item
  \ci{arg0}, \ci{arg1}, \ci{arg2}, ... are parse arguments;
\item
  \ci{a} is a semantic action; and
\item
  \ci{p}, \ci{p1}, \ci{p2}, ... are parsers that generate attributes.
\end{itemize}

\textbf{Table~26.9.~Combining Operations and Their Attributes}



\hfill\break

\begin{marker}[title=Important ]
All the character parsers, like \ci{char\_}, \ci{cp} and \ci{cu} produce either \ci{char} or \ci{char32\_t} attributes. So when you see "\ci{std::string} if \emph{\ci{ATTR}}\ci{(p)} is \ci{char} or \ci{char32\_t}, otherwise \ci{std::vector<}\emph{\ci{ATTR}}\ci{(p)>}" in the table above, that effectively means that every sequences of character attributes get turned into a \ci{std::string}. The only time this does not happen is when you introduce your own rules with attributes using another character type (or use \ci{attribute} to do so). 
\end{marker}

\begin{marker}[title=Important ]
In case you did not notice it above, adding a semantic action to a parser erases the parser's attribute. The attribute is still available inside the semantic action as \ci{\_attr(ctx)}. 
\end{marker}

There are a relatively small number of rules that define how sequence parsers and alternative parsers'{} attributes are generated. (Don't worry, there are examples below.)

\subsection{Sequence parser attribute rules}

The attribute generation behavior of sequence parsers is conceptually pretty simple:

\begin{itemize}
\item
  the attributes of subparsers form a tuple of values;
\item
  subparsers that do not generate attributes do not contribute to the sequence's attribute;
\item
  subparsers that do generate attributes usually contribute an individual element to the tuple result; except
\item
  when containers of the same element type are next to each other, or individual elements are next to containers of their type, the two adjacent attributes collapse into one attribute; and
\item
  if the result of all that is a degenerate tuple \ci{boost::parser::tuple<T>} (even if \ci{T} is a type that means "no attribute"), the attribute becomes \ci{T}.
\end{itemize}

More formally, the attribute generation algorithm works like this. For a sequence parser \ci{p}, let the list of attribute types for the subparsers of \ci{p} be \ci{a0,\ a1,\ a2,\ ...,\ an}.

We get the attribute of \ci{p} by evaluating a compile-time left fold operation, \ci{left-fold(\{a1,\ a2,\ ...,\ an\},\ tuple<a0>,\ OP)}. \ci{OP} is the combining operation that takes the current attribute type (initially \ci{boost::parser::tuple<a0>}) and the next attribute type, and returns the new current attribute type. The current attribute type at the end of the fold operation is the attribute type for \ci{p}.

\ci{OP} attempts to apply a series of rules, one at a time. The rules are noted as \ci{X\ >>\ Y\ ->\ Z}, where \ci{X} is the type of the current attribute, \ci{Y} is the type of the next attribute, and \ci{Z} is the new current attribute type. In these rules, \ci{C<T>} is a container of \ci{T}; \ci{none} is a special type that indicates that there is no attribute; \ci{T} is a type; \ci{CHAR} is a character type, either \ci{char} or \ci{char32\_t}; and \ci{Ts...} is a parameter pack of one or more types. Note that \ci{T} may be the special type \ci{none}. The current attribute is always a tuple (call it \ci{Tup}), so the "current attribute \ci{X}" refers to the last element of \ci{Tup}, not \ci{Tup} itself, except for those rules that explicitly mention \ci{boost::parser::tuple<>} as part of \ci{X}'s type.

\begin{itemize}
\item
  \ci{none\ >>\ T\ ->\ T}
\item
  \ci{CHAR} \textgreater> \ci{CHAR} -> \ci{std::string}
\item
  \ci{T\ >>\ none\ ->\ T}
\item
  \ci{C<T>\ >>\ T\ ->\ C<T>}
\item
  \ci{T\ >>\ C<T>\ ->\ C<T>}
\item
  \ci{C<T>\ >>\ optional<T>\ ->\ C<T>}
\item
  \ci{optional<T>\ >>\ C<T>\ ->\ C<T>}
\item
  \ci{boost::parser::tuple<none>\ >>\ T\ ->\ boost::parser::tuple<T>}
\item
  \ci{boost::parser::tuple<Ts...>\ >>\ T\ ->\ boost::parser::tuple<Ts...,\ T>}
\end{itemize}

The rules that combine containers with (possibly optional) adjacent values (e.g. \ci{C<T>\ >>\ optional<T>\ ->\ C<T>}) have a special case for strings. If \ci{C<T>} is exactly \ci{std::string}, and \ci{T} is either \ci{char} or \ci{char32\_t}, the combination yields a \ci{std::string}.

Again, if the final result is that the attribute is \ci{boost::parser::tuple<T>}, the attribute becomes \ci{T}.



\subsection{Alternative parser attribute rules}

The rules for alternative parsers are much simpler. For an alternative parer \ci{p}, let the list of attribute types for the subparsers of \ci{p} be \ci{a0,\ a1,\ a2,\ ...,\ an}. The attribute of \ci{p} is \ci{std::variant<a0,\ a1,\ a2,\ ...,\ an>}, with the following steps applied:

\begin{itemize}
\item
  all the \ci{none} attributes are left out, and if any are, the attribute is wrapped in a \ci{std::optional}, like \ci{std::optional<std::variant</*...*/>>};
\item
  duplicates in the \ci{std::variant} template parameters \ci{<T1,\ T2,\ ...\ Tn>} are removed; every type that appears does so exacly once;
\item
  if the attribute is \ci{std::variant<T>} or \ci{std::optional<std::variant<T>>}, the attribute becomes instead \ci{T} or \ci{std::optional<T>}, respectively; and
\item
  if the attribute is \ci{std::variant<>} or \ci{std::optional<std::variant<>>}, the result becomes \ci{none} instead.
\end{itemize}

\subsection{Formation of containers in attributes}

The rule for forming containers from non-containers is simple. You get a vector from any of the repeating parsers, like \ci{+p}, \ci{*p}, \ci{repeat(3){[}p{]}}, etc. The value type of the vector is \emph{\ci{ATTR}}\ci{(p)}.

Another rule for sequence containers is that a value \ci{x} and a container \ci{c} containing elements of \ci{x}'s type will form a single container. However, \ci{x}'s type must be exactly the same as the elements in \ci{c}. There is an exception to this in the special case for strings and characters noted above. For instance, consider the attribute of \ci{char\_\ >>\ string("str")}. In the non-Unicode code path, \ci{char\_}'s attribute type is guaranteed to be \ci{char}, so \emph{\ci{ATTR}}\ci{(char\_\ >>\ string("str"))} is \ci{std::string}. If you are parsing UTF-8 in the Unicode code path, \ci{char\_}'s attribute type is \ci{char32\_t}, and the special rule makes it also produce a \ci{std::string}. Otherwise, the attribute for \emph{\ci{ATTR}}\ci{(char\_\ >>\ string("str"))} would be \ci{boost::parser::tuple<char32\_t,\ std::string>}.

Again, there are no special rules for combining values and containers. Every combination results from an exact match, or fall into the string+character special case.

\subsection{{Another special case: \ci{std::string} assignment}{Another special case: std::string assignment}}

\ci{std::string} can be assigned from a \ci{char}. This is dumb. But, we're stuck with it. When you write a parser with a \ci{char} attribute, and you try to parse it into a \ci{std::string}, you've almost certainly made a mistake. More importantly, if you write this:

\begin{code}
namespace bp = boost::parser;
std::string result;
auto b = bp::parse("3", bp::int_, bp::ws, result);
\end{code}

... you are even more likely to have made a mistake. Though this should work, because the assignment in \ci{std::string\ s;\ s\ =\ 3;} is well-formed, Boost.Parser forbids it. If you write parsing code like the snippet above, you will get a static assertion. If you really do want to assign a \ci{float} or whatever to a \ci{std::string}, do it in a semantic action.

\subsection{Examples of attributes generated by sequence and alternative parsers}

In the table: \ci{a} is a semantic action; and \ci{p}, \ci{p1}, \ci{p2}, ... are parsers that generate attributes. Note that only \ci{>>} is used here; \ci{>} has the exact same attribute generation rules.

\textbf{Table~26.10.~Sequence and Alternative Combining Operations and Their Attributes}



\hfill\break

\subsection{Controlling attribute generation with merge{[}{]} and separate{[}{]}}

As we saw in the previous Parsing into \ci{struct}s and \ci{class}es section, if you parse two strings in a row, you get two separate strings in the resulting attribute. The parser from that example was this:

\begin{code}
namespace bp = boost::parser;
auto employee_parser = bp::lit("employee")
    >> '{'
    >> bp::int_ >> ','
    >> quoted_string >> ','
    >> quoted_string >> ','
    >> bp::double_
    >> '}';
\end{code}

\ci{employee\_parser}'s attribute is \ci{boost::parser::tuple<int,\ std::string,\ std::string,\ double>}. The two \ci{quoted\_string} parsers produce \ci{std::string} attributes, and those attributes are not combined. That is the default behavior, and it is just what we want for this case; we don't want the first and last name fields to be jammed together such that we can't tell where one name ends and the other begins. What if we were parsing some string that consisted of a prefix and a suffix, and the prefix and suffix were defined separately for reuse elsewhere?

\begin{code}
namespace bp = boost::parser;
auto prefix = /* ... */;
auto suffix = /* ... */;
auto special_string = prefix >> suffix;
// Continue to use prefix and suffix to make other parsers....
\end{code}

In this case, we might want to use these separate parsers, but want \ci{special\_string} to produce a single \ci{std::string} for its attribute. \ci{merge{[}{]}} exists for this purpose.

\begin{code}
namespace bp = boost::parser;
auto prefix = /* ... */;
auto suffix = /* ... */;
auto special_string = bp::merge[prefix >> suffix];
\end{code}

\ci{merge{[}{]}} only applies to sequence parsers (like \ci{p1\ >>\ p2}), and forces all subparsers in the sequence parser to use the same variable for their attribute.

Another directive, \ci{separate{[}{]}}, also applies only to sequence parsers, but does the opposite of \ci{merge{[}{]}}. If forces all the attributes produced by the subparsers of the sequence parser to stay separate, even if they would have combined. For instance, consider this parser.

\begin{code}
namespace bp = boost::parser;
auto string_and_char = +bp::char_('a') >> ' ' >> bp::cp;
\end{code}

\ci{string\_and\_char} matches one or more \ci{'{}a'{}}s, followed by some other character. As written above, \ci{string\_and\_char} produces a \ci{std::string}, and the final character is appended to the string, after all the \ci{'{}a'{}}s. However, if you wanted to store the final character as a separate value, you would use \ci{separate{[}{]}}.

\begin{code}
namespace bp = boost::parser;
auto string_and_char = bp::separate[+bp::char_('a') >> ' ' >> bp::cp];
\end{code}

With this change, \ci{string\_and\_char} produces the attribute \ci{boost::parser::tuple<std::string,\ char32\_t>}.

\subsection{merge{[}{]} and separate{[}{]} in more detail}

As mentioned previously, \ci{merge{[}{]}} applies only to sequence parsers. All subparsers must have the same attribute, or produce no attribute at all. At least one subparser must produce an attribute. When you use \ci{merge{[}{]}}, you create a \emph{combining group}. Every parser in a combining group uses the same variable for its attribute. No parser in a combining group interacts with the attributes of any parsers outside of its combining group. Combining groups are disjoint; \ci{merge{[}/*...*/{]}\ >>\ merge{[}/*...*/{]}} will produce a tuple of two attributes, not one.

\ci{separate{[}{]}} also applies only to sequence parsers. When you use \ci{separate{[}{]}}, you disable interaction of all the subparsers'{} attributes with adjacent attributes, whether they are inside or outside the \ci{separate{[}{]}} directive; you force each subparser to have a separate attribute.

The rules for \ci{merge{[}{]}} and \ci{separate{[}{]}} overrule the steps of the algorithm described above for combining the attributes of a sequence parser. Consider an example.

\begin{code}
namespace bp = boost::parser;
constexpr auto parser =
    bp::char_ >> bp::merge[(bp::string("abc") >> bp::char_ >> bp::char_) >> bp::string("ghi")];
\end{code}

You might think that \emph{\ci{ATTR}}\ci{(parser)} would be \ci{bp::tuple<char,\ std::string>}. It is not. The parser above does not even compile. Since we created a merge group above, we disabled the default behavior in which the \ci{char\_} parsers would have collapsed into the \ci{string} parser that preceded them. Since they are all treated as separate entities, and since they have different attribute types, the use of \ci{merge{[}{]}} is an error.

Many directives create a new parser out of the parser they are given. \ci{merge{[}{]}} and \ci{separate{[}{]}} do not. Since they operate only on sequence parsers, all they do is create a copy of the sequence parser they are given. The \ci{seq\_parser} template has a template parameter \ci{CombiningGroups}, and all \ci{merge{[}{]}} and \ci{separate{[}{]}} do is take a given \ci{seq\_parser} and create a copy of it with a different \ci{CombiningGroups} template parameter. This means that \ci{merge{[}{]}} and \ci{separate{[}{]}} are can be ignored in \ci{operator>>} expressions much like parentheses are. Consider an example.

\begin{code}
namespace bp = boost::parser;
constexpr auto parser1 = bp::separate[bp::int_ >> bp::int_] >> bp::int_;
constexpr auto parser2 = bp::lexeme[bp::int_ >> ' ' >> bp::int_] >> bp::int_;
\end{code}

Note that \ci{separate{[}{]}} is a no-op here; it's only being used this way for this example. These parsers have different attribute types. \emph{\ci{ATTR}}\ci{(parser1)} is \ci{boost::parser::tuple(int,\ int,\ int)}. \emph{\ci{ATTR}}\ci{(parser2)} is \ci{boost::parser::tuple(boost::parser::tuple(int,\ int),\ int)}. This is because \ci{bp::lexeme{[}{]}} wraps its given parser in a new parser. \ci{merge{[}{]}} does not. That's why, even though \ci{parser1} and \ci{parser2} look so structurally similar, they have different attributes.

\subsection{{\ci{transform(f){[}{]}}}{transform(f){[}{]}}}

\ci{transform(f){[}{]}} is a directive that transforms the attribute of a parser using the given function \ci{f}. For example:

\begin{code}
auto str_sum = [&](std::string const & s) {
    int retval = 0;
    for (auto ch : s) {
        retval += ch - '0';
    }
    return retval;
};

namespace bp = boost::parser;
constexpr auto parser = +bp::char_;
std::string str = "012345";

auto result = bp::parse(str, bp::transform(str_sum)[parser]);
assert(result);
assert(*result == 15);
static_assert(std::is_same_v<decltype(result), std::optional<int>>);
\end{code}

Here, we have a function \ci{str\_sum} that we use for \ci{f}. It assumes each character in the given \ci{std::string} \ci{s} is a digit, and returns the sum of all the digits in \ci{s}. Out parser \ci{parser} would normally return a \ci{std::string}. However, since \ci{str\_sum} returns a different type --- \ci{int} --- that is the attribute type of the full parser, \ci{bp::transform(by\_value\_str\_sum){[}parser{]}}, as you can see from the \ci{static\_assert}.

As is the case with attributes all throughout Boost.Parser, the attribute passed to \ci{f} will be moved. You can take it by \ci{const\ \&}, \ci{\&\&}, or by value.

No distinction is made between parsers with and without an attribute, because there is a Regular special no-attribute type that is generated by parsers with no attribute. You may therefore write something like \ci{transform(f){[}eps{]}}, and Boost.Parser will happily call \ci{f} with this special no-attribute type.

\subsection{Other directives that affect attribute generation}

\ci{omit{[}p{]}} disables attribute generation for the parser \ci{p}. \ci{raw{[}p{]}} changes the attribute from \emph{\ci{ATTR}}\ci{(p)} to a view that indicates the subrange of the input that was matched by \ci{p}. \ci{string\_view{[}p{]}} is just like \ci{raw{[}p{]}}, except that it produces \ci{std::basic\_string\_view}s. See Directives for details.

\section{{The \ci{parse()} API}{The parse() API}}

There are multiple top-level parse functions. They have some things in common:

\begin{itemize}
\item
  They each return a value contextually convertible to \ci{bool}.
\item
  They each take at least a range to parse and a parser. The "range to parse" may be an iterator/sentinel pair or an single range object.
\item
  They each require forward iterability of the range to parse.
\item
  They each accept any range with a character element type. This means that they can each parse ranges of \ci{char}, \ci{wchar\_t}, \ci{char8\_t}, \ci{char16\_t}, or \ci{char32\_t}.
\item
  The overloads with \ci{prefix\_} in their name take an iterator/sentinel pair. For example \ci{prefix\_parse(first,\ last,\ p,\ ws)}, which parses the range \ci{{[}first,\ last)}, advancing \ci{first} as it goes. If the parse succeeds, the entire input may or may not have been matched. The value of \ci{first} will indicate the last location within the input that \ci{p} matched. The \textbf{whole} input was matched if and only if \ci{first\ ==\ last} after the call to \ci{parse()}.
\item
  When you call any of the range overloads of \ci{parse()}, for example \ci{parse(r,\ p,\ ws)}, \ci{parse()} only indicates success if \textbf{all} of \ci{r} was matched by \ci{p}.
\end{itemize}

\begin{marker}[title=Note ]
\ci{wchar\_t} is an accepted value type for the input. Please note that this is interpreted as UTF-16 on MSVC, and UTF-32 everywhere else. 
\end{marker}

\subsection{The overloads}

There are eight overloads of \ci{parse()} and \ci{prefix\_parse()} combined, because there are three either/or options in how you call them.

\subsection{Iterator/sentinel versus range}

You can call \ci{prefix\_parse()} with an iterator and sentinel that delimit a range of character values. For example:

\begin{code}
namespace bp = boost::parser;
auto const p = /* some parser ... */;

char const * str_1 = /* ... */;
// Using null_sentinel, str_1 can point to three billion characters, and
// we can call prefix_parse() without having to find the end of the string first.
auto result_1 = bp::prefix_parse(str_1, bp::null_sentinel, p, bp::ws);

char str_2[] = /* ... */;
auto result_2 = bp::prefix_parse(std::begin(str_2), std::end(str_2), p, bp::ws);
\end{code}

The iterator/sentinel overloads can parse successfully without matching the entire input. You can tell if the entire input was matched by checking if \ci{first\ ==\ last} is true after \ci{prefix\_parse()} returns.

By contrast, you call \ci{parse()} with a range of character values. When the range is a reference to an array of characters, any terminating \ci{0} is ignored; this allows calls like \ci{parse("str",\ p)} to work naturally.

\begin{code}
namespace bp = boost::parser;
auto const p = /* some parser ... */;

std::u8string str_1 = "str";
auto result_1 = bp::parse(str_1, p, bp::ws);

// The null terminator is ignored.  This call parses s-t-r, not s-t-r-0.
auto result_2 = bp::parse(U"str", p, bp::ws);

char const * str_3 = "str";
auto result_3 = bp::parse(bp::null_term(str_3) | bp::as_utf16, p, bp::ws);
\end{code}

Since there is no way to indicate that \ci{p} matches the input, but only a prefix of the input was matched, the range (non-iterator/sentinel) overloads of \ci{parse()} indicate failure if the entire input is not matched.

\subsection{With or without an attribute out-parameter}

\begin{code}
namespace bp = boost::parser;
auto const p = '"' >> *(bp::char_ - '"') >> '"';
char const * str = "\"two words\"" ;

std::string result_1;
bool const success = bp::parse(str, p, result_1);   // success is true; result_1 is "two words"
auto result_2 = bp::parse(str, p);                  // !!result_2 is true; *result_2 is "two words"
\end{code}

When you call \ci{parse()} \textbf{with} an attribute out-parameter and parser \ci{p}, the expected type is \textbf{something like} \emph{\ci{ATTR}}\ci{(p)}. It doesn't have to be exactly that; I'll explain in a bit. The return type is \ci{bool}.

When you call \ci{parse()} \textbf{without} an attribute out-parameter and parser \ci{p}, the return type is \ci{std::optional<}\emph{\ci{ATTR}}\ci{(p)>}. Note that when \emph{\ci{ATTR}}\ci{(p)} is itself an \ci{optional}, the return type is \ci{std::optional<std::optional<...>>}. Each of those optionals tells you something different. The outer one tells you whether the parse succeeded. If so, the parser was successful, but it still generates an attribute that is an \ci{optional} --- that's the inner one.

\subsection{With or without a skipper}

\begin{code}
namespace bp = boost::parser;
auto const p = '"' >> *(bp::char_ - '"') >> '"';
char const * str = "\"two words\"" ;

auto result_1 = bp::parse(str, p);         // !!result_1 is true; *result_1 is "two words"
auto result_2 = bp::parse(str, p, bp::ws); // !!result_2 is true; *result_2 is "twowords"
\end{code}

\subsection{Compatibility of attribute out-parameters}

For any call to \ci{parse()} that takes an attribute out-parameter, like \ci{parse("str",\ p,\ bp::ws,\ out)}, the call is well-formed for a number of possible types of \ci{out}; \ci{decltype(out)} does not need to be exactly \emph{\ci{ATTR}}\ci{(p)}.

For instance, this is well-formed code that does not abort (remember that the attribute type of \ci{string()} is \ci{std::string}):

\begin{code}
namespace bp = boost::parser;
auto const p = bp::string("foo");

std::vector<char> result;
bool const success = bp::parse("foo", p, result);
assert(success && result == std::vector<char>({'f', 'o', 'o'}));
\end{code}

Even though \ci{p} generates a \ci{std::string} attribute, when it actually takes the data it generates and writes it into an attribute, it only assumes that the attribute is a \ci{container} (see Concepts), not that it is some particular container type. It will happily \ci{insert()} into a \ci{std::string} or a \ci{std::vector<char>} all the same. \ci{std::string} and \ci{std::vector<char>} are both containers of \ci{char}, but it will also insert into a container with a different element type. \ci{p} just needs to be able to insert the elements it produces into the attribute-container. As long as an implicit conversion allows that to work, everything is fine:

\begin{code}
namespace bp = boost::parser;
auto const p = bp::string("foo");

std::deque<int> result;
bool const success = bp::parse("foo", p, result);
assert(success && result == std::deque<int>({'f', 'o', 'o'}));
\end{code}

This works, too, even though it requires inserting elements from a generated sequence of \ci{char32\_t} into a container of \ci{char} (remember that the attribute type of \ci{+cp} is \ci{std::vector<char32\_t>}):

\begin{code}
namespace bp = boost::parser;
auto const p = +bp::cp;

std::string result;
bool const success = bp::parse("foo", p, result);
assert(success && result == "foo");
\end{code}

This next example works as well, even though the change to a container is not at the top level. It is an element of the result tuple:

\begin{code}
namespace bp = boost::parser;
auto const p = +(bp::cp - ' ') >> ' ' >> string("foo");

using attr_type = decltype(bp::parse(u8"", p));
static_assert(std::is_same_v<
              attr_type,
              std::optional<bp::tuple<std::string, std::string>>>);

using namespace bp::literals;

{
    // This is similar to attr_type, with the first std::string changed to a std::vector<int>.
    bp::tuple<std::vector<int>, std::string> result;
    bool const success = bp::parse(u8"rôle foo" | bp::as_utf8, p, result);
    assert(success);
    assert(bp::get(result, 0_c) == std::vector<int>({'r', U'ô', 'l', 'e'}));
    assert(bp::get(result, 1_c) == "foo");
}
{
    // This time, we have a std::vector<char> instead of a std::vector<int>.
    bp::tuple<std::vector<char>, std::string> result;
    bool const success = bp::parse(u8"rôle foo" | bp::as_utf8, p, result);
    assert(success);
    // The 4 code points "rôle" get transcoded to 5 UTF-8 code points to fit in the std::string.
    assert(bp::get(result, 0_c) == std::vector<char>({'r', (char)0xc3, (char)0xb4, 'l', 'e'}));
    assert(bp::get(result, 1_c) == "foo");
}
\end{code}

As indicated in the inline comments, there are a couple of things to take away from this example:

\begin{itemize}
\item
  If you change an attribute out-param (such as \ci{std::string} to \ci{std::vector<int>}, or \ci{std::vector<char32\_t>} to \ci{std::deque<int>}), the call to \ci{parse()} will often still be well-formed.
\item
  When changing out a container type, if both containers contain character values, the removed container's element type is \ci{char32\_t} (or \ci{wchar\_t} for non-MSVC builds), and the new container's element type is \ci{char} or \ci{char8\_t}, Boost.Parser assumes that this is a UTF-32-to-UTF-8 conversion, and silently transcodes the data when inserting into the new container.
\end{itemize}

Let's look at a case where another simple-seeming type replacement does \textbf{not} work. First, the case that works:

\begin{code}
namespace bp = boost::parser;
auto parser = -(bp::char_ % ',');
std::vector<int> result;
auto b = bp::parse("a, b", parser, bp::ws, result);
\end{code}

\emph{\ci{ATTR}}\ci{(parser)} is \ci{std::optional<std::string>}. Even though we pass a \ci{std::vector<int>}, everything is fine. However, if we modify this case only sightly, so that the \ci{std::optional<std::string>} is nested within the attribute, the code becomes ill-formed.

\begin{code}
struct S
{
    std::vector<int> chars;
    int i;
};
namespace bp = boost::parser;
auto parser = -(bp::char_ % ',') >> bp::int_;
S result;
auto b = bp::parse("a, b 42", parser, bp::ws, result);
\end{code}

If we change \ci{chars} to a \ci{std::vector<char>}, the code is still ill-formed. Same if we change \ci{chars} to a \ci{std::string}. We must actually use \ci{std::optional<std::string>} exactly to make the code well-formed again.

The reason the same looseness from the top-level parser does not apply to a nested parser is that, at some point in the code, the parser \ci{-(bp::char\_\ \%\ '{},'{})} would try to assign a \ci{std::optional<std::string>} --- the element type of the attribute type it normally generates --- to a \ci{chars}. If there's no implicit conversion there, the code is ill-formed.

The take-away for this last example is that the ability to arbitrarily swap out data types within the type of the attribute you pass to \ci{parse()} is very flexible, but is also limited to structurally simple cases. When we discuss \ci{rules} in the next section, we'll see how this flexibility in the types of attributes can help when writing complicated parsers.

Those were examples of swapping out one container type for another. They make good examples because that is more likely to be surprising, and so it's getting lots of coverage here. You can also do much simpler things like parse using a \ci{uint\_}, and writing its attribute into a \ci{double}. In general, you can swap any type \ci{T} out of the attribute, as long as the swap would not result in some ill-formed assignment within the parse.

Here is another example that also produces surprising results, for a different reason.

\begin{code}
namespace bp = boost::parser;
constexpr auto parser = bp::char_('a') >> bp::char_('b') >> bp::char_('c') |
                        bp::char_('x') >> bp::char_('y') >> bp::char_('z');
std::string str = "abc";
bp::tuple<char, char, char> chars;
bool b = bp::parse(str, parser, chars);
assert(b);
assert(chars == bp::tuple('c', '\0', '\0'));
\end{code}

This looks wrong, but is expected behavior. At every stage of the parse that produces an attribute, Boost.Parser tries to assign that attribute to some part of the out-param attribute provided to \ci{parse()}, if there is one. Note that \emph{\ci{ATTR}}\ci{(parser)} is \ci{std::string}, because each sequence parser is three \ci{char\_} parsers in a row, which forms a \ci{std::string}; there are two such alternatives, so the overall attribute is also \ci{std::string}. During the parse, when the first parser \ci{bp::char\_('{}a'{})} matches the input, it produces the attribute \ci{'{}a'{}} and needs to assign it to its destination. Some logic inside the sequence parser indicates that this \ci{'{}a'{}} contributes to the value in the \ci{0}th position in the result tuple, if the result is being written into a tuple. Here, we passed a \ci{bp::tuple<char,\ char,\ char>}, so it writes \ci{'{}a'{}} into the first element. Each subsequent \ci{char\_} parser does the same thing, and writes over the first element. If we had passed a \ci{std::string} as the out-param instead, the logic would have seen that the out-param attribute is a string, and would have appended \ci{'{}a'{}} to it. Then each subsequent parser would have appended to the string.

Boost.Parser never looks at the arity of the tuple passed to \ci{parse()} to see if there are too many or too few elements in it, compared to the expected attribute for the parser. In this case, there are two extra elements that are never touched. If there had been too few elements in the tuple, you would have seen a compilation error. The reason that Boost.Parser never does this kind of type-checking up front is that the loose assignment logic is spread out among the individual parsers; the top-level parse can determine what the expected attribute is, but not whether a passed attribute of another type is a suitable stand-in.

\subsection{{Compatibility of \ci{variant} attribute out-parameters}{Compatibility of variant attribute out-parameters}}

The use of a variant in an out-param is compatible if the default attribute can be assigned to the \ci{variant}. No other work is done to make the assignment compatible. For instance, this will work as you'd expect:

\begin{code}
namespace bp = boost::parser;
std::variant<int, double> v;
auto b = bp::parse("42", bp::int_, v);
assert(b);
assert(v.index() == 0);
assert(std::get<0>(v) == 42);
\end{code}

Again, this works because \ci{v\ =\ 42} is well-formed. However, other kinds of substitutions will not work. In particular, the \ci{boost::parser::tuple} to aggregate or aggregate to \ci{boost::parser::tuple} transformations will not work. Here's an example.

\begin{code}
struct key_value
{
    int key;
    double value;
};

namespace bp = boost::parser;
std::variant<key_value, double> kv_or_d;
key_value kv;
bp::parse("42 13.0", bp::int_ >> bp::double_, kv);      // Ok.
bp::parse("42 13.0", bp::int_ >> bp::double_, kv_or_d); // Error: ill-formed!
\end{code}

In this case, it would be easy for Boost.Parser to look at the alternative types covered by the variant, and do a conversion. However, there are many cases in which there is no obviously correct variant alternative type, or in which the user might expect one variant alternative type and get another. Consider a couple of cases.

\begin{code}
struct i_d { int i; double d; };
struct d_i { double d; int i; };
using v1 = std::variant<i_d, d_i>;

struct i_s { int i; short s; };
struct d_d { double d1; double d2; };
using v2 = std::variant<i_s, d_d>;

using tup_t = boost::parser::tuple<short, short>;
\end{code}

If we have a parser that produces a \ci{tup\_t}, and we have a \ci{v1} attribute out-param, the correct variant alternative type clearly does not exist --- this case is ambiguous, and anyone can see that neither variant alternative is a better match. If we were assigning a \ci{tup\_t} to \ci{v2}, it's even worse. The same ambiguity exists, but to the user, \ci{i\_s} is clearly "closer" than \ci{d\_d}.

So, Boost.Parser only does assignment. If some parser \ci{P} generates a default attribute that is not assignable to a variant alternative that you want to assign it to, you can just create a \ci{rule} that creates either an exact variant alternative type, or the variant itself, and use \ci{P} as your rule's parser.

\subsection{Unicode versus non-Unicode parsing}

A call to \ci{parse()} either considers the entire input to be in a UTF format (UTF-8, UTF-16, or UTF-32), or it considers the entire input to be in some unknown encoding. Here is how it deduces which case the call falls under:

\begin{itemize}
\item
  If the range is a sequence of \ci{char8\_t}, or if the input is a \ci{boost::parser::utf8\_view}, the input is UTF-8.
\item
  Otherwise, if the value type of the range is \ci{char}, the input is in an unknown encoding.
\item
  Otherwise, the input is in a UTF encoding.
\end{itemize}

\begin{marker}[title=Tip ]
if you want to want to parse in ASCII-only mode, or in some other non-Unicode encoding, use only sequences of \ci{char}, like \ci{std::string} or \ci{char\ const\ *}. 
\end{marker}

\begin{marker}[title=Tip ]
If you want to ensure all input is parsed as Unicode, pass the input range \ci{r} as \ci{r\ \textbar{}\ boost::parser::as\_utf32} --- that's the first thing that happens to it inside \ci{parse()} in the Unicode parsing path anyway. 
\end{marker}

\begin{marker}[title=Note ]
Since passing \ci{boost::parser::utfN\_view} is a special case, and since a sequence of \ci{char}s \ci{r} is otherwise considered an unknown encoding, \ci{boost::parser::parse(r\ \textbar{}\ boost::parser::as\_utf8,\ p)} treats \ci{r} as UTF-8, whereas \ci{boost::parser::parse(r,\ p)} does not. 
\end{marker}

\subsection{{The \ci{trace\_mode} parameter to parse()}{The trace\_mode parameter to parse()}}

Debugging parsers is notoriously difficult once they reach a certain size. To get a verbose trace of your parse, pass \ci{boost::parser::trace::on} as the final parameter to \ci{parse()}. It will show you the current parser being matched, the next few characters to be parsed, and any attributes generated. See the Error Handling and Debugging section of the tutorial for details.

\subsection{Globals and error handlers}

Each call to \ci{parse()} can optionally have a globals object associated with it. To use a particular globals object with you parser, you call \ci{with\_globals()} to create a new parser with the globals object in it:

\begin{code}
struct globals_t
{
    int foo;
    std::string bar;
};
auto const parser = /* ... */;
globals_t globals{42, "yay"};
auto result = boost::parser::parse("str", boost::parser::with_globals(parser, globals));
\end{code}

Every semantic action within that call to \ci{parse()} can access the same \ci{globals\_t} object using \ci{\_globals(ctx)}.

The default error handler is great for most needs, but if you want to change it, you can do so by creating a new parser with a call to \ci{with\_error\_handler()}:

\begin{code}
auto const parser = /* ... */;
my_error_handler error_handler;
auto result = boost::parser::parse("str", boost::parser::with_error_handler(parser, error_handler));
\end{code}

\begin{marker}[title=Tip ]
If your parsing environment does not allow you to report errors to a terminal, you may want to use \ci{callback\_error\_handler} instead of the default error handler. 
\end{marker}

\begin{marker}[title=Important ]
Globals and the error handler are ignored, if present, on any parser except the top-level parser. 
\end{marker}

\section{More About Rules}

In the earlier page about \ci{rules} (Rule Parsers), I described \ci{rules} as being analogous to functions. \ci{rules} are, at base, organizational. Here are the common use cases for \ci{rules}. Use a \ci{rule} if you want to:

\begin{itemize}
\item
  fix the attribute type produced by a parser to something other than the default;
\item
  create a parser that produces useful diagnostic text;
\item
  create a recursive rule (more on this below);
\item
  create a set of mutually-recursive parsers;
\item
  do callback parsing.
\end{itemize}

Let's look at the use cases in detail.

\subsection{Fixing the attribute type}

We saw in the previous section how \ci{parse()} is flexible in what types it will accept as attribute out-parameters. Here's another example.

\begin{code}
namespace bp = boost::parser;
auto result = bp::parse(input, bp::int % ',', result);
\end{code}

\ci{result} can be one of many different types. It could be \ci{std::vector<int>}. It could be \ci{std::set<long\ long>}. It could be a lot of things. Often, this is a very useful property; if you had to rewrite all of your parser logic because you changed the desired container in some part of your attribute from a \ci{std::vector} to a \ci{std::deque}, that would be annoying. However, that flexibility comes at the cost of type checking. If you want to write a parser that \textbf{always} produces exactly a \ci{std::vector<unsigned\ int>} \textbf{and no other type}, you also probably want a compilation error if you accidentally pass that parser a \ci{std::set<unsigned\ int>} attribute instead. There is no way with a plain parser to enforce that its attribute type may only ever be a single, fixed type.

Fortunately, \ci{rules} allow you to write a parser that has a fixed attribute type. Every rule has a specific attribute type, provided as a template parameter. If one is not specified, the rule has no attribute. The fact that the attribute is a specific type allows you to remove attribute flexibility. For instance, say we have a rule defined like this:

\begin{code}
bp::rule<struct doubles, std::vector<double>> doubles = "doubles";
auto const doubles_def = bp::double_ % ',';
BOOST_PARSER_DEFINE_RULES(doubles);
\end{code}

You can then use it in a call to \ci{parse()}, and \ci{parse()} will return a \ci{std::optional<std::vector<double>>}:

\begin{code}
auto const result = bp::parse(input, doubles, bp::ws);
\end{code}

If you call \ci{parse()} with an attribute out-parameter, it must be exactly \ci{std::vector<double>}:

\begin{code}
std::vector<double> vec_result;
bp::parse(input, doubles, bp::ws, vec_result); // Ok.
std::deque<double> deque_result;
bp::parse(input, doubles, bp::ws, deque_result); // Ill-formed!
\end{code}

If we wanted to use a \ci{std::deque<double>} as the attribute type of our rule:

\begin{code}
// Attribute changed to std::deque<double>.
bp::rule<struct doubles, std::deque<double>> doubles = "doubles";
auto const doubles_def = bp::double_ % ',';
BOOST_PARSER_DEFINE_RULES(doubles);

int main()
{
    std::deque<double> deque_result;
    bp::parse(input, doubles, bp::ws, deque_result); // Ok.
}
\end{code}

The take-away here is that the attribute flexibility is still available, but only \textbf{within} the rule --- the parser \ci{bp::double\_\ \%\ '{},'{}} can parse into a \ci{std::vector<double>} or a \ci{std::deque<double>}, but the rule \ci{doubles} must parse into only the exact attribute it was declared to generate.

The reason for this is that, inside the rule parsing implementation, there is code something like this:

\begin{code}
using attr_t = ATTR(doubles_def);
attr_t attr;
parse(first, last, parser, attr);
attribute_out_param = std::move(attr);
\end{code}

Where \ci{attribute\_out\_param} is the attribute out-parameter we pass to \ci{parse()}. If that final move assignment is ill-formed, the call to \ci{parse()} is too.

You can also use rules to exploit attribute flexibility. Even though a rule reduces the flexibility of attributes it can generate, the fact that it is so easy to write a new rule means that we can use rules themselves to get the attribute flexibility we want across our code:

\begin{code}
namespace bp = boost::parser;

// We only need to write the definition once...
auto const generic_doubles_def = bp::double_ % ',';

bp::rule<struct vec_doubles, std::vector<double>> vec_doubles = "vec_doubles";
auto const & vec_doubles_def = generic_doubles_def; // ... and re-use it,
BOOST_PARSER_DEFINE_RULES(vec_doubles);

// Attribute changed to std::deque<double>.
bp::rule<struct deque_doubles, std::deque<double>> deque_doubles = "deque_doubles";
auto const & deque_doubles_def = generic_doubles_def; // ... and re-use it again.
BOOST_PARSER_DEFINE_RULES(deque_doubles);
\end{code}

Now we have one of each, and we did not have to copy any parsing logic that would have to be maintained in two places.

Sometimes, you need to create a rule to enforce a certain attribute type, but the rule's attribute is not constructible from its parser's attribute. When that happens, you'll need to write a semantic action.

\begin{code}
struct type_t
{
    type_t() = default;
    explicit type_t(double x) : x_(x) {}
    // etc.

    double x_;
};

namespace bp = boost::parser;

auto doubles_to_type = [](auto & ctx) {
    using namespace bp::literals;
    _val(ctx) = type_t(_attr(ctx)[0_c] * _attr(ctx)[1_c]);
};

bp::rule<struct type_tag, type_t> type = "type";
auto const type_def = (bp::double_ >> bp::double_)[doubles_to_type];
BOOST_PARSER_DEFINE_RULES(type);
\end{code}

For a rule \ci{R} and its parser \ci{P}, we do not need to write such a semantic action if:

- \emph{\ci{ATTR}}\ci{(R)} is an aggregate, and \emph{\ci{ATTR}}\ci{(P)} is a compatible tuple;

- \emph{\ci{ATTR}}\ci{(R)} is a tuple, and \emph{\ci{ATTR}}\ci{(P)} is a compatible aggregate;

- \emph{\ci{ATTR}}\ci{(R)} is a non-aggregate class type \ci{C}, and \emph{\ci{ATTR}}\ci{(P)} is a tuple whose elements can be used to construct \ci{C}; or

- \emph{\ci{ATTR}}\ci{(R)} and \emph{\ci{ATTR}}\ci{(P)} are compatible types.

The notion of "compatible" is defined in The \ci{parse()} API.

\subsection{Creating a parser for better diagnostics}

Each \ci{rule} has associated diagnostic text that Boost.Parser can use for failures of that rule. This is useful when the parse reaches a parse failure at an expectation point (see Expectation points). Let's say you have the following code defined somewhere.

\begin{code}
namespace bp = boost::parser;

bp::rule<struct value_tag> value =
    "an integer, or a list of integers in braces";

auto const ints = '{' > (value % ',') > '}';
auto const value_def = bp::int_ | ints;

BOOST_PARSER_DEFINE_RULES(value);
\end{code}

Notice the two expectation points. One before \ci{(value\ \%\ '{},'{})}, one before the final \ci{'{}\}'{}}. Later, you call parse in some input:

\begin{code}
bp::parse("{ 4, 5 a", value, bp::ws);
\end{code}

This runs should of the second expectation point, and produces output like this:

\begin{code}
1:7: error: Expected '}' here:
{ 4, 5 a
       ^
\end{code}

That's a pretty good error message. Here's what it looks like if we violate the earlier expectation:

\begin{code}
bp::parse("{ }", value, bp::ws);
\end{code}

\begin{code}
1:2: error: Expected an integer, or a list of integers in braces % ',' here:
{ }
  ^
\end{code}

Not nearly as nice. The problem is that the expectation is on \ci{(value\ \%\ '{},'{})}. So, even thought we gave \ci{value} reasonable dianostic text, we put the text on the wrong thing. We can introduce a new rule to put the diagnstic text in the right place.

\begin{code}
namespace bp = boost::parser;

bp::rule<struct value_tag> value =
    "an integer, or a list of integers in braces";
bp::rule<struct comma_values_tag> comma_values =
    "a comma-delimited list of integers";

auto const ints = '{' > comma_values > '}';
auto const value_def = bp::int_ | ints;
auto const comma_values_def = (value % ',');

BOOST_PARSER_DEFINE_RULES(value, comma_values);
\end{code}

Now when we call \ci{bp::parse("\{\ \}",\ value,\ bp::ws)} we get a much better message:

\begin{code}
1:2: error: Expected a comma-delimited list of integers here:
{ }
  ^
\end{code}

The \ci{rule} \ci{value} might be useful elsewhere in our code, perhaps in another parser. It's diagnostic text is appropriate for those other potential uses.

\subsection{Recursive rules}

It's pretty common to see grammars that include recursive rules. Consider this EBNF rule for balanced parentheses:

\begin{code}
<parens> ::= "" | ( "(" <parens> ")" )
\end{code}

We can try to write this using Boost.Parser like this:

\begin{code}
namespace bp = boost::parser;
auto const parens = '(' >> parens >> ')' | bp::eps;
\end{code}

We had to put the \ci{bp::eps} second, because Boost.Parser's parsing algorithm is greedy. Otherwise, it's just a straight transliteration. Unfortunately, it does not work. The code is ill-formed because you can't define a variable in terms of itself. Well you can, but nothing good comes of it. If we instead make the parser in terms of a forward-declared \ci{rule}, it works.

\begin{code}
namespace bp = boost::parser;
bp::rule<struct parens_tag> parens = "matched parentheses";
auto const parens_def = '(' >> parens > ')' | bp::eps;
BOOST_PARSER_DEFINE_RULES(parens);
\end{code}

Later, if we use it to parse, it does what we want.

\begin{code}
assert(bp::parse("(((())))", parens, bp::ws));
\end{code}

When it fails, it even produces nice diagnostics.

\begin{code}
bp::parse("(((()))", parens, bp::ws);
\end{code}

\begin{code}
1:7: error: Expected ')' here (end of input):
(((()))
       ^
\end{code}

Recursive \ci{rules} work differently from other parsers in one way: when re-entering the rule recursively, only the attribute variable (\ci{\_attr(ctx)} in your semantic actions) is unique to that instance of the rule. All the other state of the uppermost instance of that rule is shared. This includes the value of the rule (\ci{\_val(ctx)}), and the locals and parameters to the rule. In other words, \ci{\_val(ctx)} returns a reference to the \textbf{same object} in every instance of a recursive \ci{rule}. This is because each instance of the rule needs a place to put the attribute it generates from its parse. However, we only want a single return value for the uppermost rule; if each instance had a separate value in \ci{\_val(ctx)}, then it would be impossible to build up the result of a recursive rule step by step during in the evaluation of the recursive instantiations.

Also, consider this rule:

\begin{code}
namespace bp = boost::parser;
bp::rule<struct ints_tag, std::vector<int>> ints = "ints";
auto const ints_def = bp::int_ >> ints | bp::eps;
\end{code}

What is the default attribute type for ints\_def? It sure looks like \ci{std::optional<std::vector<int>>}. Inside the evaluation of \ci{ints}, Boost.Parser must evaluate \ci{ints\_def}, and then produce a \ci{std::vector<int>} --- the return type of \ci{ints} --- from it. How? How do you turn a \ci{std::optional<std::vector<int>>} into a \ci{std::vector<int>}? To a human, it seems obvious, but the metaprogramming that properly handles this simple example and the general case is certainly beyond me.

Boost.Parser has a specific semantic for what consitutes a recursive rule. Each rule has a tag type associated with it, and if Boost.Parser enters a rule with a certain tag \ci{Tag}, and the currently-evaluating rule (if there is one) also has the tag \ci{Tag}, then rule instance being entered is considered to be a recursion. No other situations are considered recursion. In particular, if you have rules \ci{Ra} and \ci{Rb}, and \ci{Ra} uses \ci{Rb}, which in turn used \ci{Ra}, the second use of \ci{Ra} is not considered recursion. \ci{Ra} and \ci{Rb} are of course mutually recursive, but neither is considered a "recursive rule" for purposes of getting a unique value, locals, and parameters.

\subsection{Mutually-recursive rules}

One of the advantages of using rules is that you can declare all your rules up front and then use them immediately afterward. This lets you make rules that use each other without introducing cycles:

\begin{code}
namespace bp = boost::parser;

// Assume we have some polymorphic type that can be an object/dictionary,
// array, string, or int, called `value_type`.

bp::rule<class string, std::string> const string = "string";
bp::rule<class object_element, bp::tuple<std::string, value_type>> const object_element = "object-element";
bp::rule<class object, value_type> const object = "object";
bp::rule<class array, value_type> const array = "array";
bp::rule<class value_tag, value_type> const value = "value";

auto const string_def = bp::lexeme['"' >> *(bp::char_ - '"') > '"'];
auto const object_element_def = string > ':' > value;
auto const object_def = '{'_l >> -(object_element % ',') > '}';
auto const array_def = '['_l >> -(value % ',') > ']';
auto const value_def = bp::int_ | bp::bool_ | string | array | object;

BOOST_PARSER_DEFINE_RULES(string, object_element, object, array, value);
\end{code}

Here we have a parser for a Javascript-value-like type \ci{value\_type}. \ci{value\_type} may be an array, which itself may contain other arrays, objects, strings, etc. Since we need to be able to parse objects within arrays and vice versa, we need each of those two parsers to be able to refer to each other.

\subsection{Callback parsing}

Only \ci{rules} can be callback parsers, so if you want to get attributes supplied to you via callbacks instead of somewhere in the middle of a giant attribute that represents the whole parse result, you need to use \ci{rules}. See Parsing JSON With Callbacks for an extended example of callback parsing.

\subsection{Accessors available in semantic actions on rules}

\subsection{\_val()}

Inside all of a rule's semantic actions, the expression \ci{\_val(ctx)} is a reference to the attribute that the rule generates. This can be useful when you want subparsers to build up the attribute in a specific way:

\begin{code}
namespace bp = boost::parser;
using namespace bp::literals;

bp::rule<class ints, std::vector<int>> const ints = "ints";
auto twenty_zeros = [](auto & ctx) { _val(ctx).resize(20, 0); };
auto push_back = [](auto & ctx) { _val(ctx).push_back(_attr(ctx)); };
auto const ints_def = "20-zeros"_l[twenty_zeros] | +bp::int_[push_back];
BOOST_PARSER_DEFINE_RULES(ints);
\end{code}

\begin{marker}[title=Tip ]
That's just an example. It's almost always better to do things without using semantic actions. We could have instead written \ci{ints\_def} as \ci{"20-zeros"\ >>\ bp::attr(std::vector<int>(20))\ \textbar{}\ +bp::int\_}, which has the same semantics, is a lot easier to read, and is a lot less code. 
\end{marker}

\subsection{Locals}

The \ci{rule} template takes another template parameter we have not discussed yet. You can pass a third parameter \ci{LocalState} to \ci{rule}, which will be defaulted csontructed by the \ci{rule}, and made available within semantic actions used in the rule as \ci{\_locals(ctx)}. This gives your rule some local state, if it needs it. The type of \ci{LocalState} can be anything regular. It could be a single value, a struct containing multiple values, or a tuple, among others.

\begin{code}
struct foo_locals
{
    char first_value = 0;
};

namespace bp = boost::parser;

bp::rule<class foo, int, foo_locals> const foo = "foo";

auto record_first = [](auto & ctx) { _locals(ctx).first_value = _attr(ctx); }
auto check_against_first = [](auto & ctx) {
    char const first = _locals(ctx).first_value;
    char const attr = _attr(ctx);
    if (attr == first)
        _pass(ctx) = false;
    _val(ctx) = (int(first) << 8) | int(attr);
};

auto const foo_def = bp::cu[record_first] >> bp::cu[check_against_first];
BOOST_PARSER_DEFINE_RULES(foo);
\end{code}

\ci{foo} matches the input if it can match two elements of the input in a row, but only if they are not the same value. Without locals, it's a lot harder to write parsers that have to track state as they parse.

\subsection{Parameters}

Sometimes, it is convenient to parameterize parsers. Consider these parsing rules from the YAML 1.2 spec:

\begin{code}
[80]
s-separate(n,BLOCK-OUT) ::= s-separate-lines(n)
s-separate(n,BLOCK-IN)  ::= s-separate-lines(n)
s-separate(n,FLOW-OUT)  ::= s-separate-lines(n)
s-separate(n,FLOW-IN)   ::= s-separate-lines(n)
s-separate(n,BLOCK-KEY) ::= s-separate-in-line
s-separate(n,FLOW-KEY)  ::= s-separate-in-line

[136]
in-flow(n,FLOW-OUT)  ::= ns-s-flow-seq-entries(n,FLOW-IN)
in-flow(n,FLOW-IN)   ::= ns-s-flow-seq-entries(n,FLOW-IN)
in-flow(n,BLOCK-KEY) ::= ns-s-flow-seq-entries(n,FLOW-KEY)
in-flow(n,FLOW-KEY)  ::= ns-s-flow-seq-entries(n,FLOW-KEY)

[137]
c-flow-sequence(n,c) ::= “[” s-separate(n,c)? in-flow(c)? “]”
\end{code}

YAML {[}137{]} says that the parsing should proceed into two YAML subrules, both of which have these \ci{n} and \ci{c} parameters. It is certainly possible to transliterate these YAML parsing rules to something that uses unparameterized Boost.Parser \ci{rules}, but it is quite painful to do so. It is better to use a parameterized rule.

You give parameters to a \ci{rule} by calling its \ci{with()} member. The values you pass to \ci{with()} are used to create a \ci{boost::parser::tuple} that is available in semantic actions attached to the rule, using \ci{\_params(ctx)}.

Passing parameters to \ci{rules} like this allows you to easily write parsers that change the way they parse depending on contextual data that they have already parsed.

Here is an implementation of YAML {[}137{]}. It also implements the two YAML rules used directly by {[}137{]}, rules {[}136{]} and {[}80{]}. The rules that \textbf{those} rules use are also represented below, but are implemented using only \ci{eps}, so that I don't have to repeat too much of the (very large) YAML spec.

\begin{code}
namespace bp = boost::parser;

// A type to represent the YAML parse context.
enum class context {
    block_in,
    block_out,
    block_key,
    flow_in,
    flow_out,
    flow_key
};

// A YAML value; no need to fill it in for this example.
struct value
{
    // ...
};

// YAML [66], just stubbed in here.
auto const s_separate_in_line = bp::eps;

// YAML [137].
bp::rule<struct c_flow_seq_tag, value> c_flow_sequence = "c-flow-sequence";
// YAML [80].
bp::rule<struct s_separate_tag> s_separate = "s-separate";
// YAML [136].
bp::rule<struct in_flow_tag, value> in_flow = "in-flow";
// YAML [138]; just eps below.
bp::rule<struct ns_s_flow_seq_entries_tag, value> ns_s_flow_seq_entries =
    "ns-s-flow-seq-entries";
// YAML [81]; just eps below.
bp::rule<struct s_separate_lines_tag> s_separate_lines = "s-separate-lines";

// Parser for YAML [137].
auto const c_flow_sequence_def =
    '[' >>
    -s_separate.with(bp::_p<0>, bp::_p<1>) >>
    -in_flow.with(bp::_p<0>, bp::_p<1>) >>
    ']';
// Parser for YAML [80].
auto const s_separate_def = bp::switch_(bp::_p<1>)
    (context::block_out, s_separate_lines.with(bp::_p<0>))
    (context::block_in, s_separate_lines.with(bp::_p<0>))
    (context::flow_out, s_separate_lines.with(bp::_p<0>))
    (context::flow_in, s_separate_lines.with(bp::_p<0>))
    (context::block_key, s_separate_in_line)
    (context::flow_key, s_separate_in_line);
// Parser for YAML [136].
auto const in_flow_def = bp::switch_(bp::_p<1>)
    (context::flow_out, ns_s_flow_seq_entries.with(bp::_p<0>, context::flow_in))
    (context::flow_in, ns_s_flow_seq_entries.with(bp::_p<0>, context::flow_in))
    (context::block_out, ns_s_flow_seq_entries.with(bp::_p<0>, context::flow_key))
    (context::flow_key, ns_s_flow_seq_entries.with(bp::_p<0>, context::flow_key));

auto const ns_s_flow_seq_entries_def = bp::eps;
auto const s_separate_lines_def = bp::eps;

BOOST_PARSER_DEFINE_RULES(
    c_flow_sequence,
    s_separate,
    in_flow,
    ns_s_flow_seq_entries,
    s_separate_lines);
\end{code}

YAML {[}137{]} (\ci{c\_flow\_sequence}) parses a list. The list may be empty, and must be surrounded by brackets, as you see here. But, depending on the current YAML context (the \ci{c} parameter to {[}137{]}), we may require certain spacing to be matched by \ci{s-separate}, and how sub-parser \ci{in-flow} behaves also depends on the current context.

In \ci{s\_separate} above, we parse differently based on the value of \ci{c}. This is done above by using the value of the second parameter to \ci{s\_separate} in a switch-parser. The second parameter is looked up by using \ci{\_p} as a parse argument.

\ci{in\_flow} does something similar. Note that \ci{in\_flow} calls its subrule by passing its first parameter, but using a fixed value for the second value. \ci{s\_separate} only passes its \ci{n} parameter conditionally. The point is that a rule can be used with and without \ci{.with()}, and that you can pass constants or parse arguments to \ci{.with()}.

With those rules defined, we could write a unit test for YAML {[}137{]} like this:

\begin{code}
auto const test_parser = c_flow_sequence.with(4, context::block_out);
auto result = bp::parse("[]", test_parser);
assert(result);
\end{code}

You could extend this with tests for different values of \ci{n} and \ci{c}. Obviously, in real tests, you parse actual contents inside the \ci{"{[}{]}"}, if the other rules were implemented, like {[}138{]}.

\subsection{The \_p variable template}

Getting at one of a rule's arguments and passing it as an argument to another parser can be very verbose. \ci{\_p} is a variable template that allows you to refer to the \ci{n}th argument to the current rule, so that you can, in turn, pass it to one of the rule's subparsers. Using this, \ci{foo\_def} above can be rewritten as:

\begin{code}
auto const foo_def = bp::repeat(bp::_p<0>)[' '_l];
\end{code}

Using \ci{\_p} can prevent you from having to write a bunch of lambdas that get each get an argument out of the parse context using \ci{\_params(ctx){[}0\_c{]}} or similar.

Note that \ci{\_p} is a parse argument (see The Parsers And Their Uses), meaning that it is an invocable that takes the context as its only parameter. If you want to use it inside a semantic action, you have to call it.

\subsection{Special forms of semantic actions usable within a rule}

Semantic actions in this tutorial are usually of the signature \ci{void\ (auto\ \&\ ctx)}. That is, they take a context by reference, and return nothing. If they were to return something, that something would just get dropped on the floor.

It is a pretty common pattern to create a rule in order to get a certain kind of value out of a parser, when you don't normally get it automatically. If I want to parse an \ci{int}, \ci{int\_} does that, and the thing that I parsed is also the desired attribute. If I parse an \ci{int} followed by a \ci{double}, I get a \ci{boost::parser::tuple} containing one of each. But what if I don't want those two values, but some function of those two values? I probably write something like this.

\begin{code}
struct obj_t { /* ... */ };
obj_t to_obj(int i, double d) { /* ... */ }

namespace bp = boost::parser;
bp::rule<struct obj_tag, obj_t> obj = "obj";
auto make_obj = [](auto & ctx) {
    using boost::hana::literals;
    _val(ctx) = to_obj(_attr(ctx)[0_c], _attr(ctx)[1_c]);
};
constexpr auto obj_def = (bp::int_ >> bp::double_)[make_obj];
\end{code}

That's fine, if a little verbose. However, you can also do this instead:

\begin{code}
namespace bp = boost::parser;
bp::rule<struct obj_tag, obj_t> obj = "obj";
auto make_obj = [](auto & ctx) {
    using boost::hana::literals;
    return to_obj(_attr(ctx)[0_c], _attr(ctx)[1_c]);
};
constexpr auto obj_def = (bp::int_ >> bp::double_)[make_obj];
\end{code}

Above, we return the value from a semantic action, and the returned value gets assigned to \ci{\_val(ctx)}.

Finally, you can provide a function that takes the individual elements of the attribute (if it's a tuple), and returns the value to assign to \ci{\_val(ctx)}:

\begin{code}
namespace bp = boost::parser;
bp::rule<struct obj_tag, obj_t> obj = "obj";
constexpr auto obj_def = (bp::int_ >> bp::double_)[to_obj];
\end{code}

More formally, within a rule, the use of a semantic action is determined as follows. Assume we have a function \ci{APPLY} that calls a function with the elements of a tuple, like \ci{std::apply}. For some context \ci{ctx}, semantic action \ci{action}, and attribute \ci{attr}, \ci{action} is used like this:

- \ci{\_val(ctx)\ =\ APPLY(action,\ std::move(attr))}, if that is well-formed, and \ci{attr} is a tuple of size 2 or larger;

- otherwise, \ci{\_val(ctx)\ =\ action(ctx)}, if that is well-formed;

- otherwise, \ci{action(ctx)}.

The first case does not pass the context to the action at all. The last case is the normal use of semantic actions outside of rules.

\section{Algorithms and Views That Use Parsers}

Unless otherwise noted, all the algorithms and views are constrained very much like the way the \ci{parse()} overloads are. The kinds of ranges, parsers, etc., that they accept are the same.

\subsection{boost::parser::search()}

As shown in The \ci{parse()} API, the two patterns of parsing in Boost.Parser are whole-parse and prefix-parse. When you want to find something in the middle of the range being parsed, there's no \ci{parse} API for that. You can of course make a simple parser that skips everything before what you're looking for.

\begin{code}
namespace bp = boost::parser;
constexpr auto parser = /* ... */;
constexpr auto middle_parser = bp::omit[*(bp::char_ - parser)] >> parser;
\end{code}

\ci{middle\_parser} will skip over everything, one \ci{char\_} at a time, as long as the next \ci{char\_} is not the beginning of a successful match of \ci{parser}. After this, control passes to \ci{parser} itself. Ok, so that's not too hard to write. If you need to parse something from the middle in order to generate attributes, this is what you should use.

However, it often turns out you only need to find some subrange in the parsed range. In these cases, it would be nice to turn this into a proper algorithm in the pattern of the ones in \ci{std::ranges}, since that's more idiomatic. \ci{boost::parser::search()} is that algorithm. It has very similar semantics to \ci{std::ranges::search}, except that it searches not for a match to an exact subrange, but to a match with the given parser. Like \ci{std::ranges::search()}, it returns a subrange (\ci{boost::parser::subrange} in C++17, \ci{std::ranges::subrange} in C++20 and later).

\begin{code}
namespace bp = boost::parser;
auto result = bp::search("aaXYZq", bp::lit("XYZ"), bp::ws);
assert(!result.empty());
assert(std::string_view(result.begin(), result.end() - result.begin()) == "XYZ");
\end{code}

Since \ci{boost::parser::search()} returns a subrange, whatever parser you give it produces no attribute. I wrote \ci{bp::lit("XYZ")} above; if I had written \ci{bp::string("XYZ")} instead, the result (and lack of \ci{std::string} construction) would not change.

As you can see above, one aspect of \ci{boost::parser::search()} differs intentionally from the conventions of the \ci{std::ranges} algorithms --- it accepts C-style strings, treating them as if they were proper ranges.

Also, \ci{boost::parser::search()} knows how to accommodate your iterator type. You can pass the C-style string \ci{"aaXYZq"} as in the example above, or \ci{"aaXYZq"\ \textbar{}\ bp::as\_utf32}, or \ci{"aaXYZq"\ \textbar{}\ bp::as\_utf8}, or even \ci{"aaXYZq"\ \textbar{}\ bp::as\_utf16}, and it will return a subrange whose iterators are the type that you passed as input, even though internally the iterator type might be something different (a UTF-8 -> UTF-32 transcoding iterator in Unicode parsing, as with all the \ci{\textbar{}\ bp::as\_utfN} examples above). As long as you pass a range to be parsed whose value type is \ci{char}, \ci{char8\_t}, \ci{char32\_t}, or that is adapted using some combination of \ci{as\_utfN} adaptors, this accommodation will operate correctly.

\ci{boost::parser::search()} has multiple overloads. You can pass a range or an iterator/sentinel pair, and you can pass a skip parser or not. That's four overloads. Also, all four overloads take an optional \ci{boost::parser::trace} parameter at the end. This is really handy for investigating why you're not finding something in the input that you expected to.

\subsection{boost::parser::search\_all}

\ci{boost::parser::search\_all} creates \ci{boost::parser::search\_all\_views}. \ci{boost::parser::search\_all\_view} is a \ci{std::views}-style view. It produces a range of subranges. Each subrange it produces is the next match of the given parser in the parsed range.

\begin{code}
namespace bp = boost::parser;
auto r = "XYZaaXYZbaabaXYZXYZ" | bp::search_all(bp::lit("XYZ"));
int count = 0;
// Prints XYZ XYZ XYZ XYZ.
for (auto subrange : r) {
    std::cout << std::string_view(subrange.begin(), subrange.end() - subrange.begin()) << " ";
    ++count;
}
std::cout << "\n";
assert(count == 4);
\end{code}

All the details called out in the subsection on \ci{boost::parser::search()} above apply to \ci{boost::parser::search\_all}: its parser produces no attributes; it accepts C-style strings as if they were ranges; and it knows how to get from the internally-used iterator type back to the given iterator type, in typical cases.

\ci{boost::parser::search\_all} can be called with, and \ci{boost::parser::search\_all\_view} can be constructed with, a skip parser or not, and you can always pass \ci{boost::parser::trace} at the end of any of their overloads.

\subsection{boost::parser::split}

\ci{boost::parser::split} creates \ci{boost::parser::split\_views}. \ci{boost::parser::split\_view} is a \ci{std::views}-style view. It produces a range of subranges of the parsed range split on matches of the given parser. You can think of \ci{boost::parser::split\_view} as being the complement of \ci{boost::parser::search\_all\_view}, in that \ci{boost::parser::split\_view} produces the subranges between the subranges produced by \ci{boost::parser::search\_all\_view}. \ci{boost::parser::split\_view} has very similar semantics to \ci{std::views::split\_view}. Just like \ci{std::views::split\_view}, \ci{boost::parser::split\_view} will produce empty ranges between the beginning/end of the parsed range and an adjacent match, or between adjacent matches.

\begin{code}
namespace bp = boost::parser;
auto r = "XYZaaXYZbaabaXYZXYZ" | bp::split(bp::lit("XYZ"));
int count = 0;
// Prints '' 'aa' 'baaba' '' ''.
for (auto subrange : r) {
    std::cout << "'" << std::string_view(subrange.begin(), subrange.end() - subrange.begin()) << "' ";
    ++count;
}
std::cout << "\n";
assert(count == 5);
\end{code}

All the details called out in the subsection on \ci{boost::parser::search()} above apply to \ci{boost::parser::split}: its parser produces no attributes; it accepts C-style strings as if they were ranges; and it knows how to get from the internally-used iterator type back to the given iterator type, in typical cases.

\ci{boost::parser::split} can be called with, and \ci{boost::parser::split\_view} can be constructed with, a skip parser or not, and you can always pass \ci{boost::parser::trace} at the end of any of their overloads.

\subsection{boost::parser::replace}

\begin{marker}[title=Important ]
\ci{boost::parser::replace} and \ci{boost::parser::replace\_view} are not available on MSVC in C++17 mode. 
\end{marker}

\ci{boost::parser::replace} creates \ci{boost::parser::replace\_views}. \ci{boost::parser::replace\_view} is a \ci{std::views}-style view. It produces a range of subranges from the parsed range \ci{r} and the given replacement range \ci{replacement}. Wherever in the parsed range a match to the given parser \ci{parser} is found, \ci{replacement} is the subrange produced. Each subrange of \ci{r} that does not match \ci{parser} is produced as a subrange as well. The subranges are produced in the order in which they occur in \ci{r}. Unlike \ci{boost::parser::split\_view}, \ci{boost::parser::replace\_view} does not produce empty subranges, unless \ci{replacement} is empty.

\begin{code}
namespace bp = boost::parser;
auto card_number = bp::int_ >> bp::repeat(3)['-' >> bp::int_];
auto rng = "My credit card number is 1234-5678-9012-3456." | bp::replace(card_number, "XXXX-XXXX-XXXX-XXXX");
int count = 0;
// Prints My credit card number is XXXX-XXXX-XXXX-XXXX.
for (auto subrange : rng) {
    std::cout << std::string_view(subrange.begin(), subrange.end() - subrange.begin());
    ++count;
}
std::cout << "\n";
assert(count == 3);
\end{code}

If the iterator types \ci{Ir} and \ci{Ireplacement} for the \ci{r} and \ci{replacement} ranges passed are identical (as in the example above), the iterator type for the subranges produced is \ci{Ir}. If they are different, an implementation-defined type is used for the iterator. This type is the moral equivalent of a \ci{std::variant<Ir,\ Ireplacement>}. This works as long as \ci{Ir} and \ci{Ireplacement} are compatible. To be compatible, they must have common reference, value, and rvalue reference types, as determined by \ci{std::common\_type\_t}. One advantage to this scheme is that the range of subranges represented by \ci{boost::parser::replace\_view} is easily joined back into a single range.

\begin{code}
namespace bp = boost::parser;
auto card_number = bp::int_ >> bp::repeat(3)['-' >> bp::int_];
auto rng = "My credit card number is 1234-5678-9012-3456." | bp::replace(card_number, "XXXX-XXXX-XXXX-XXXX") | std::views::join;
std::string replace_result;
for (auto ch : rng) {
    replace_result.push_back(ch);
}
assert(replace_result == "My credit card number is XXXX-XXXX-XXXX-XXXX.");
\end{code}

Note that we could \textbf{not} have written \ci{std::string\ replace\_result(r.begin(),\ r.end())}. This is ill-formed because the \ci{std::string} range constructor takes two iterators of the same type, but \ci{decltype(rng.end())} is a sentinel type different from \ci{decltype(rng.begin())}.

Though the ranges \ci{r} and \ci{replacement} can both be C-style strings, \ci{boost::parser::replace\_view} must know the end of \ci{replacement} before it does any work. This is because the subranges produced are all common ranges, and so if \ci{replacement} is not, a common range must be formed from it. If you expect to pass very long C-style strings to \ci{boost::parser::replace} and not pay to see the end until the range is used, don't.

\ci{ReplacementV} is constrained almost exactly the same as \ci{V}. \ci{V} must model \ci{parsable\_range} and \ci{std::ranges::viewable\_range}. \ci{ReplacementV} is the same, except that it can also be a \ci{std::ranges::input\_range}, whereas \ci{V} must be a \ci{std::ranges::forward\_range}.

You may wonder what happens when you pass a UTF-N range for \ci{r}, and a UTF-M range for \ci{replacement}. What happens in this case is silent transcoding of \ci{replacement} from UTF-M to UTF-N by the \ci{boost::parser::replace} range adaptor. This doesn't require memory allocation; \ci{boost::parser::replace} just slaps \ci{\textbar{}\ boost::parser::as\_utfN} onto \ci{replacement}. However, since Boost.Parser treats \ci{char} ranges as unknown encoding, \ci{boost::parser::replace} will not transcode from \ci{char} ranges. So calls like this won't work:

\begin{code}
char const str[] = "some text";
char const replacement_str[] = "some text";
using namespace bp = boost::parser;
auto r = empty_str | bp::replace(parser, replacement_str | bp::as_utf8); // Error: ill-formed!  Can't mix plain-char inputs and UTF replacements.
\end{code}

This does not work, even though \ci{char} and UTF-8 are the same size. If \ci{r} and \ci{replacement} are both ranges of \ci{char}, everything will work of course. It's just mixing \ci{char} and UTF-encoded ranges that does not work.

All the details called out in the subsection on \ci{boost::parser::search()} above apply to \ci{boost::parser::replace}: its parser produces no attributes; it accepts C-style strings for the \ci{r} and \ci{replacement} parameters as if they were ranges; and it knows how to get from the internally-used iterator type back to the given iterator type, in typical cases.

\ci{boost::parser::replace} can be called with, and \ci{boost::parser::replace\_view} can be constructed with, a skip parser or not, and you can always pass \ci{boost::parser::trace} at the end of any of their overloads.

\subsection{boost::parser::transform\_replace}

\begin{marker}[title=Important ]
\ci{boost::parser::transform\_replace} and \ci{boost::parser::transform\_replace\_view} are not available on MSVC in C++17 mode. 
\end{marker}

\begin{marker}[title=Important ]
\ci{boost::parser::transform\_replace} and \ci{boost::parser::transform\_replace\_view} are not available on GCC in C++20 mode before GCC 12. 
\end{marker}

\ci{boost::parser::transform\_replace} creates \ci{boost::parser::transform\_replace\_views}. \ci{boost::parser::transform\_replace\_view} is a \ci{std::views}-style view. It produces a range of subranges from the parsed range \ci{r} and the given invocable \ci{f}. Wherever in the parsed range a match to the given parser \ci{parser} is found, let \ci{parser}'s attribute be \ci{attr}; \ci{f(std::move(attr))} is the subrange produced. Each subrange of \ci{r} that does not match \ci{parser} is produced as a subrange as well. The subranges are produced in the order in which they occur in \ci{r}. Unlike \ci{boost::parser::split\_view}, \ci{boost::parser::transform\_replace\_view} does not produce empty subranges, unless \ci{f(std::move(attr))} is empty. Here is an example.

\begin{code}
auto string_sum = [](std::vector<int> const & ints) {
    return std::to_string(std::accumulate(ints.begin(), ints.end(), 0));
};

auto rng = "There are groups of [1, 2, 3, 4, 5] in the set." |
           bp::transform_replace('[' >> bp::int_ % ',' >> ']', bp::ws, string_sum);
int count = 0;
// Prints "There are groups of 15 in the set".
for (auto subrange : rng) {
    for (auto ch : subrange) {
        std::cout << ch;
    }
    ++count;
}
std::cout << "\n";
assert(count == 3);
\end{code}

Let the type \ci{decltype(f(std::move(attr)))} be \ci{Replacement}. \ci{Replacement} must be a range, and must be compatible with \ci{r}. See the description of \ci{boost::parser::replace\_view}'s iterator compatibility requirements in the section above for details.

As with \ci{boost::parser::replace}, \ci{boost::parser::transform\_replace} can be flattened from a view of subranges into a view of elements by piping it to \ci{std::views::join}. See the section on \ci{boost::parser::replace} above for an example.

Just like \ci{boost::parser::replace} and \ci{boost::parser::replace\_view}, \ci{boost::parser::transform\_replace} and \ci{boost::parser::transform\_replace\_view} do silent transcoding of the result to the appropriate UTF, if applicable. If both \ci{r} and \ci{f(std::move(attr))} are ranges of \ci{char}, or are both the same UTF, no transcoding occurs. If one of \ci{r} and \ci{f(std::move(attr))} is a range of \ci{char} and the other is some UTF, the program is ill-formed.

\ci{boost::parser::transform\_replace\_view} will move each attribute into \ci{f}; \ci{f} may move from the argument or copy it as desired. \ci{f} may return an lvalue reference. If it does so, the address of the reference will be taken and stored within \ci{boost::parser::transform\_replace\_view}. Otherwise, the value returned by \ci{f} is moved into \ci{boost::parser::transform\_replace\_view}. In either case, the value type of \ci{boost::parser::transform\_replace\_view} is always a subrange.

\ci{boost::parser::transform\_replace} can be called with, and \ci{boost::parser::transform\_replace\_view} can be constructed with, a skip parser or not, and you can always pass \ci{boost::parser::trace} at the end of any of their overloads.

\section{Unicode Support}

Boost.Parser was designed from the start to be Unicode friendly. There are numerous references to the "Unicode code path" and the "non-Unicode code path" in the Boost.Parser documentation. Though there are in fact two code paths for Unicode and non-Unicode parsing, the code is not very different in the two code paths, as they are written generically. The only difference is that the Unicode code path parses the input as a range of code points, and the non-Unicode path does not. In effect, this means that, in the Unicode code path, when you call \ci{parse(r,\ p)} for some input range \ci{r} and some parser \ci{p}, the parse happens as if you called \ci{parse(r\ \textbar{}\ boost::parser::as\_utf32,\ p)} instead. (Of course, it does not matter if \ci{r} is a proper range, or an iterator/sentinel pair; those both work fine with \ci{boost::parser::as\_utf32}.)

Matching "characters" within Boost.Parser's parsers is assumed to be a code point match. In the Unicode path there is a code point from the input that is matched to each \ci{char\_} parser. In the non-Unicode path, the encoding is unknown, and so each element of the input is considered to be a whole "character" in the input encoding, analogous to a code point. From this point on, I will therefore refer to a single element of the input exclusively as a code point.

So, let's say we write this parser:

\begin{code}
constexpr auto char8_parser = boost::parser::char_('\xcc');
\end{code}

For any \ci{char\_} parser that should match a value or values, the type of the value to match is retained. So \ci{char8\_parser} contains a \ci{char} that it will use for matching. If we had written:

\begin{code}
constexpr auto char32_parser = boost::parser::char_(U'\xcc');
\end{code}

\ci{char32\_parser} would instead contain a \ci{char32\_t} that it would use for matching.

So, at any point during the parse, if \ci{char8\_parser} were being used to match a code point \ci{next\_cp} from the input, we would see the moral equivalent of \ci{next\_cp\ ==\ '{}\textbackslash{}xcc'{}}, and if \ci{char32\_parser} were being used to match \ci{next\_cp}, we'd see the equivalent of \ci{next\_cp\ ==\ U'{}\textbackslash{}xcc'{}}. The take-away here is that you can write \ci{char\_} parsers that match specific values, without worrying if the input is Unicode or not because, under the covers, what takes place is a simple comparison of two integral values.

\begin{marker}[title=Note ]
Boost.Parser actually promotes any two values to a common type using \ci{std::common\_type} before comparing them. This is almost always works because the input and any parameter passed to \ci{char\_} must be character types. 
\end{marker}

Since matches are always done at a code point level (remember, a "code point" in the non-Unicode path is assumed to be a single \ci{char}), you get different results trying to match UTF-8 input in the Unicode and non-Unicode code paths:

\begin{code}
namespace bp = boost::parser;

{
    std::string str = (char const *)u8"\xcc\x80"; // encodes the code point U+0300
    auto first = str.begin();

    // Since we've done nothing to indicate that we want to do Unicode
    // parsing, and we've passed a range of char to parse(), this will do
    // non-Unicode parsing.
    std::string chars;
    assert(bp::parse(first, str.end(), *bp::char_('\xcc'), chars));

    // Finds one match of the *char* 0xcc, because the value in the parser
    // (0xcc) was matched against the two code points in the input (0xcc and
    // 0x80), and the first one was a match.
    assert(chars == "\xcc");
}
{
    std::u8string str = u8"\xcc\x80"; // encodes the code point U+0300
    auto first = str.begin();

    // Since the input is a range of char8_t, this will do Unicode
    // parsing.  The same thing would have happened if we passed
    // str | boost::parser::as_utf32 or even str | boost::parser::as_utf8.
    std::string chars;
    assert(bp::parse(first, str.end(), *bp::char_('\xcc'), chars));

    // Finds zero matches of the *code point* 0xcc, because the value in
    // the parser (0xcc) was matched against the single code point in the
    // input, 0x0300.
    assert(chars == "");
}
\end{code}

\subsection{Implicit transcoding}

Additionally, it is expected that most programs will use UTF-8 for the encoding of Unicode strings. Boost.Parser is written with this typical case in mind. This means that if you are parsing 32-bit code points (as you always are in the Unicode path), and you want to catch the result in a container \ci{C} of \ci{char} or \ci{char8\_t} values, Boost.Parser will silently transcode from UTF-32 to UTF-8 and write the attribute into \ci{C}. This means that \ci{std::string}, \ci{std::u8string}, etc. are fine to use as attribute out-parameters for \ci{*char\_}, and the result will be UTF-8.

\begin{marker}[title=Note ]
UTF-16 strings as attributes are not supported directly. If you want to use UTF-16 strings as attributes, you may need to do so by transcoding a UTF-8 or UTF-32 attribute to UTF-16 within a semantic action. You can do this by using \ci{boost::parser::as\_utf16}. 
\end{marker}

The treatment of strings as UTF-8 is nearly ubiquitous within Boost.Parser. For instance, though the entire interface of \ci{symbols} uses \ci{std::string} or \ci{std::string\_view}, UTF-32 comparisons are used internally.

\subsection{Explicit transcoding}

I mentioned above that the use of \ci{boost::parser::utf*\_view} as the range to parse opts you in to Unicode parsing. Here's a bit more about these views and how best to use them.

If you want to do Unicode parsing, you're always going to be comparing code points at each step of the parse. As such, you're going to implicitly convert any parse input to UTF-32, if needed. This is what all the parse API functions do internally.

However, there are times when you have parse input that is a sequence of UTF-8-encoded \ci{char}s, and you want to do Unicode-aware parsing. As mentioned previously, Boost.Parser has a special case for \ci{char} inputs, and it will \textbf{not} assume that \ci{char} sequences are UTF-8. If you want to tell the parse API to do Unicode processing on them anyway, you can use the \ci{as\_utf32} range adapter. (Note that you can use any of the \ci{as\_utf*} adaptors and the semantics will not differ from the semantics below.)

\begin{code}
namespace bp = boost::parser;

auto const p = '"' >> *(bp::char_ - '"' - 0xb6) >> '"';
char const * str = "\"two wörds\""; // ö is two code units, 0xc3 0xb6

auto result_1 = bp::parse(str, p);                // Treat each char as a code point (typically ASCII).
assert(!result_1);
auto result_2 = bp::parse(str | bp::as_utf32, p); // Unicode-aware parsing on code points.
assert(result_2);
\end{code}

The first call to \ci{parse()} treats each \ci{char} as a code point, and since \ci{"ö"} is the pair of code units \ci{0xc3} \ci{0xb6}, the parse matches the second code unit against the \ci{-\ 0xb6} part of the parser above, causing the parse to fail. This happens because each code unit/\ci{char} in \ci{str} is treated as an independent code point.

The second call to \ci{parse()} succeeds because, when the parse gets to the code point for \ci{'{}ö'{}}, it is \ci{0xf6} (U+00F6), which does not match the \ci{-\ 0xb6} part of the parser.

The other adaptors \ci{as\_utf8} and \ci{as\_utf16} are also provided for completeness, if you want to use them. They each can transcode any sequence of character types.

\begin{marker}[title=Important ]
The \ci{as\_utfN} adaptors are optional, so they don't come with \ci{parser.hpp}. To get access to them, \ci{\#include\ <boost/parser/transcode\_view.hpp>}. 
\end{marker}

\subsection{(Lack of) normalization}

One thing that Boost.Parser does not handle for you is normalization; Boost.Parser is completely normalization-agnostic. Since all the parsers do their matching using equality comparisons of code points, you should make sure that your parsed range and your parsers all use the same normalization form.

\section{Callback Parsing}

In most parsing cases, being able to generate an attribute that represents the result of the parse, or being able to parse into such an attribute, is sufficient. Sometimes, it is not. If you need to parse a very large chunk of text, the generated attribute may be too large to fit in memory. In other cases, you may want to generate attributes sometimes, and not others. \ci{callback\_rules} exist for these kinds of uses. A \ci{callback\_rule} is just like a rule, except that it allows the rule's attribute to be returned to the caller via a callback, as long as the parse is started with a call to \ci{callback\_parse()} instead of \ci{parse()}. Within a call to \ci{parse()}, a \ci{callback\_rule} is identical to a regular \ci{rule}.

For a rule with no attribute, the signature of a callback function is \ci{void\ (tag)}, where \ci{tag} is the tag-type used when declaring the rule. For a rule with an attribute \ci{attr}, the signature is \ci{void\ (tag,\ attr)}. For instance, with this rule:

\begin{code}
boost::parser::callback_rule<struct foo_tag> foo = "foo";
\end{code}

this would be an appropriate callback function:

\begin{code}
void foo_callback(foo_tag)
{
    std::cout << "Parsed a 'foo'!\n";
}
\end{code}

For this rule:

\begin{code}
boost::parser::callback_rule<struct bar_tag, std::string> bar = "bar";
\end{code}

this would be an appropriate callback function:

\begin{code}
void bar_callback(bar_tag, std::string const & s)
{
    std::cout << "Parsed a 'bar' containing " << s << "!\n";
}
\end{code}

\begin{marker}[title=Important ]
In the case of \ci{bar\_callback()}, we don't need to do anything with \ci{s} besides insert it into a stream, so we took it as a \ci{const} lvalue reference. Boost.Parser moves all attributes into callbacks, so the signature could also have been \ci{void\ bar\_callback(bar\_tag,\ std::string\ s)} or \ci{void\ bar\_callback(bar\_tag,\ std::string\ \&\&\ s)}. 
\end{marker}

You opt into callback parsing by parsing with a call to \ci{callback\_parse()} instead of \ci{parse()}. If you use \ci{callback\_rules} with \ci{parse()}, they're just regular \ci{rules}. This allows you to choose whether to do "normal" attribute-generating/attribute-assigning parsing with \ci{parse()}, or callback parsing with \ci{callback\_parse()}, without rewriting much parsing code, if any.

The only reason all \ci{rules} are not \ci{callback\_rules} is that you may want to have some \ci{rules} use callbacks within a parse, and have some that do not. For instance, if you want to report the attribute of \ci{callback\_rule} \ci{r1} via callback, \ci{r1}'s implementation may use some rule \ci{r2} to generate some or all of its attribute.

See Parsing JSON With Callbacks for an extended example of callback parsing.

\section{Error Handling and Debugging}

\subsection{Error handling}

Boost.Parser has good error reporting built into it. Consider what happens when we fail to parse at an expectation point (created using \ci{operator>}). If I feed the parser from the Parsing JSON With Callbacks example a file called sample.json containing this input (note the unmatched \ci{'{}{[}'{}}):

\begin{code}
{
    "key": "value",
    "foo": [, "bar": []
}
\end{code}

This is the error message that is printed to the terminal:

\begin{code}
sample.json:3:12: error: Expected ']' here:
    "foo": [, "bar": []
            ^
\end{code}

That message is formatted like the diagnostics produced by Clang and GCC. It quotes the line on which the failure occurred, and even puts a caret under the exact position at which the parse failed. This error message is suitable for many kinds of end-users, and interoperates well with anything that supports Clang and/or GCC diagnostics.

Most of Boost.Parser's error handlers format their diagnostics this way, though you are not bound by that. You can make an error handler type that does whatever you want, as long as it meets the error handler interface.

The Boost.Parser error handlers are:

\begin{itemize}
\item
  \ci{default\_error\_handler}: Produces formatted diagnostics like the one above, and prints them to \ci{std::cerr}. \ci{default\_error\_handler} has no associated file name, and both errors and diagnostics are printed to \ci{std::cerr}. This handler is \ci{constexpr}-friendly.
\item
  \ci{stream\_error\_handler}: Produces formatted diagnostics. One or two streams may be used. If two are used, errors go to one stream and warnings go to the other. A file name can be associated with the parse; if it is, that file name will appear in all diagnostics.
\item
  \ci{callback\_error\_handler}: Produces formatted diagnostics. Calls a callback with the diagnostic message to report the diagnostic, rather than streaming out the diagnostic. A file name can be associated with the parse; if it is, that file name will appear in all diagnostics. This handler is useful for recording the diagnostics in memory.
\item
  \ci{rethrow\_error\_handler}: Does nothing but re-throw any exception that it is asked to handle. Its \ci{diagnose()} member functions are no-ops.
\item
  \ci{vs\_output\_error\_handler}: Directs all errors and warnings to the debugging output panel inside Visual Studio. Available on Windows only. Probably does nothing useful desirable when executed outside of Visual Studio.
\end{itemize}

You can set the error handler to any of these, or one of your own, using \ci{with\_error\_handler()} (see The \ci{parse()} API). If you do not set one, \ci{default\_error\_handler} will be used.

\subsection{How diagnostics are generated}

Boost.Parser only generates error messages like the ones in this page at failed expectation points, like \ci{a\ >\ b}, where you have successfully parsed \ci{a}, but then cannot successfully parse \ci{b}. This may seem limited to you. It's actually the best that we can do.

In order for error handling to happen other than at expectation points, we have to know that there is no further processing that might take place. This is true because Boost.Parser has \ci{P1\ \textbar{}\ P2\ \textbar{}\ ...\ \textbar{}\ Pn} parsers ("\ci{or\_parser}s"). If any one of these parsers \ci{Pi} fails to match, it is not allowed to fail the parse --- the next one (\ci{Pi+1}) might match. If we get to the end of the alternatives of the or\_parser and \ci{Pn} fails, we still cannot fail the top-level parse, because the \ci{or\_parser} might be a subparser within a parent \ci{or\_parser}.

Ok, so what might we do? Perhaps we could at least indicate when we ran into end-of-input. But we cannot, for exactly the same reason already stated. For any parser \ci{P}, reaching end-of-input is a failure for \ci{P}, but not necessarily for the whole parse.

Perhaps we could record the farthest point ever reached during the parse, and report that at the top level, if the top level parser fails. That would be little help without knowing which parser was active when we reached that point. This would require some sort of repeated memory allocation, since in Boost.Parser the progress point of the parser is stored exclusively on the stack --- by the time we fail the top-level parse, all those far-reaching stack frames are long gone. Not the best.

Worse still, knowing how far you got in the parse and which parser was active is not very useful. Consider this.

\begin{code}
namespace bp = boost::parser;
auto a_b = bp::char_('a') >> bp::char_('b');
auto c_b = bp::char_('c') >> bp::char_('b');
auto result = bp::parse("acb", a_b | c_b);
\end{code}

If we reported the farthest-reaching parser and it's position, it would be the \ci{a\_b} parser, at position \ci{"bc"} in the input. Is this really enlightening? Was the error in the input putting the \ci{'{}a'{}} at the beginning or putting the \ci{'{}c'{}} in the middle? If you point the user at \ci{a\_b} as the parser that failed, and never mention \ci{c\_b}, you are potentially just steering them in the wrong direction.

All error messages must come from failed expectation points. Consider parsing JSON. If you open a list with \ci{'{}{[}'{}}, you know that you're parsing a list, and if the list is ill-formed, you'll get an error message saying so. If you open an object with \ci{'{}\{'{}}, the same thing is possible --- when missing the matching \ci{'{}\}'{}}, you can tell the user, "That's not an object", and this is useful feedback. The same thing with a partially parsed number, etc. If the JSON parser does not build in expectations like matched braces and brackets, how can Boost.Parser know that a missing \ci{'{}\}'{}} is really a problem, and that no later parser will match the input even without the \ci{'{}\}'{}}?

\begin{marker}[title=Important ]
The bottom line is that you should build expectation points into your parsers using \ci{operator>} as much as possible. 
\end{marker}

\subsection{Using error handlers in semantic actions}

You can get access to the error handler within any semantic action by calling \ci{\_error\_handler(ctx)} (see The Parse Context). Any error handler must have the following member functions:

\begin{code}
template<typename Context, typename Iter>
void diagnose(
    diagnostic_kind kind,
    std::string_view message,
    Context const & context,
    Iter it) const;
\end{code}

\begin{code}
template<typename Context>
void diagnose(
    diagnostic_kind kind,
    std::string_view message,
    Context const & context) const;
\end{code}

If you call the second one, the one without the iterator parameter, it will call the first with \ci{\_where(context).begin()} as the iterator parameter. The one without the iterator is the one you will use most often. The one with the explicit iterator parameter can be useful in situations where you have messages that are related to each other, associated with multiple locations. For instance, if you are parsing XML, you may want to report that a close-tag does not match its associated open-tag by showing the line where the open-tag was found. That may of course not be located anywhere near \ci{\_where(ctx).begin()}. (A description of \ci{\_globals()} is below.)

\begin{code}
[](auto & ctx) {
    // Assume we have a std::vector of open tags, and another
    // std::vector of iterators to where the open tags were parsed, in our
    // globals.
    if (_attr(ctx) != _globals(ctx).open_tags.back()) {
        std::string open_tag_msg =
            "Previous open-tag \"" + _globals(ctx).open_tags.back() + "\" here:";
        _error_handler(ctx).diagnose(
            boost::parser::diagnostic_kind::error,
            open_tag_msg,
            ctx,
            _globals(ctx).open_tags_position.back());
        std::string close_tag_msg =
            "does not match close-tag \"" + _attr(ctx) + "\" here:";
        _error_handler(ctx).diagnose(
            boost::parser::diagnostic_kind::error,
            close_tag_msg,
            ctx);

        // Explicitly fail the parse.  Diagnostics do not affect parse success.
        _pass(ctx) = false;
    }
}
\end{code}

\subsection{\_report\_error() and \_report\_warning()}

There are also some convenience functions that make the above code a little less verbose, \ci{\_report\_error()} and \ci{\_report\_warning()}:

\begin{code}
[](auto & ctx) {
    // Assume we have a std::vector of open tags, and another
    // std::vector of iterators to where the open tags were parsed, in our
    // globals.
    if (_attr(ctx) != _globals(ctx).open_tags.back()) {
        std::string open_tag_msg =
            "Previous open-tag \"" + _globals(ctx).open_tags.back() + "\" here:";
        _report_error(ctx, open_tag_msg, _globals(ctx).open_tag_positions.back());
        std::string close_tag_msg =
            "does not match close-tag \"" + _attr(ctx) + "\" here:";
        _report_error(ctx, close_tag_msg);

        // Explicitly fail the parse.  Diagnostics do not affect parse success.
        _pass(ctx) = false;
    }
}
\end{code}

You should use these less verbose functions almost all the time. The only time you would want to use \ci{\_error\_handler()} directly is when you are using a custom error handler, and you want access to some part of its interface besides \ci{diagnose()}.

Though there is support for reporting warnings using the functions above, none of the error handlers supplied by Boost.Parser will ever report a warning. Warnings are strictly for user code.

For more information on the rest of the error handling and diagnostic API, see the header reference pages for \ci{error\_handling\_fwd.hpp} and \ci{error\_handling.hpp}.

\subsection{Creating your own error handler}

Creating your own error handler is pretty easy; you just need to implement three member functions. Say you want an error handler that writes diagnostics to a file. Here's how you might do that.

\begin{code}
struct logging_error_handler
{
    logging_error_handler() {}
    logging_error_handler(std::string_view filename) :
        filename_(filename), ofs_(filename_)
    {
        if (!ofs_)
            throw std::runtime_error("Could not open file.");
    }

    // This is the function called by Boost.Parser after a parser fails the
    // parse at an expectation point and throws a parse_error.  It is expected
    // to create a diagnostic message, and put it where it needs to go.  In
    // this case, we're writing it to a log file.  This function returns a
    // bp::error_handler_result, which is an enum with two enumerators -- fail
    // and rethrow.  Returning fail fails the top-level parse; returning
    // rethrow just re-throws the parse_error exception that got us here in
    // the first place.
    template<typename Iter, typename Sentinel>
    bp::error_handler_result
    operator()(Iter first, Sentinel last, bp::parse_error<Iter> const & e) const
    {
        bp::write_formatted_expectation_failure_error_message(
            ofs_, filename_, first, last, e);
        return bp::error_handler_result::fail;
    }

    // This function is for users to call within a semantic action to produce
    // a diagnostic.
    template<typename Context, typename Iter>
    void diagnose(
        bp::diagnostic_kind kind,
        std::string_view message,
        Context const & context,
        Iter it) const
    {
        bp::write_formatted_message(
            ofs_,
            filename_,
            bp::_begin(context),
            it,
            bp::_end(context),
            message);
    }

    // This is just like the other overload of diagnose(), except that it
    // determines the Iter parameter for the other overload by calling
    // _where(ctx).
    template<typename Context>
    void diagnose(
        bp::diagnostic_kind kind,
        std::string_view message,
        Context const & context) const
    {
        diagnose(kind, message, context, bp::_where(context).begin());
    }

    std::string filename_;
    mutable std::ofstream ofs_;
};
\end{code}

That's it. You just need to do the important work of the error handler in its call operator, and then implement the two overloads of \ci{diagnose()} that it must provide for use inside semantic actions. The default implementation of these is even available as the free function \ci{write\_formatted\_message()}, so you can just call that, as you see above. Here's how you might use it.

\begin{code}
int main()
{
    std::cout << "Enter a list of integers, separated by commas. ";
    std::string input;
    std::getline(std::cin, input);

    constexpr auto parser = bp::int_ >> *(',' > bp::int_);
    logging_error_handler error_handler("parse.log");
    auto const result = bp::parse(input, bp::with_error_handler(parser, error_handler));

    if (result) {
        std::cout << "It looks like you entered:\n";
        for (int x : *result) {
            std::cout << x << "\n";
        }
    }
}
\end{code}

We just define a \ci{logging\_error\_handler}, and pass it by reference to \ci{with\_error\_handler()}, which decorates the top-level parser with the error handler. We \textbf{could not} have written \ci{bp::with\_error\_handler(parser,\ logging\_error\_handler("parse.log"))}, because \ci{with\_error\_handler()} does not accept rvalues. This is becuse the error handler eventually goes into the parse context. The parse context only stores pointers and iterators, keeping it cheap to copy.

If we run the example and give it the input \ci{"1,"}, this shows up in the log file:

\begin{code}
parse.log:1:2: error: Expected int_ here (end of input):
1,
  ^
\end{code}

\subsection{Fixing ill-formed code}

Sometimes, during the writing of a parser, you make a simple mistake that is diagnosed horrifyingly, due to the high number of template instantiations between the line you just wrote and the point of use (usually, the call to \ci{parse()}). By "sometimes", I mean "almost always and many, many times". Boost.Parser has a workaround for situations like this. The workaround is to make the ill-formed code well-formed in as many circumstances as possible, and then do a runtime assert instead.

Usually, C++ programmers try whenever they can to catch mistakes as early as they can. That usually means making as much bad code ill-formed as possible. Counter-intuitively, this does not work well in parser combinator situations. For an example of just how dramatically different these two debugging scenarios can be with Boost.Parser, please see the very long discussion in the \ci{none} is weird section of Rationale.

If you are morally opposed to this approach, or just hate fun, good news: you can turn off the use of this technique entirely by defining \ci{BOOST\_PARSER\_NO\_RUNTIME\_ASSERTIONS}.

\subsection{Runtime debugging}

Debugging parsers is hard. Any parser above a certain complexity level is nearly impossible to debug simply by looking at the parser's code. Stepping through the parse in a debugger is even worse. To provide a reasonable chance of debugging your parsers, Boost.Parser has a trace mode that you can turn on simply by providing an extra parameter to \ci{parse()} or \ci{callback\_parse()}:

\begin{code}
boost::parser::parse(input, parser, boost::parser::trace::on);
\end{code}

Every overload of \ci{parse()} and \ci{callback\_parse()} takes this final parameter, which is defaulted to \ci{boost::parser::trace::off}.

If we trace a substantial parser, we will see a \textbf{lot} of output. Each code point of the input must be considered, one at a time, to see if a certain rule matches. An an example, let's trace a parse using the JSON parser from Parsing JSON. The input is \ci{"null"}. \ci{null} is one of the types that a Javascript value can have; the top-level parser in the JSON parser example is:

\begin{code}
auto const value_p_def =
    number | bp::bool_ | null | string | array_p | object_p;
\end{code}

So, a JSON value can be a number, or a Boolean, a \ci{null}, etc. During the parse, each alternative will be tried in turn, until one is matched. I picked \ci{null} because it is relatively close to the beginning of the \ci{value\_p\_def} alternative parser. Even so, the output is pretty huge. Let's break it down as we go:

\begin{code}
[begin value; input="null"]
\end{code}

Each parser is traced as \ci{{[}begin\ foo;\ ...{]}}, then the parsing operations themselves, and then \ci{{[}end\ foo;\ ...{]}}. The name of a rule is used as its name in the \ci{begin} and \ci{end} parts of the trace. Non-rules have a name that is similar to the way the parser looked when you wrote it. Most lines will have the next few code points of the input quoted, as we have here (\ci{input="null"}).

\begin{code}
[begin number | bool_ | null | string | ...; input="null"]
\end{code}

This shows the beginning of the parser \textbf{inside} the rule \ci{value} --- the parser that actually does all the work. In the example code, this parser is called \ci{value\_p\_def}. Since it isn't a rule, we have no name for it, so we show its implementation in terms of subparsers. Since it is a bit long, we don't print the entire thing. That's why that ellipsis is there.

\begin{code}
[begin number; input="null"]
  [begin raw[lexeme[ >> ...]][<<action>>]; input="null"]
\end{code}

Now we're starting to see the real work being done. \ci{number} is a somewhat complicated parser that does not match \ci{"null"}, so there's a lot to wade through when following the trace of its attempt to do so. One thing to note is that, since we cannot print a name for an action, we just print \ci{"<<action>>"}. Something similar happens when we come to an attribute that we cannot print, because it has no stream insertion operation. In that case, \ci{"<<unprintable-value>>"} is printed.

\begin{code}
    [begin raw[lexeme[ >> ...]]; input="null"]
      [begin lexeme[-char_('-') >> char_('1', '9') >> ... | ... >> ...]; input="null"]
        [begin -char_('-') >> char_('1', '9') >> *digit | char_('0') >> -(char_('.') >> ...) >> -( >> ...); input="null"]
          [begin -char_('-'); input="null"]
            [begin char_('-'); input="null"]
              no match
            [end char_('-'); input="null"]
            matched ""
            attribute: <<empty>>
          [end -char_('-'); input="null"]
          [begin char_('1', '9') >> *digit | char_('0'); input="null"]
            [begin char_('1', '9') >> *digit; input="null"]
              [begin char_('1', '9'); input="null"]
                no match
              [end char_('1', '9'); input="null"]
              no match
            [end char_('1', '9') >> *digit; input="null"]
            [begin char_('0'); input="null"]
              no match
            [end char_('0'); input="null"]
            no match
          [end char_('1', '9') >> *digit | char_('0'); input="null"]
          no match
        [end -char_('-') >> char_('1', '9') >> *digit | char_('0') >> -(char_('.') >> ...) >> -( >> ...); input="null"]
        no match
      [end lexeme[-char_('-') >> char_('1', '9') >> ... | ... >> ...]; input="null"]
      no match
    [end raw[lexeme[ >> ...]]; input="null"]
    no match
  [end raw[lexeme[ >> ...]][<<action>>]; input="null"]
  no match
[end number; input="null"]
[begin bool_; input="null"]
  no match
[end bool_; input="null"]
\end{code}

\ci{number} and \ci{boost::parser::bool\_} did not match, but \ci{null} will:

\begin{code}
[begin null; input="null"]
  [begin "null" >> attr(null); input="null"]
    [begin "null"; input="null"]
      [begin string("null"); input="null"]
        matched "null"
        attribute:
      [end string("null"); input=""]
      matched "null"
      attribute: null
\end{code}

Finally, this parser actually matched, and the match generated the attribute \ci{null}, which is a special value of the type \ci{json::value}. Since we were matching a string literal \ci{"null"}, earlier there was no attribute until we reached the \ci{attr(null)} parser.

\begin{code}
        [end "null"; input=""]
        [begin attr(null); input=""]
          matched ""
          attribute: null
        [end attr(null); input=""]
        matched "null"
        attribute: null
      [end "null" >> attr(null); input=""]
      matched "null"
      attribute: null
    [end null; input=""]
    matched "null"
    attribute: null
  [end number | bool_ | null | string | ...; input=""]
  matched "null"
  attribute: null
[end value; input=""]
--------------------
parse succeeded
--------------------
\end{code}

At the very end of the parse, the trace code prints out whether the top-level parse succeeded or failed.

Some things to be aware of when looking at Boost.Parser trace output:

\begin{itemize}
\item
  There are some parsers you don't know about, because they are not directly documented. For instance, \ci{p{[}a{]}} forms an \ci{action\_parser} containing the parser \ci{p} and semantic action \ci{a}. This is essentially an implementation detail, but unfortunately the trace output does not hide this from you.
\item
  For a parser \ci{p}, the trace-name may be intentionally different from the actual structure of \ci{p}. For example, in the trace above, you see a parser called simply \ci{"null"}. This parser is actually \ci{boost::parser::omit{[}boost::parser::string("null"){]}}, but what you typically write is just \ci{"null"}, so that's the name used. There are two special cases like this: the one described here for \ci{omit{[}string{]}}, and another for \ci{omit{[}char\_{]}}.
\item
  Since there are no other special cases for how parser names are printed, you may see parsers that are unlike what you wrote in your code. In the sections about the parsers and combining operations, you will sometimes see a parser or combining operation described in terms of an equivalent parser. For example, \ci{if\_(pred){[}p{]}} is described as "Equivalent to \ci{eps(pred)\ >>\ p}". In a trace, you will not see \ci{if\_}; you will see \ci{eps} and \ci{p} instead.
\item
  The values of arguments passed to parsers is printed whenever possible. Sometimes, a parse argument is not a value itself, but a callable that produces that value. In these cases, you'll see the resolved value of the parse argument.
\end{itemize}

\section{Memory Allocation}

Boost.Parser seldom allocates memory. The exceptions to this are:

\begin{itemize}
\item
  \ci{symbols} allocates memory for the symbol/attribute pairs it contains. If symbols are added during the parse, allocations must also occur then. The data structure used by \ci{symbols} is also a trie, which is a node-based tree. So, lots of allocations are likely if you use \ci{symbols}.
\item
  The error handlers that can take a file name allocate memory for the file name, if one is provided.
\item
  If trace is turned on by passing \ci{boost::parser::trace::on} to a top-level parsing function, the names of parsers are allocated.
\item
  When a failed expectation is encountered (using \ci{operator>}), the name of the failed parser is placed into a \ci{std::string}, which will usually cause an allocation.
\item
  \ci{string()}'s attribute is a \ci{std::string}, the use of which implies allocation. You can avoid this allocation by explicitly using a different string type for the attribute that does not allocate.
\item
  The attribute for \ci{repeat(p)} in all its forms, including \ci{operator*}, \ci{operator+}, and \ci{operator\%}, is \ci{std::vector<}\emph{\ci{ATTR}}\ci{(p)>}, the use of which implies allocation. You can avoid this allocation by explicitly using a different sequence container for the attribute that does not allocate. \ci{boost::container::static\_vector} or C++26's \ci{std::inplace\_vector} may be useful as such replacements.
\end{itemize}

With the exception of allocating the name of the parser that was expected in a failed expectation situation, Boost.Parser does not does not allocate unless you tell it to, by using \ci{symbols}, using a particular error\_handler, turning on trace, or parsing into attributes that allocate.

\section{Best Practices}

\subsection{Parse unicode from the start}

If you want to parse ASCII, using the Unicode parsing API will not actually cost you anything. Your input will be parsed, \ci{char} by \ci{char}, and compared to values that are Unicode code points (which are \ci{char32\_t}s). One caveat is that there may be an extra branch on each char, if the input is UTF-8. If your performance requirements can tolerate this, your life will be much easier if you just start with Unicode and stick with it.

Starting with Unicode support and UTF-8 input will allow you to properly handle unexpected input, like non-ASCII languages (that's most of them), with no additional effort on your part.

\subsection{Write rules, and test them in isolation}

Treat rules as the unit of work in your parser. Write a rule, test its corners, and then use it to build larger rules or parsers. This allows you to get better coverage with less work, since exercising all the code paths of your rules, one by one, keeps the combinatorial number of paths through your code manageable.

\subsection{Prefer auto-generated attributes to semantic actions}

There are multiple ways to get attributes out of a parser. You can:

\begin{itemize}
\item
  use whatever attribute the parser generates;
\item
  provide an attribute out-argument to \ci{parse()} for the parser to fill in;
\item
  use one or more semantic actions to assign attributes from the parser to variables outside the parser;
\item
  use callback parsing to provide attributes via callback calls.
\end{itemize}

All of these are fairly similar in how much effort they require, except for the semantic action method. For the semantic action approach, you need to have values to fill in from your parser, and keep them in scope for the duration of the parse.

It is much more straight forward, and leads to more reusable parsers, to have the parsers produce the attributes of the parse directly as a result of the parse.

This does not mean that you should never use semantic actions. They are sometimes necessary. However, you should default to using the other non-semantic action methods, and only use semantic actions with a good reason.

\subsection{If your parser takes end-user input, give rules names that you would want an end-user to see}

A typical error message produced by Boost.Parser will say something like, "Expected FOO here", where FOO is some rule or parser. Give your rules names that will read well in error messages like this. For instance, the JSON examples have these rules:

\begin{code}
bp::rule<class escape_seq, uint32_t> const escape_seq =
    "\\uXXXX hexadecimal escape sequence";
bp::rule<class escape_double_seq, uint32_t, double_escape_locals> const
    escape_double_seq = "\\uXXXX hexadecimal escape sequence";
bp::rule<class single_escaped_char, uint32_t> const single_escaped_char =
    "'\"', '\\', '/', 'b', 'f', 'n', 'r', or 't'";
\end{code}

Some things to note:

- \ci{escape\_seq} and \ci{escape\_double\_seq} have the same name-string. To an end-user who is trying to figure out why their input failed to parse, it doesn't matter which kind of result a parser rule generates. They just want to know how to fix their input. For either rule, the fix is the same: put a hexadecimal escape sequence there.

- \ci{single\_escaped\_char} has a terrible-looking name. However, it's not really used as a name anywhere per se. In error messages, it works nicely, though. The error will be "Expected '"', '', '/', 'b', 'f', 'n', 'r', or 't'{} here", which is pretty helpful.

\subsection{Have a simple test that you can run to find ill-formed-code-as-asserts}

Most of these errors are found at parser construction time, so no actual parsing is even necessary. For instance, a test case might look like this:

\begin{code}
TEST(my_parser_tests, my_rule_test) {
    my_rule r;
}
\end{code}

\section{Writing Your Own Parsers}

You should probably never need to write your own low-level parser. You have primitives like \ci{char\_} from which to build up the parsers that you need. It is unlikely that you're going to need to do things on a lower level than a single character.

However. Some people are obsessed with writing everything for themselves. We call them C++ programmers. This section is for them. However, this section is not an in-depth tutorial. It is a basic orientation to get you familiar enough with all the moving parts of writing a parser that you can then learn by reading the Boost.Parser code.

Each parser must provide two overloads of a function \ci{call()}. One overload parses, producing an attribute (which may be the special no-attribute type \ci{detail::nope}). The other one parses, filling in a given attribute. The type of the given attribute is a template parameter, so it can take any type that you can form a reference to.

Let's take a look at a Boost.Parser parser, \ci{opt\_parser}. This is the parser produced by use of \ci{operator-}. First, here is the beginning of its definition.

\begin{code}
template<typename Parser>
struct opt_parser
{
\end{code}

The end of its definition is:

\begin{code}
    Parser parser_;
};
\end{code}

As you can see, \ci{opt\_parser}'s only data member is the parser it adapts, \ci{parser\_}. Here is its attribute-generating overload to \ci{call()}.

\begin{code}
template<
    typename Iter,
    typename Sentinel,
    typename Context,
    typename SkipParser>
auto call(
    Iter & first,
    Sentinel last,
    Context const & context,
    SkipParser const & skip,
    detail::flags flags,
    bool & success) const
{
    using attr_t = decltype(parser_.call(
        first, last, context, skip, flags, success));
    detail::optional_of<attr_t> retval;
    call(first, last, context, skip, flags, success, retval);
    return retval;
}
\end{code}

First, let's look at the template and function parameters.

\begin{itemize}
\item
  \ci{Iter\ \&\ first} is the iterator. It is taken as an out-param. It is the responsibility of \ci{call()} to advance \ci{first} if and only if the parse succeeds.
\item
  \ci{Sentinel\ last} is the sentinel. If the parse has not yet succeeded within \ci{call()}, and \ci{first\ ==\ last} is \ci{true}, \ci{call()} must fail (by setting \ci{bool\ \&\ success} to \ci{false}).
\item
  \ci{Context\ const\ \&\ context} is the parse context. It will be some specialization of \ci{detail::parse\_context}. The context is used in any call to a subparser's \ci{call()}, and in some cases a new context should be created, and the new context passed to a subparser instead; more on that below.
\item
  \ci{SkipParser\ const\ \&\ skip} is the current skip parser. \ci{skip} should be used at the beginning of the parse, and in between any two uses of any subparser(s).
\item
  \ci{detail::flags\ flags} are a collection of flags indicating various things about the current state of the parse. \ci{flags} is concerned with whether to produce attributes at all; whether to apply the skip parser \ci{skip}; whether to produce a verbose trace (as when \ci{boost::parser::trace::on} is passed at the top level); and whether we are currently inside the utility function \ci{detail::apply\_parser}.
\item
  \ci{bool\ \&\ success} is the final function parameter. It should be set to \ci{true} if the parse succeeds, and \ci{false} otherwise.
\end{itemize}

Now the body of the function. Notice that it just dispatches to the other \ci{call()} overload. This is really common, since both overloads need to to the same parsing; only the attribute may differ. The first line of the body defines \ci{attr\_t}, the default attribute type of our wrapped parser \ci{parser\_}. It does this by getting the \ci{decltype()} of a use of \ci{parser\_.call()}. (This is the logic represented by \emph{\ci{ATTR}}\ci{()} in the rest of the documentation.) Since \ci{opt\_parser} represents an optional value, the natural type for its attribute is \ci{std::optional<}\emph{\ci{ATTR}}\ci{(parser)>}. However, this does not work for all cases. In particular, it does not work for the "no-attribute" type \ci{detail::nope}, nor for \ci{std::optional<T>} --- \emph{\ci{ATTR}}\ci{(-\/-p)} is just \emph{\ci{ATTR}}\ci{(-p)}. So, the second line uses an alias that takes care of those details, \ci{detail::optional\_of<>}. The third line just calls the other overload of \ci{call()}, passing \ci{retval} as the out-param. Finally, \ci{retval} is returned on the last line.

Now, on to the other overload.

\begin{code}
template<
    typename Iter,
    typename Sentinel,
    typename Context,
    typename SkipParser,
    typename Attribute>
void call(
    Iter & first,
    Sentinel last,
    Context const & context,
    SkipParser const & skip,
    detail::flags flags,
    bool & success,
    Attribute & retval) const
{
    [[maybe_unused]] auto _ = detail::scoped_trace(
        *this, first, last, context, flags, retval);

    detail::skip(first, last, skip, flags);

    if (!detail::gen_attrs(flags)) {
        parser_.call(first, last, context, skip, flags, success);
        success = true;
        return;
    }

    parser_.call(first, last, context, skip, flags, success, retval);
    success = true;
}
\end{code}

The template and function parameters here are identical to the ones from the other overload, except that we have \ci{Attribute\ \&\ retval}, our out-param.

Let's look at the implementation a bit at a time.

\begin{code}
[[maybe_unused]] auto _ = detail::scoped_trace(
    *this, first, last, context, flags, retval);
\end{code}

This defines a RAII trace object that will produce the verbose trace requested by the user if they passed \ci{boost::parser::trace::on} to the top-level parse. It only has effect if \ci{detail::enable\_trace(flags)} is \ci{true}. If trace is enabled, it will show the state of the parse at the point at which it is defined, and then again when it goes out of scope.

\begin{marker}[title=Important ]
For the tracing code to work, you must define an overload of \ci{detail::print\_parser} for your new parser type/template. See \ci{<boost/parser/detail/printing.hpp>} for examples. 
\end{marker}

\begin{code}
detail::skip(first, last, skip, flags);
\end{code}

This one is pretty simple; it just applies the skip parser. \ci{opt\_parser} only has one subparser, but if it had more than one, or if it had one that it applied more than once, it would need to repeat this line using \ci{skip} between every pair of uses of any subparser.

\begin{code}
if (!detail::gen_attrs(flags)) {
    parser_.call(first, last, context, skip, flags, success);
    success = true;
    return;
}
\end{code}

This path accounts for the case where we don't want to generate attributes at all, perhaps because this parser sits inside an \ci{omit{[}{]}} directive.

\begin{code}
parser_.call(first, last, context, skip, flags, success, retval);
success = true;
\end{code}

This is the other, typical, path. Here, we do want to generate attributes, and so we do the same call to \ci{parser\_.call()}, except that we also pass \ci{retval}.

Note that we set \ci{success} to \ci{true} after the call to \ci{parser\_.call()} in both code paths. Since \ci{opt\_parser} is zero-or-one, if the subparser fails, \ci{opt\_parse} still succeeds.

\subsection{When to make a new parse context}

Sometimes, you need to change something about the parse context before calling a subparser. For instance, \ci{rule\_parser} sets up the value, locals, etc., that are available for that rule. \ci{action\_parser} adds the generated attribute to the context (available as \ci{\_attr(ctx)}). Contexts are immutable in Boost.Parser. To "modify" one for a subparser, you create a new one with the appropriate call to \ci{detail::make\_context()}.

\subsection{{\ci{detail::apply\_parser()}}{detail::apply\_parser()}}

Sometimes a parser needs to operate on an out-param that is not exactly the same as its default attribute, but that is compatible in some way. To do this, it's often useful for the parser to call itself, but with slightly different parameters. \ci{detail::apply\_parser()} helps with this. See the out-param overload of \ci{repeat\_parser::call()} for an example. Note that since this creates a new scope for the ersatz parser, the \ci{scoped\_trace} object needs to know whether we're inside \ci{detail::apply\_parser} or not.

That's a lot, I know. Again, this section is not meant to be an in-depth tutorial. You know enough now that the parsers in \ci{parser.hpp} are at least readable.

\end{document}
